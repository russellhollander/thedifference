% Options for packages loaded elsewhere
\PassOptionsToPackage{unicode}{hyperref}
\PassOptionsToPackage{hyphens}{url}
%
\documentclass[
]{book}
\usepackage{amsmath,amssymb}
\usepackage{iftex}
\ifPDFTeX
  \usepackage[T1]{fontenc}
  \usepackage[utf8]{inputenc}
  \usepackage{textcomp} % provide euro and other symbols
\else % if luatex or xetex
  \usepackage{unicode-math} % this also loads fontspec
  \defaultfontfeatures{Scale=MatchLowercase}
  \defaultfontfeatures[\rmfamily]{Ligatures=TeX,Scale=1}
\fi
\usepackage{lmodern}
\ifPDFTeX\else
  % xetex/luatex font selection
\fi
% Use upquote if available, for straight quotes in verbatim environments
\IfFileExists{upquote.sty}{\usepackage{upquote}}{}
\IfFileExists{microtype.sty}{% use microtype if available
  \usepackage[]{microtype}
  \UseMicrotypeSet[protrusion]{basicmath} % disable protrusion for tt fonts
}{}
\makeatletter
\@ifundefined{KOMAClassName}{% if non-KOMA class
  \IfFileExists{parskip.sty}{%
    \usepackage{parskip}
  }{% else
    \setlength{\parindent}{0pt}
    \setlength{\parskip}{6pt plus 2pt minus 1pt}}
}{% if KOMA class
  \KOMAoptions{parskip=half}}
\makeatother
\usepackage{xcolor}
\usepackage{longtable,booktabs,array}
\usepackage{calc} % for calculating minipage widths
% Correct order of tables after \paragraph or \subparagraph
\usepackage{etoolbox}
\makeatletter
\patchcmd\longtable{\par}{\if@noskipsec\mbox{}\fi\par}{}{}
\makeatother
% Allow footnotes in longtable head/foot
\IfFileExists{footnotehyper.sty}{\usepackage{footnotehyper}}{\usepackage{footnote}}
\makesavenoteenv{longtable}
\usepackage{graphicx}
\makeatletter
\def\maxwidth{\ifdim\Gin@nat@width>\linewidth\linewidth\else\Gin@nat@width\fi}
\def\maxheight{\ifdim\Gin@nat@height>\textheight\textheight\else\Gin@nat@height\fi}
\makeatother
% Scale images if necessary, so that they will not overflow the page
% margins by default, and it is still possible to overwrite the defaults
% using explicit options in \includegraphics[width, height, ...]{}
\setkeys{Gin}{width=\maxwidth,height=\maxheight,keepaspectratio}
% Set default figure placement to htbp
\makeatletter
\def\fps@figure{htbp}
\makeatother
\setlength{\emergencystretch}{3em} % prevent overfull lines
\providecommand{\tightlist}{%
  \setlength{\itemsep}{0pt}\setlength{\parskip}{0pt}}
\setcounter{secnumdepth}{5}
\ifLuaTeX
  \usepackage{selnolig}  % disable illegal ligatures
\fi
\IfFileExists{bookmark.sty}{\usepackage{bookmark}}{\usepackage{hyperref}}
\IfFileExists{xurl.sty}{\usepackage{xurl}}{} % add URL line breaks if available
\urlstyle{same}
\hypersetup{
  pdftitle={The Difference},
  pdfauthor={I.G. Monson},
  hidelinks,
  pdfcreator={LaTeX via pandoc}}

\title{The Difference}
\usepackage{etoolbox}
\makeatletter
\providecommand{\subtitle}[1]{% add subtitle to \maketitle
  \apptocmd{\@title}{\par {\large #1 \par}}{}{}
}
\makeatother
\subtitle{A Popular Guide to Denominational History and Doctrine}
\author{I.G. Monson}
\date{1915}

\begin{document}
\maketitle

{
\setcounter{tocdepth}{1}
\tableofcontents
}
\hypertarget{preface}{%
\chapter*{Preface}\label{preface}}

What is the difference between the various Christian sects? The common answer is that there is no difference, that they all have the same Bible, that all are trying to reach the same place, etc. For the ignorant this is the easiest way to get away from this question. The lazy, indifferent, and the unbeliever will find the same answer the most convenient. But any one interested in truth and in Christianity will gladly accept information concerning the most important questions in life.

Of the many good works on the subject none are suited for general distribution among the several Lutheran church-bodies in this country. In the theological seminaries the works of Winer, Guericke, Guenther, and Gumlich are, of course, indispensible; but in the preparatory schools, parochial and Sunday-schools, as well as in the average Lutheran home, such learned works are impractical, if not wholly useless.

The aim has, therefore, been in the present volume to be more suggestive than exhaustive, to avoid all technicalities, and to treat the different dogmas in such a manner that the common, every-day Christian could understand them.

In order to attin this end, the dogmatic, positive, Biblical, Lutheran method was found most serviceable. And for this we make no apology, the cry of ``represtination of Reformation theology'' by rationalists and unionists to the contrary notwithstanding. As God at the last day will judge man according to His Word (John 12, 48), the only right way is to ask for the Old Paths, where is the good way, and walk therein (Jer. 6, 16).

Now and then names and phrases will be found inclosed in brackets. This has been done in order to stimulate further research by both teacher and student. Books of reference of some kind are now in nearly every home.

To write on Symbolism in our unionistic and irreligious age is, indeed, a thankless task. And yet \emph{the full truth was never popular}. Now more than in the last century the words of Stahl have full force: ``The powers that be are against us. the masses are against us. The tendency of the times is against us. The powerful errorists within the Church are against us.'' To this may here be added: The Reformed civilization under which we live is against us.

But God has a remnant yet to save. \emph{And it will be saved, but only through the Word of Truth}.

\textsc{The Author.}

\hypertarget{part-i.}{%
\chapter*{PART I.}\label{part-i.}}
\addcontentsline{toc}{chapter}{PART I.}

\hypertarget{origin-of-divided-protestantism}{%
\section{Origin of Divided Protestantism}\label{origin-of-divided-protestantism}}

When Luther in Germany opposed Roman Catholicism and popery, it was because the Roman Church had perverted the Gospel of Christ, taught righteousness by works, and prohibited the circulation of the Bible among the people. Strive as he would, his troubled soul could find no consolation in following the precepts of the Church. It was only after years of fruitless toil, in and out of the cloister, that he had at last found peace and consolation in the precious promise of the Lord: ``The just shall live by his faith'' (Hab. 2, 4)

The Swiss Reformers, Calvin and Zwingli, on the other hand, do not seem to have been thus troubled. Though earnest in their reform work, they worked more in the direction of emancipating their Churches from the yoke of the Roman organization, in eradicating superstition, correcting outward, flagrant abuses, and instilling into the minds of the people a spirit of independence, political as well as religious.

Luther contended that an effectual reformation must come from within; for if the tree is made good, the fruit will also be good. The Swiss made the branches, instead of the root, their point of attack.

Luther's weapon in the battle of the Reformation was the Word of God only. With this means, he said, the Savior repelled even Satan himself. With this means the apostles conquered the heathen world and laid it at the feet of Christ. To the Swiss this method was too slow; they were not content with letting the leaven of the Gospel work in its peculiar way; they must needs also use physical force and political power, a trait to this day characteristic of the Reformed Church.

Closely allied to this was the different views taken by the Reformers as to the value of the Scriptures. Luther contended that the Word of God was not merely a message to mankind, but the vehicle of the Holy Spirit through which He gives the proffered grace, a truth attested by the Apostle Paul, when he says that the Holy Spirit was given through the preaching of the Gospel (Gal. 3, 2), and faith through the Word of God (Rom. 10, 17).

Zwingli, on the other hand, believ ed that the Spirit needed no means, that He worked independently of the Scriptures. The preaching of the Word was to him only ``an outward sound,'' outside of which the Holy Spirit must work regeneration and salvation. Hence, he hoped ``that good men in all ages, who had walked uprightly according to the light they had, were also saved.'' (\emph{Expos. Fidei}, 2, 559.)

But the real difference between the Reformers came to light in their discussion concerning the authority of the Scriptures. Luther contended with the Apostle Paul that the natural man receiveth not the things of the Spirit of God, for they are foolishness unto him, neither can he know them, because they are spiritually discerned. (1 Cor. 2, 14) Hence, if he would not be adjudged wise in his own conceit (Rom. 12, 16), he must not think above that which is written (1 Cor. 4, 6), but bring into captivity every thought to the obedience of Christ (2 Cor. 10, 5). History as well as his own experience had taught him that it was futile to worship God, teaching for doctrines the commandments of men (Matt. 15, 9), and therefore he set up this formula in the Smalcald Articles: ``It is certain that the Word of God alone shall set up articles of faith, and none else, not even an angel.'' (II, 15.)

No doubt, the Swiss would have said Yes and Amen to this, adding the rationalistic phrase: ``rightly understood.'' (John Calvin declared that he had signed the Augsburg Confession in this manner.) They did not deny the authenticity of any part of the canonical Scriptures, but ever and anon they would follow human reason, and assert that the language of the Holy Spirit must not be taken in a literal, but in a figurative sense, as, for instance, in the doctrine of the Lord's Supper. Any other interpretation, they said, would be ``absurd'' and ``unreasonable.'' What becomes of the Scriptures under such treatment we shall observe as we enter more fully into our subject.

That such opposite views concerning the Scriptures should be followed by Different methods of church-work is not surprising.

To Luther only such changes seemed necessary as would promote true piety. Any ceremony, therefore, which had come down from antiquity, and was not connected with any idolatrous practice, or could be cleansed from such, was retained. Church interiors, with altars, pulpits, sacristies, organs, paintings, and statuary, were retained, also clerical vestments and the observance of many holidays; in short, anything and everything which could be used properly, and was not prohibited by the Word of God.

To the Swiss everything Catholic was ``a heathenish abomination.'' What the Scriptures did not directly command was to them unscriptural, and hence they not only countenanced the ruthless destruction of statuary and paintings in churches, but even counseled to do so. Church interiors were entirely remodeled, and ceremonies of the most simple kind instituted. This difference between the Reformed and the Lutheran position may best be illustrated by an anecdote. In discussing the retention or rejection of the old liturgies, Zwingli said, ``Show me a passage in the Scriptures wherein they are enjoined.'' To which Luther replied, ``Show me a passage wherein they are condemned.''

The different stand taken by the Reformers concerning Church and State must not be overlooked. The Lutheran Church soon became a State Church, it is true, but in spite of the teachings of Luther. ``It is with the Word we must contend,'' he observed, ``and by the Word we must refute and expel what has gained a footing by violence. I would not resort to force against such as are superstitious, nor even against unbelievers! Whosoever believeth let him draw nigh, and whoso believeth not, stand afar off. Let there be no compulsion. Liberty is of the very essence of faith. God does more by the simple power of His Word than you and I and the whole world could effect by all our efforts put together! God arrests the heart, and that once taken, all is won.''

To the Swiss Reformers the Jewish Church of the Old Testament was the only correct model. Zwingli looked upon the Council as representative both of Church and State, and the theocratic State Church organized by Calvin at Geneva is a fact well known to all students of history. Even in America the Reformed Church commenced as a State Church, and the roots of it are still sprouting such shoots as Sunday-laws, chaplains for conventions, Congress, legislatures, and for the army and navy, and the fanatical activities of the National Reform Association.

The Reformed Churchis split up into a multitude of sects. They have no confession to which they all subscribe; many of them have no confession at all, and others modify their beliefs as occasion demands. This is because a belief founded partly on ``reason,'' ``views,'' ``science,'' ``feeling,'' and the like, can have no stability. Views change; the rational way of one is irrational to another; the science of to-day is a fable to-morrow. In order, therefore, to gain recruits for this or that sect, innumerable schemes, methods, and measures are resorted to, -- camp-meetings, revivals, the Y. M. C. A., and of late the Laymen's Movements, Sociali Service, etc.

We shall now pass in review the principal denomination which have a following in the United States. Our next chapter briefly summarizes the history of these denominations, and succeeding chapters will take up, one at a time, the chief doctrines of Scripture, stating the agreement of our Lutheran Confessions with the same, and the differences which exist between us and the various denominations.

\hypertarget{historical-summaries}{%
\section{Historical Summaries}\label{historical-summaries}}

\hypertarget{the-anglican-church}{%
\subsection{THE ANGLICAN CHURCH}\label{the-anglican-church}}

Many dignitaries of the Anglican Church do not, at the present time, like to date the beginning of their denomination from the time of the Reformation. And some would like to ignore the Reformation altogether. Says one: ``England has never had but one religion; it has never changed that religion, and it has that religion still.''

From the sixth century until the year 1534, when Henry VIII was refused the sanction of the pope to a divorce from his first wife (he had six in all), England was as much a Catholic country as any other of the European states. In that year, to avenge himself on the pope, Henry ``denied the papal supremacy, and declared himself to be the one protector of the English Church, its only and supreme lord, and, so far as might be by the laws of Christ, its supreme head.'' And this Parliament later confirmed. This was the beginning of the revolt which carried England outside the Catholic fold.

Writings of Luther and, later, of the Swiss Reformers were early and eagerly read in England. Henry, who had written against Luther before the breach with the pope, continued in the Church, and, as far as Roman doctrine is concerned, died a Catholid; but when his son, Edward VI, ascended the throne, in 1547, reform ideas were so well rooted that Archbishop Thomas Cranmer, counseled mainly by Reformed theologians (Martin Bucer, Peter Martyr), in 1552 set up a confession of faith containing forty-two articles, later cut down to thirty-nine at the Synod of London, in 1562. This confession was ratified by Parliament in 1571, and thereby became a part of the law of the land. ``The Book of Common Prayer,'' which is still in use, contains not only this document, but also a full ritual for all occasions during the church-year.

The Thirty-nine Articles, the confession of the Anglican Church, are decidedly Reformed in tone and tendency, though in many respects taking a middle course between the Lutheran and the Reformed. But when Episcopal missionaries say to Lutheran immigrants that their Church is the Lutheran, excepting the language, they ought to know that they are not telling the truth. The Episcopal Church in the United States is only the Anglican Church without state-connections. It is officially known as The Protestant Episcopal Church. As such it dates from 1787, when it received its first bishop, Dr.~Seabury, and constituted itself independent of England.

In 1873, it suffered a small loss through a few priests, who left it on account of a ritualistic (Romish) tendency, which was rapidly gaining ground in the Church. They reduced the confession to thirty-five articles, and changed a few expressions here and there. They are known as The Reformed Episcopal Church.

Though composed of three more or less antagonistic elements, the ``high,'' the ``low,'' and the ``broad,'' the Episcopal Church has escaped any serious division, and has remained virtually one united Church up to this time. It is governed by bishops, -- hence its name, -- and believes in the so-called Apostolic Succession, an invention inherited from popery.

In late years the Episcopal Church has become very lax in many ways. Higher criticism and rationalism dominate its theologians, and ritualism opens the way back to Rome for many every year.

In many Episcopal churches, both in our country and in England, prayers are said for the dead, the saints are worshipped, mass is read, and auricular confession to the priest is practised. Indeed, nothing but the claims of the supremacy and infallibility of the pope seems to stand in the way of a large section of the Episcopal Church going bodily into the Roman Catholic communion.

\hypertarget{presbyterians}{%
\subsection{PRESBYTERIANS}\label{presbyterians}}

\hypertarget{unchristian-cults}{%
\section{Unchristian Cults}\label{unchristian-cults}}

How many sects there are in this country alone no one knows, not even the Director of the Census. Catholics would fain tell us that this state of affairs dates back only to the time of the Reformation. There may, at the present time, be more sects than formerly, but divisions in the Church have always existed, and the Pope of Rome has not only had his hands full in pacifying and judging between warring factions of monks and their so-called ``eminent teachers,'' but he has constantly had on hand a war of extermination against dissenters, who would not acquiesce in his teachings. The attitude of the pope towards the Modernists of our own day is only the old story over again. But their silence after his condemnation is the slence of the dead, -- and of the same value.

\hypertarget{part-ii.-doctrines-of-the-church}{%
\chapter*{PART II. Doctrines of the Church}\label{part-ii.-doctrines-of-the-church}}
\addcontentsline{toc}{chapter}{PART II. Doctrines of the Church}

\hypertarget{concerning-the-truth-of-the-christian-religion}{%
\section{Concerning the Truth of the Christian Religion}\label{concerning-the-truth-of-the-christian-religion}}

\hypertarget{the-holy-scriptures}{%
\section{The Holy Scriptures}\label{the-holy-scriptures}}

\hypertarget{original-sin}{%
\section{Original Sin}\label{original-sin}}

\hypertarget{redemption}{%
\section{Redemption}\label{redemption}}

\hypertarget{conversion}{%
\section{Conversion}\label{conversion}}

\hypertarget{predestination}{%
\section{Predestination}\label{predestination}}

\hypertarget{justification}{%
\section{Justification}\label{justification}}

\hypertarget{sanctification}{%
\section{Sanctification}\label{sanctification}}

\hypertarget{the-means-of-grace}{%
\section{The Means of Grace}\label{the-means-of-grace}}

\hypertarget{after-death}{%
\section{After Death}\label{after-death}}

\end{document}
