% Options for packages loaded elsewhere
\PassOptionsToPackage{unicode}{hyperref}
\PassOptionsToPackage{hyphens}{url}
%
\documentclass[
]{book}
\usepackage{amsmath,amssymb}
\usepackage{iftex}
\ifPDFTeX
  \usepackage[T1]{fontenc}
  \usepackage[utf8]{inputenc}
  \usepackage{textcomp} % provide euro and other symbols
\else % if luatex or xetex
  \usepackage{unicode-math} % this also loads fontspec
  \defaultfontfeatures{Scale=MatchLowercase}
  \defaultfontfeatures[\rmfamily]{Ligatures=TeX,Scale=1}
\fi
\usepackage{lmodern}
\ifPDFTeX\else
  % xetex/luatex font selection
\fi
% Use upquote if available, for straight quotes in verbatim environments
\IfFileExists{upquote.sty}{\usepackage{upquote}}{}
\IfFileExists{microtype.sty}{% use microtype if available
  \usepackage[]{microtype}
  \UseMicrotypeSet[protrusion]{basicmath} % disable protrusion for tt fonts
}{}
\makeatletter
\@ifundefined{KOMAClassName}{% if non-KOMA class
  \IfFileExists{parskip.sty}{%
    \usepackage{parskip}
  }{% else
    \setlength{\parindent}{0pt}
    \setlength{\parskip}{6pt plus 2pt minus 1pt}}
}{% if KOMA class
  \KOMAoptions{parskip=half}}
\makeatother
\usepackage{xcolor}
\usepackage{longtable,booktabs,array}
\usepackage{calc} % for calculating minipage widths
% Correct order of tables after \paragraph or \subparagraph
\usepackage{etoolbox}
\makeatletter
\patchcmd\longtable{\par}{\if@noskipsec\mbox{}\fi\par}{}{}
\makeatother
% Allow footnotes in longtable head/foot
\IfFileExists{footnotehyper.sty}{\usepackage{footnotehyper}}{\usepackage{footnote}}
\makesavenoteenv{longtable}
\usepackage{graphicx}
\makeatletter
\def\maxwidth{\ifdim\Gin@nat@width>\linewidth\linewidth\else\Gin@nat@width\fi}
\def\maxheight{\ifdim\Gin@nat@height>\textheight\textheight\else\Gin@nat@height\fi}
\makeatother
% Scale images if necessary, so that they will not overflow the page
% margins by default, and it is still possible to overwrite the defaults
% using explicit options in \includegraphics[width, height, ...]{}
\setkeys{Gin}{width=\maxwidth,height=\maxheight,keepaspectratio}
% Set default figure placement to htbp
\makeatletter
\def\fps@figure{htbp}
\makeatother
\setlength{\emergencystretch}{3em} % prevent overfull lines
\providecommand{\tightlist}{%
  \setlength{\itemsep}{0pt}\setlength{\parskip}{0pt}}
\setcounter{secnumdepth}{5}
\usepackage{gentium}
\ifLuaTeX
  \usepackage{selnolig}  % disable illegal ligatures
\fi
\IfFileExists{bookmark.sty}{\usepackage{bookmark}}{\usepackage{hyperref}}
\IfFileExists{xurl.sty}{\usepackage{xurl}}{} % add URL line breaks if available
\urlstyle{same}
\hypersetup{
  pdftitle={The Difference},
  pdfauthor={I.G. Monson},
  hidelinks,
  pdfcreator={LaTeX via pandoc}}

\title{The Difference}
\usepackage{etoolbox}
\makeatletter
\providecommand{\subtitle}[1]{% add subtitle to \maketitle
  \apptocmd{\@title}{\par {\large #1 \par}}{}{}
}
\makeatother
\subtitle{A Popular Guide to Denominational History and Doctrine}
\author{I.G. Monson}
\date{1915}

\begin{document}
\maketitle

{
\setcounter{tocdepth}{1}
\tableofcontents
}
\chapter*{Preface}\label{preface}

What is the difference between the various Christian sects? The common answer is that there is no difference, that they all have the same Bible, that all are trying to reach the same place, etc. For the ignorant this is the easiest way to get away from this question. The lazy, indifferent, and the unbeliever will find the same answer the most convenient. But any one interested in truth and in Christianity will gladly accept information concerning the most important questions in life.

Of the many good works on the subject none are suited for general distribution among the several Lutheran church-bodies in this country. In the theological seminaries the works of Winer, Guericke, Guenther, and Gumlich are, of course, indispensible; but in the preparatory schools, parochial and Sunday-schools, as well as in the average Lutheran home, such learned works are impractical, if not wholly useless.

The aim has, therefore, been in the present volume to be more suggestive than exhaustive, to avoid all technicalities, and to treat the different dogmas in such a manner that the common, every-day Christian could understand them.

In order to attin this end, the dogmatic, positive, Biblical, Lutheran method was found most serviceable. And for this we make no apology, the cry of ``represtination of Reformation theology'' by rationalists and unionists to the contrary notwithstanding. As God at the last day will judge man according to His Word (John 12, 48), the only right way is to ask for the Old Paths, where is the good way, and walk therein (Jer. 6, 16).

Now and then names and phrases will be found inclosed in brackets. This has been done in order to stimulate further research by both teacher and student. Books of reference of some kind are now in nearly every home.

To write on Symbolism in our unionistic and irreligious age is, indeed, a thankless task. And yet \emph{the full truth was never popular}. Now more than in the last century the words of Stahl have full force: ``The powers that be are against us. the masses are against us. The tendency of the times is against us. The powerful errorists within the Church are against us.'' To this may here be added: The Reformed civilization under which we live is against us.

But God has a remnant yet to save. \emph{And it will be saved, but only through the Word of Truth}.

\textsc{The Author.}

\chapter*{PART I.}\label{part-i.}
\addcontentsline{toc}{chapter}{PART I.}

\section{Origin of Divided Protestantism}\label{origin-of-divided-protestantism}

When Luther in Germany opposed Roman Catholicism and popery, it was because the Roman Church had perverted the Gospel of Christ, taught righteousness by works, and prohibited the circulation of the Bible among the people. Strive as he would, his troubled soul could find no consolation in following the precepts of the Church. It was only after years of fruitless toil, in and out of the cloister, that he had at last found peace and consolation in the precious promise of the Lord: ``The just shall live by his faith'' (Hab. 2, 4)

The Swiss Reformers, Calvin and Zwingli, on the other hand, do not seem to have been thus troubled. Though earnest in their reform work, they worked more in the direction of emancipating their Churches from the yoke of the Roman organization, in eradicating superstition, correcting outward, flagrant abuses, and instilling into the minds of the people a spirit of independence, political as well as religious.

Luther contended that an effectual reformation must come from within; for if the tree is made good, the fruit will also be good. The Swiss made the branches, instead of the root, their point of attack.

Luther's weapon in the battle of the Reformation was the Word of God only. With this means, he said, the Savior repelled even Satan himself. With this means the apostles conquered the heathen world and laid it at the feet of Christ. To the Swiss this method was too slow; they were not content with letting the leaven of the Gospel work in its peculiar way; they must needs also use physical force and political power, a trait to this day characteristic of the Reformed Church.

Closely allied to this was the different views taken by the Reformers as to the value of the Scriptures. Luther contended that the Word of God was not merely a message to mankind, but the vehicle of the Holy Spirit through which He gives the proffered grace, a truth attested by the Apostle Paul, when he says that the Holy Spirit was given through the preaching of the Gospel (Gal. 3, 2), and faith through the Word of God (Rom. 10, 17).

Zwingli, on the other hand, believed that the Spirit needed no means, that He worked independently of the Scriptures. The preaching of the Word was to him only ``an outward sound,'' outside of which the Holy Spirit must work regeneration and salvation. Hence, he hoped ``that good men in all ages, who had walked uprightly according to the light they had, were also saved.'' (\emph{Expos. Fidei}, 2, 559.)

But the real difference between the Reformers came to light in their discussion concerning the authority of the Scriptures. Luther contended with the Apostle Paul that the natural man receiveth not the things of the Spirit of God, for they are foolishness unto him, neither can he know them, because they are spiritually discerned. (1 Cor. 2, 14) Hence, if he would not be adjudged wise in his own conceit (Rom. 12, 16), he must not think above that which is written (1 Cor. 4, 6), but bring into captivity every thought to the obedience of Christ (2 Cor. 10, 5). History as well as his own experience had taught him that it was futile to worship God, teaching for doctrines the commandments of men (Matt. 15, 9), and therefore he set up this formula in the Smalcald Articles: ``It is certain that the Word of God alone shall set up articles of faith, and none else, not even an angel.'' (II, 15.)

No doubt, the Swiss would have said Yes and Amen to this, adding the rationalistic phrase: ``rightly understood.'' (John Calvin declared that he had signed the Augsburg Confession in this manner.) They did not deny the authenticity of any part of the canonical Scriptures, but ever and anon they would follow human reason, and assert that the language of the Holy Spirit must not be taken in a literal, but in a figurative sense, as, for instance, in the doctrine of the Lord's Supper. Any other interpretation, they said, would be ``absurd'' and ``unreasonable.'' What becomes of the Scriptures under such treatment we shall observe as we enter more fully into our subject.

That such opposite views concerning the Scriptures should be followed by Different methods of church-work is not surprising.

To Luther only such changes seemed necessary as would promote true piety. Any ceremony, therefore, which had come down from antiquity, and was not connected with any idolatrous practice, or could be cleansed from such, was retained. Church interiors, with altars, pulpits, sacristies, organs, paintings, and statuary, were retained, also clerical vestments and the observance of many holidays; in short, anything and everything which could be used properly, and was not prohibited by the Word of God.

To the Swiss everything Catholic was ``a heathenish abomination.'' What the Scriptures did not directly command was to them unscriptural, and hence they not only countenanced the ruthless destruction of statuary and paintings in churches, but even counseled to do so. Church interiors were entirely remodeled, and ceremonies of the most simple kind instituted. This difference between the Reformed and the Lutheran position may best be illustrated by an anecdote. In discussing the retention or rejection of the old liturgies, Zwingli said, ``Show me a passage in the Scriptures wherein they are enjoined.'' To which Luther replied, ``Show me a passage wherein they are condemned.''

The different stand taken by the Reformers concerning Church and State must not be overlooked. The Lutheran Church soon became a State Church, it is true, but in spite of the teachings of Luther. ``It is with the Word we must contend,'' he observed, ``and by the Word we must refute and expel what has gained a footing by violence. I would not resort to force against such as are superstitious, nor even against unbelievers! Whosoever believeth let him draw nigh, and whoso believeth not, stand afar off. Let there be no compulsion. Liberty is of the very essence of faith. God does more by the simple power of His Word than you and I and the whole world could effect by all our efforts put together! God arrests the heart, and that once taken, all is won.''

To the Swiss Reformers the Jewish Church of the Old Testament was the only correct model. Zwingli looked upon the Council as representative both of Church and State, and the theocratic State Church organized by Calvin at Geneva is a fact well known to all students of history. Even in America the Reformed Church commenced as a State Church, and the roots of it are still sprouting such shoots as Sunday-laws, chaplains for conventions, Congress, legislatures, and for the army and navy, and the fanatical activities of the National Reform Association.

The Reformed Churchis split up into a multitude of sects. They have no confession to which they all subscribe; many of them have no confession at all, and others modify their beliefs as occasion demands. This is because a belief founded partly on ``reason,'' ``views,'' ``science,'' ``feeling,'' and the like, can have no stability. Views change; the rational way of one is irrational to another; the science of to-day is a fable to-morrow. In order, therefore, to gain recruits for this or that sect, innumerable schemes, methods, and measures are resorted to, -- camp-meetings, revivals, the Y. M. C. A., and of late the Laymen's Movements, Sociali Service, etc.

We shall now pass in review the principal denomination which have a following in the United States. Our next chapter briefly summarizes the history of these denominations, and succeeding chapters will take up, one at a time, the chief doctrines of Scripture, stating the agreement of our Lutheran Confessions with the same, and the differences which exist between us and the various denominations.

\section{Historical Summaries}\label{historical-summaries}

\subsection{THE ANGLICAN CHURCH}\label{the-anglican-church}

Many dignitaries of the Anglican Church do not, at the present time, like to date the beginning of their denomination from the time of the Reformation. And some would like to ignore the Reformation altogether. Says one: ``England has never had but one religion; it has never changed that religion, and it has that religion still.''

From the sixth century until the year 1534, when Henry VIII was refused the sanction of the pope to a divorce from his first wife (he had six in all), England was as much a Catholic country as any other of the European states. In that year, to avenge himself on the pope, Henry ``denied the papal supremacy, and declared himself to be the one protector of the English Church, its only and supreme lord, and, so far as might be by the laws of Christ, its supreme head.'' And this Parliament later confirmed. This was the beginning of the revolt which carried England outside the Catholic fold.

Writings of Luther and, later, of the Swiss Reformers were early and eagerly read in England. Henry, who had written against Luther before the breach with the pope, continued in the Church, and, as far as Roman doctrine is concerned, died a Catholid; but when his son, Edward VI, ascended the throne, in 1547, reform ideas were so well rooted that Archbishop Thomas Cranmer, counseled mainly by Reformed theologians (Martin Bucer, Peter Martyr), in 1552 set up a confession of faith containing forty-two articles, later cut down to thirty-nine at the Synod of London, in 1562. This confession was ratified by Parliament in 1571, and thereby became a part of the law of the land. ``The Book of Common Prayer,'' which is still in use, contains not only this document, but also a full ritual for all occasions during the church-year.

The Thirty-nine Articles, the confession of the Anglican Church, are decidedly Reformed in tone and tendency, though in many respects taking a middle course between the Lutheran and the Reformed. But when Episcopal missionaries say to Lutheran immigrants that their Church is the Lutheran, excepting the language, they ought to know that they are not telling the truth. The Episcopal Church in the United States is only the Anglican Church without state-connections. It is officially known as The Protestant Episcopal Church. As such it dates from 1787, when it received its first bishop, Dr.~Seabury, and constituted itself independent of England.

In 1873, it suffered a small loss through a few priests, who left it on account of a ritualistic (Romish) tendency, which was rapidly gaining ground in the Church. They reduced the confession to thirty-five articles, and changed a few expressions here and there. They are known as The Reformed Episcopal Church.

Though composed of three more or less antagonistic elements, the ``high,'' the ``low,'' and the ``broad,'' the Episcopal Church has escaped any serious division, and has remained virtually one united Church up to this time. It is governed by bishops, -- hence its name, -- and believes in the so-called Apostolic Succession, an invention inherited from popery.

In late years the Episcopal Church has become very lax in many ways. Higher criticism and rationalism dominate its theologians, and ritualism opens the way back to Rome for many every year.

In many Episcopal churches, both in our country and in England, prayers are said for the dead, the saints are worshipped, mass is read, and auricular confession to the priest is practised. Indeed, nothing but the claims of the supremacy and infallibility of the pope seems to stand in the way of a large section of the Episcopal Church going bodily into the Roman Catholic communion.

\subsection{PRESBYTERIANS}\label{presbyterians}

The attitude of Henry VIII to the REformation, the instability of the government under his first successors, and the double influence of Lutheran and Reformed theologians during the stirring times of the sixteenth century did not allow the English people to study the problems of the Reformation properly. They were not only continually crossing the English Channel to and fro on account of persecutions of one kind or another, but they were also ``tossed hither and thither by all sorts of winds of learning,'' and as ``it is ever to the injury of essentials that the midn of man is preoccupied with secondary matters,'' the English church people suffered on this account more than any other nation. ``Dissenters,'' ``Independents,'' and ``Non-conformists,'' and ``Puritans'' are names which remind the serious-minded Evangelical Christian of to-day of the futile and unnecessary strifes of centuries ago in England, religious, of course, in a way, but mostly about ceremonies, vestments, church furnishings, and church government.

The principles underlying Presbyterianism came originally from Geneva, where John Calvin held sway in Church and State. This cold, unbending, and rigoristic reformer had patterned his church government mostly on Old Testament lines, and, through strict discipline, sought to attain, if possible, an immaculate Church on earth. Through John Knox, the Scotch reformer, these and other Calvinistic peculiarities were transplanted on English soil, where they not only became the source of endless troubles, but also the cause of an un-evangelical rigorism and persecutions.

Presbyterians are called so on account of their mode of church government. When a church is organized, a board of elders is chosen. The elders elect the minister, and these, called a ``session,'' rule in all temporal as well as spiritual affairs. Over the session, as a court of appeals, is the presbytery, composed of churches within a given area; a number of presbyteries make up a Synod, a still higher court, and the General Assembley, which meets every three years, constitutes the court of last resort.

But the Presbyterian Church has also been noted for its Calvinistic doctrine concerning predestination, the belief according to which God has loved, not the whole world, as Christ explicitly declared (John 3, 16), but only a few, namely the elect. To their Westminster Confession, adopted in 1647, a ``declarative statement'' was added a few years ago, which, to some degree, softens the language employed in the confession, but as the old confession still stands, the situation is practically unchanged.

The Presbyterians are divided into numerous sects. A few may be enumerated: ``The Associated Presbyterian Church''; ``The Reformed Presbyterian Church'' (Covenanters, Cameronians); ``The Associate Reformed Presbyterian Church''; ``The United Presbyterian Church''' ``The Cumberland Presbyterian Church,'' etc.

Some of the questions which divide these Presbyterian church-bodies are altogether trivial. The Southern Presbyterians and the United Presbyterian Church, for instance, remain separate mainly because they cannot agree on the question whether any hymns outside of the metrical versions of the Psalms should be sung in public worship. While the Southern Presbyterians use hymns and ``Gospel songs,'' The United Presbyterians restrict themselves to the use of the Psalms, in agreement with the strict Zwinglian idea that ``nothing is permitted except what the Bible commands.'' The correct, Lutheran, doctrine on this point is that only that is prohibited which the Bible forbids, and that all else is a matter of Christian liberty, and must not keep the Churches separate.

\subsection{CONGREGATIONALISTS}\label{congregationalists}

It is a question whether Congregationalists may properly be called a denomination, as the different churches of this body have nothing in common but the name.

When Congregationalists maintain, ``That every Christian church is entitled to elect its own officers, to manage all its own affairs, and to stand independent of, and irresponsible to, all authority, saving that only of the Supreme and Divine Head of the CHurch, the Lord Jesus Christ,'' they so far utter a precious Scriptural truth. But when they condemn the formation of even \emph{advisory} church-bodie (synods), the acceptance of a common confession or creed by such bodies, and when in the confessions of their several local churches they sometimes use terms so vague that even a Brahmin could subscribe to them, then it is no wonder that the Universlaist and the Unitarian churches, \emph{which are not Christian}, originated in the Congregational Church in this country, and that even within the Congregational Church there is a large group of congregations in which even the elements of Christianity are no longer taught nor confessed.

Originally the Congregationalists were one with the Presbyterians in fighting Episcopalism in England, and with them accepted the Westminster COnfession, in 1643. Later they broke with their allies on account of the Presbyterian church government, and, chiefly through the fanatical Robert Brown (1586), constituted themselves as Independent churches.

In 1658, the Independents of London changed the Westminster Confession in a few minor points. (The Savoy Declaration.) THis altered confession was accepted by the Independents of this country at a synod in Boston, in 1680. But this acceptance they do not consider binding on anybody, not even on those who subscribe to it.

Both before and after Robert Brown took a hand in shaping the destiny of the Independents, they suffered severe persecutions in England. But, sad to relate, when, in 1620, they landed at Plymouth Rock, Mass., none became more intolerate than they (Pilgrim Fathers). Only their own church-members could become citizens. Church and State were one. Churchgoing was compulsory. Those who disagreed with them were sorely persecuted, unmercifully banished, or condemned to death.

The Congregationalists make special effort to draw Lutherans into their fold, and are not very particular about the means which they employ toward this end. In strong Lutheran communities they will, for instance, introduce a ceremony which closely resembles the Lutheran rite of confirmation, and some of their congregations have even adopted the Lutheran name (``Congregationalist Lutheran Church,'' etc.), with the sam epurpose in view.

\subsection{THE BAPTISTS}\label{the-baptists}

Baptists are generally not very anxious to discuss the beginning of their denomination. They try to show at all times that Baptists, as a Church, never have persecuted other believers. Lately, however, teachers in their higher schools have been making researches into historical archives of the Old World, and one after the other concedes that ``early Baptists were called by their enemies Anabaptists.'' (\emph{Bapt. Congr.}, 1911, 106.)

During great political and religious upheavals there are always some connected with the movement who commit excesses or are dissatisfied with the final settlement. Of such there were not a few during the beginning of the Reformation. Some, as Muenzer, Storch, and others, called ``the heavenly prophets of Zwickau,'' found the Scriptures inadequate to their needs, and maintained that GOd had given them additional revelations. Among the first things which they discarded was Infant Baptism. After the death of Muenzer on the battlefield in the Peasants' War, 1524, John of Leyden gathered the prophet's followers, took possession of the city of Muenster, in Westphalia, and inaugurated a government along socialistic lines. Not only community of goods, but also polygamy was practiced. After the storming of the city by the neighboring princes, and execution of the chief leaders, their adherents were scattered throughout Switzerland, Germany, the Netherlands, and England, where they henceforth became known by the name Anabaptists, because, not acknowledging Infant Baptism, they rebaptized those who had been baptized as children. Anabaptists means ``those who baptize again.''

The man who succeeded in creating order out of this mess is the Frieslander Menno Simmons. Zealous, yet prudent, he succeeded in collecting the scattered members into congregations both in the Netherlands and Germany, and gave them instructions in church government and doctrine. From that time on they became more tractable, repudiated the excesses of their fanatical ancestors, and, after the name of their leader, called themselves Mennonites. Later they divided into numerous groups.

That the Baptists of our day do not look with pride on the historical beginning of their denomination is only human. It is nothing to be proud of. And yet it is a historical fact that the persecuted Anabaptists from Holland came to England, where they first made common cause with the dissenters (Independents, etc.), and only since the middle of the seventeenth centiry they have called themselves Baptists. By this name they wish to emphasize their belief in immersion as ``the only proper mode of administering Baptism,'' and that only to adults. Until recently it has been generally believed that immersion was practiced among them from the very beginning. Such is not the case however. Up to 1641 Baptism was administered in England by sprinkling or pouring, and no other method was set forth ``as the only correct way.''

In church organization the Baptists agree with the Congregationalists. There are Baptist churches, but there is no Baptist Church, except in anme; hence, no common confession. In 1677, a ``Confession of Faith'' was set up in London, and again in 1688, but this is not binding on all who call themselves Baptists. Their old belief concerning immersion as a necessary mode of Baptism is also undergoing a marked change. Many churches hold ``that the character and confession of the candidate gives sufficient validity to the ordinance, and receive baptized believers from whatever alliance they may come.'' (\emph{Bapt. Layman's Book}, p.~130.) Hence, Baptists ``are coming to have almost nothing to stand out for in separation from other denominations, except some matters of external form.'' (\emph{Bapt. Congr.}, 1911, p.~91.)

Like all Reformed, the Baptists are divided into numerous sects. ``Old-School Baptists,'' also called ``Primitive Baptists,'' are Calvinistic in doctrine. Some of the others tell their own story with their name: ``Free-Will Baptists,'' ``Anti-Mission Baptists,'' ``Six-Principle Baptists'' (Heb. 6, 1.2), ``Seventh-Day Baptists,'' ``Campbellites'' (or ``Disciples of Christ''), ``Christians,'' ``Weinbrennerians'' (``Church of God''), ``Dunkers,'' ``River Brethren,'' etc., etc.

Their home missionaries are as fanatical as Muenzer, and respect no denominational lines. The Lutherans especially are bothered by these zealots. We can truly say of them as the Baptist professor, McGlothlin, says of the Campbellites: ``They have continued their predatory habits of proselyting down to the present time.'' (\emph{Bapt Congr.}, 1911, p.~104.) Their errors are treated in the second part of this treatise.

\subsection{THE METHODISTS}\label{the-methodists}

The Methodist Church is the youngest church-body among the larger Reformed denominations, dating from about 1739. In 1729, two brothers, John and Charles Wesley, were studying at Oxford University, and, becoming troubled about the welfare of their souls, commenced to study the Bible more earnestly, and to practise the Christian virtues.

On account of their strict methodical ways of performing these things a student, in derision, called them Methodists, a name which the Wesleys themselves shortly afterwards applied to their different ``societies,'' since organized into independent churches.

The Wesleys did not at first intend to found a new denomination. They only wished to inject more real Christian life and energy into the Anglican Churhc, of which they were ordained presbyters. But very soon the ``societies'' which they had organized for seeking and practising holiness thought that they could no longer worship together with the rest of the church-members, and therefore organized themselves into separate congregations. Though both John and Charles Wesley had a like share in starting the pietistic movement, yet John soon took the leading part, and is considered the real father of Methodism.

The Wesley brothers were zealous workers, it is true, but when they returned from America, in 1737, they confessed that they had labored to save others, but were not converted themselves. This they first became, so they confessed, when John had absorbed the truths proclaimed by Luther in his introduction to the Epistle to the Romans, and Charles, through reading Luther's exposition of the Epistle to the Galatians.

For a time the Wesleys were assisted by George Whitefield in their evangelistic efforts. But they soon parted company, as Whitefield was a strict Calvinist in the doctrine of predestination, and became the founder of the Calvinistic Methodists, or ``Huntingdon Connection.''

The confession of the Methodist Church, edited by John Wesley, is only the Thirty-nine Articles of the Anglican Church slightly modified and cut down to twenty-five. IN these we look in vain for specific Methodist doctrine and practises. These are inferred from the writings of John Wesley mainly. Characteristic practises and belifs of the various Methodist bodies are: revivals (camp-meetings, protracted meetings, etc.); condemnation of every use of alcohol and tobacco; a belief that conversion, to be genuine, must be an experience of which the believer knows the exact time, and experience brought about by an immediate action of the Holy Spirit, and which is not complete unless it has been \emph{felt} by the Christian; a belief that man may attain perfection (perfect sanctification) in this present life; the doctrine, held in common with all Reformed sects, that Baptism and the Lord's Supper are not means of grace, but only signs of God's grace. The Methodists wish to be styled evangelical, \emph{par excellence}, among the Reformed Churches, but there are certainly none more legalistic than they. The doctrine of Christian liberty is a sealed book to them.

In England the Methodist Church calls itself the Wesleyan Methodist Church, and dates its beginning from 1739, in which year John Wesley started the first congregation in Bristol, and dedicated ``The Old Foundry'' in London as a house of worship. In this country the first Methodist church was started in New York, 1766, by a Wesleyan local preacher from Ireland. Others followed and started ``societies'' in many places, so that in 1784 Dr.~Coke, whom Wesley consecrated bishop, met sixty preachers in Baltimore for the first American conference. This conference organized the Methodist Episcopal Church, and adopted the articles of faith edited by John Wesley.

The church government of the Methodist Episcopal Church is hierarchical to a high degree. The congregations have very little to say. Their ministers are appointed by the bishop, and after from two to five years are transferred to another charge by the same authority.

When the name Methodist Church is used, the official name Methodist Episcopal, is usually understood. But the number of Methodist connections is very great. Some insert ``Methodist'' in their corporate name, others omit this term. To enumerate a few: ``The Wesleyan Methodist New Connection''; ``The Primitive Methodist Connection''; ``Bible Christians''; ``Methodist Episcopal Church South''; ``The Protestant Methodist Church''; ``The Canada Wesleyan Methodist''; ``African Methodist Episcopal Church''; ``United Brethren''; ``The Albright Methodists'' (The Evangelical Association); ``Jumpers''; ``Howling Methodists''; and ``The Salvation Army.''

\section{Unchristian Cults}\label{unchristian-cults}

How many sects there are in this country alone no one knows, not even the Director of the Census. Catholics would fain tell us that this state of affairs dates back only to the time of the Reformation. There may, at the present time, be more sects than formerly, but divisions in the Church have always existed, and the Pope of Rome has not only had his hands full in pacifying and judging between warring factions of monks and their so-called ``eminent teachers,'' but he has constantly had on hand a war of extermination against dissenters, who would not acquiesce in his teachings. The attitude of the pope towards the Modernists of our own day is only the old story over again. But their silence after his condemnation is the slence of the dead, -- and of the same value.

But there are sects and sects. Some who retain the fundamental thuths of Christianity are called Christian sects, for in them there is a possibility of acquiring saving truths, -- though the danger of not finding them is also very great. The most important of these we have discussed in the preceeding chapters. But where even fundamentals are denied, there no salvation is possible, and such sects are at best only clubs for entertainment, if not, indeed, wholly anti-Christian and gates of hell. Among the chief of these the following may be mentioned: --

\subsection{UNITARIANS AND UNIVERSALISTS}\label{unitarians-and-universalists}

The late Senator Hoar of Massachusetts used to ask this conundrum, ``What is the difference between Universalists and Unitarians?'' THe answer was, ``The Universalists believe that God is too good to damn them, and the Unitarians believe that they are too good to be damned.'' The characterization is so striking and true that little more need be said.

These churches accept the Bible as their guide in spiritual matters, but only so far as it accords with human reason. They do not believe in the Trinity, original sin, redemption, atonement, endless punishment, etc., but believe that all will be saved, or -- some of them -- that the wicked will be annihilated.

\subsection{THE ADVENTISTS}\label{the-adventists}

It would not be proper to lay the heresy of the Advent Church at the door of the Baptists, and yet a Baptist clergyman is responsible for its existence.

William Miller, a licensed Baptist preacher, was neither an astronomer nor a scholarly theologian, but believed that he could foretell the day and year of the second coming of Christ. This was in the year 1833. People became hysterical, sold their belongings, and in ``ascension robes'' prepared for the event. The Day of Judgment was set for April 14, 1844. That day is long past, and yet the Adventists revere Miller as a great prophet. The day has since been set time and again by other would-be prophets, but, of course, with the same results. And yet the Adventists work with might and main to secure converts for their faith. THe Baptists have furnished them the mode of baptizing, and the Methodists camp-meeting ideas.

The larger body among them, the Seventh-Day Adventists, denies the Holy Trinity, teaches the mortality of the soul, the final destruction of the wicked, and, especially, that Saturday must be observed as the weekly day of rest.

On account of their persistent missionary work, in which they respect no denominational lines, and their juggling with the words ``Christ,'' ``faith,'' ``sacrifice,'' etc., a goodly number of simple-minded Christians are ensnared by them every year.

Adventism is represented by a great number of sects. Among the most active are, at present, the Russellites. The leader of the Russellites is ``Pastor'' (he has never been ordained) Charles T. Russell. The work of spreading the erroneous teaching of the ``Pastor'' is carried on under a confusing and misleading variety of titles, the best known of which is ``The International Bible Students' Association'' of Brooklyn. Other names are: ``Watch Tower Bible and Tracts Society,'' formerly ``Zion Watch Tower and Tract Society''; also ``The Laymen's Home Missionary Movement,'' in the interests of which are published \emph{Everybody's Paper} and \emph{the People's Pulpit}. Most important of all the publications of ``Pastor'' Russell are six large volumes, entitled \emph{Studies in Scripture}. IN order to even mislead the blind, books and articles are prepared in raised letters, under the title of ``The Gould Free Library for the Blind,'' South Boston, Mass. There are also issued what are called \emph{International Sunday-school Lessons}. It has been proved in court that ``Pastor'' Russell is totally ignorant either of Greek or Hebrew, while claiming to be a great Bible interpreter; that he has been ordained by no Church or society; that he has been divorced by his wife; that he is connected with various business corporations, all of which are under his absolute control, and which hold stock valued at millions of dollars. We append a brief summary of the false teachings assiduously being spread by Russell and his followers, the Millennial Dawn religion.

\begin{enumerate}
\def\labelenumi{\arabic{enumi}.}
\item
  Russellism denies the doctrine of the Trinity.
\item
  It denies that Jesus Christ was God before His incarnation.
\item
  It teaches that Christ was only a created spirit.
\item
  In incarnation He ceased to be a spirit, and became the second Adam.
\item
  As the second Adam He had only one nature.
\item
  His nature of humanity was annihilated on the cross.
\item
  He did not rise in the body in which he died.
\item
  The body in which He died may have been dissolved into gas.
\item
  The body in which He appeared after death was nothing more than a momentarily materialized appearance, which was finally dissolved
\item
  Jesus Christ is not now a man.
\item
  The ``Man Christ Jesus'' no longer exists.
\item
  Jesus Christ is now an invisible spirit-being.
\item
  He came to the world in 1874 as an invisible spirit-being.
\item
  The millennium will begin in 1914. (!)
\item
  All the dead out of Christ will be raised at that time.
\item
  All the unrighteous and wicked dead will be raised and made perfect and innocent like Adam before the Fall.
\item
  All the unrighteous and wicked dead will be given a second chance.
\item
  The more wicked they have been in this life, the more likely they will be, through the ``experience'' of sin, to accept the Gospel of the second chance.
\item
  Those who accept the second chance will have everlasting life.
\item
  Those who get everlasting life will sustain it by eating food.
\item
  Those who do not want to live forever will have the privilege of being asphyxiated in the lake of fire.
\item
  The assurance given to the wicked and sinful is that there is no suffering for sin.
\item
  Those who do not care for heaven need not be afraid of hell.
\item
  The finally impenitent are extinguished here, and annihilated.
\end{enumerate}

\subsection{QUAKERS}\label{quakers}

``Quiet as a Quaker meeting'' is a byword. The reason for this is that George Fox, the founder of the sect, discarded the Bible as an insufficient message from God, and sought consolation in ``inner promptings, whisperings, and moving of the spirits.'' In their so-called services no Bible-text is expounded, but ``when the Spirit moves'' a man or woman, that person speaks; but until then there is silence. And when, after waiting a reasonable time, no one is ``moved,'' the congregation goes home.

Quakers also wish to be known as the ``Society of Friends.'' They hate ostentation, never remove the hat in saluting any one, and call everybody ``thou.'' They will neither take an oath nor serve in the army.

The theology of the Quakers may be somewhat more definite now than at the time of their founder, but not sufficiently so as to characterize them as Christians. Their founder's everlasting refrain ran like this: ``Not Scripture, but Spirit! Not Christ for us, but in us! Not churches (`tower buildings') and bells, not Sacraments and dogmas, but only the inner light which GOd ignites in the conscience of every one, and through which the SPirit of Christ regenerates and comforts man.'' This sounds plausible, but as long as they deny the atonement, Christ will not send the regerating Spirit. But without regeneration no one shall see God. (John 3, 3.) The Quakers are not the only ones relying on the ``Spirit'' and the ``inner light.'' But they seem to rely on a ``Spirit'' and ``Light'' which is not always to be relied on. Thus we find them divided into ``orthodox,'' ``dry,'' ``wet,'' ``Hicksite,'' ``fighting,'' ``progressive,'' ``primitive,'' ``Baptist,'' ``Christian,'' and ``Keithian'' Quakers.

Quakerism is only a type of a large class of cults, both ancient and modern, the adherents of which, like the rich man (Luke 16, 27), find, as they think, the Bible insufficient to the needs of mankind. But whatever their achievements may be in our eyes, the Holy Spirit says of them that ``other foundation can no man lay than that is laid, which is Jesus Christ.'' (1 Cor. 3, 11.)

\subsection{THE MORMONS}\label{the-mormons}

This old world of ours has seen many queer and brazen humbugs, but not till Mary Baker-Eddy proclaimed Christian Science has anything appeared quite equal to Mormonism, or, as its adherents wish to be known, The Church of Jesus Christ of Latter-Day Saints, founded by Joseph Smith.

ne Solomon Spaulding (died 1818) amused himself, after retiring from the ministry, by writing a book, in Biblical style, purporting to be the history of the peopling of of America by the ten lost tribes of Israel.

This manuscript Joseph Smith secured, and, after altering a little here and there (without, however, improving its style, for he was very poorly educated) he published it in 1830 under the name of \emph{The Book of Mormon}, and proclaimed it to be of equal authority with the Bible.

But the real author's name does not appear on its pages. Instead, it purports to be a revelation. An angel, so he said, pointed out to Joseph Smith that ``on the west side of a hill, not far from the top, about four miles from Palmyra, N.Y., near the road to Manchester,'' he would find plates of gold inscribed with hieroglyphs in ``reformed Egyptian,'' and a pair of eye-glasses called ``Urim and Thummim.'' And Joseph found them, so he says, and by sitting behind a curtain, dictating to a confederate on the other side, he gave to a benighted world the greatest light in its history: \emph{The Book of Mormon}. What became of the plates is still a mystey. Forsooth, certain persons bear witness, on a front page in the book, that they have seen them, but it is only the testimony of ill-reputed people in favor of one of the same kind.

The plates are said to have been hidden in the hill about A. D. 420. Yet the inscriptions mention Calvinism, Universalism, Methodism, Millenarianism, and Roman Catholicism!

Though polygamy is one of the main tenets of Mormonism, still it is condemned in \emph{The Book of Mormon}. It was an after-thought, and was revealed to the churhc later!

To those who calmly compare Mormonism with Christianity there is not much danger. But those who listen only to the glib-tongued ``missionaries,'' who conceal more than they divulge, Mormonism is as alluring a scheme as the devil has yet concocted. Mormonism is by its missionaries pictured as a paradise on earth; hence the rule of the ignorant and the sensual to Salt Lake City and other Mormon centers.

\subsection{CHRISTIAN SCIENCE}\label{christian-science}

Christian Science is one of the latest cults to bid for the favor of those whom Christ characterizes as ``an evil and adulterous generation seeking after a sign.'' (Matt. 12, 39.)

Mary Baker-Glover-Eddy, divorcee and widow, a small, frail, hysterical, and nervous woman, consulted during an illness, in 1862, a noted mesmerist by the name of P. P. Quimby. From that time on she commenced to teach, and write on, mental healing. People who knew Quimby have testified that she, at least to begin with, copied his teachings verbatim, but without giving him credit.

However, Quimby would probably not make the charge of plagiarism against Mrs.~Eddy if he were to read her \emph{Science and Health, with Key to the Scriptures}, the Scientists' Bible, as it reads to-day. No man in his right mind would wish to be accredited with even having started the old lady on her Bible-producing career, as the twaddle she wrote can only emanate from a brain disordered.

That seemingly sane people can revere her as a god, and freely spend millions in furthering the cause of the cult, finds its explanation only in one cause, the one which St.~Paul mentions when he says: ``Because they receive not the love of truth, that they might be saved. And for this cause God shall send them strong delusion, that they should believe a liel; that they all might be damned who believed not the truth, but had pleasure in unrighteousness.'' (2 Thess. 2, 10-12).

Christian Science -- officially, ``The Church of Christ, Scientist'' -- has rightly been characterized as ``neither Christian nor scientific.'' In corroboration hereof the following from \emph{Science and Health} will suffice: --

~~``Got is all, and therefore matter, sickness, sin, and death have no reality; they are illusions of mortal mind.''

~~``All disease is belief.''

~~``All is mind; matter is but a belief and error.''

~~``Matter and mortal body are illusions of human belief.''

~~``Man is coexistent with God.''

~~``Man never fell into sin or death; he is forever happy, harmonious, and immortal.''

~~``God never forgives sin.''

~~``Jesus never ransomed man by paying the debt that sin incurs.''

~~``Man has neither birth nor death.''

~~``Destroy sin, sickness, and death in the mind, and they are gone forever.''

The fundamental propositions of Christian Science are stated thus: --

\begin{enumerate}
\def\labelenumi{\arabic{enumi})}
\tightlist
\item
  God is all in all; 2) God is good, Good is mind; 3) God, Spirit, being all, nothing is matter; 4) Life, God, omnipotent GOod, deny death, evil, sin disease; -- disease, sin, evil, death, deny God, omnipotent Good, Life.
\end{enumerate}

To believe such tommmy-rot together with: ``No matter in mind and no mind in matter; no matterin life and no life in matter; no matter in good and no good in matter,'' indicates that there is something the ``matter'' with the minds of Mrs.~Eddy and her followers.

\subsection{SPIRITUALISM}\label{spiritualism}

Spiritualism (Spiritism) is an unchristian cult based on a real or pretended intercourse with the souls of the dead. For the greater part, Spiritistic mediums are tricksters and frauds. In so far as they may commune with the spirits of the departed, they fall under the condemnation of the word of God: ``There shall not be found among you any one\ldots that useth divination, or an observer of times, or an enchanter, or a witch, or a charmer, or a consulter with familiar spirits, or a wizard, or a necromancer.'' (Deut. 18, 10. 11.)

Since 1848, modern Spiritualism has had adherents in the United States. Many are enticed by its trickery and love for the unknown. The Spiritualists do advise their uninitiated to read the Bible, thereby gaining the victims' confidence. These advisors then bring the reader to question certain parts of the Bible. Satan's procedure while he tempted Jesus was similar to the Spiritualists' use of the Bible now. Spiritualism denies that our Lord Jesus is divine; it denies the existence of the devil, demons, and angels. Exponents of Spiritualism say the following about our Bible: ``To assert that it is a holy and divine book, that God inspired the writers to make known His divine will, is a gross outrage on, and misleading, the public.'' (\emph{Outlines of Spiritualism for the Young}.) ``The New Testament is made up of traditions and theological speculations by unknown persons.'' (\emph{Outlines}, p.~13. 14.)

In the Spiritistic book \emph{Whatever Is, Is Right} we find the following information: ``What is evil? Evil does not exist; evil is good. What is a lie? A lie is the truth intrinsically; it holds a lawful place in creation; it is a necessity. What is vice? Vice, and virtue, too, are beautiful in the eyes of the soul. What is murder? Murder is good. Murder is a perfectly natural act.''

Spiritualism leads to infidelity and immorality. According to Mrs.~Woodhull, elected three years in succession as president of the Spiritist societies in America, it is ``the sublime mission of Spiritism to deliver humanity from the thraldom of marriage.'' Dr.~Day, of Montville, Conn., writes: ``It is a fact, and no honest, intelligent Spiritualist can deny its truth, that nine-tenths of modern Spiritualists are, either openly or secretly (as far as they dare), practically \emph{Free Lovers}, in the broadest sense of the word. I am familiar with many of the most prominent leaders, teachers, and mediums of Spiritualism, who are secret agents of Free Love secret circles.'' The same Dr.~Day quotes ``a prominent author and teacher of Spiritualism'' as saying: ``Free Love is the central doctrine of Spiritualism. The new social order is a \emph{social harmony based upon passional attractions}, or the harmony of the varied and developed passional or impulsive nature of man. Attraction is our only law.'' According to Spiritist doctrine, marriate is not a divine institution, in which in reality God joins together one man and one woman, but it is based on the laws of human nature, and is the result from ``natural and spiritual affinities.'' The two parties united are not so much united into one flesh as virtually into one spirit and one soul. Divorces are to be freely granted when desired by both, or even by only one party. ``The marriage vow imposes no obligation on the Spiritualistic husband.'' (\emph{T. L. Harris}.)

Modern Spiritualism emphatically denies the fall of man through the temptation of the devil. This denial is publicly made by the author of \emph{Outlines}. Others deny the existence of the devil; and still another makes a statement, so blasphemous that we can hardly bear repeating it. ``Whom then,'' says he, ``can we believe, God or Satan? The facts justify us in believing Satan. It was not the devil, but God, who made the mistake in the Garden of Eden. It was God, and not the devil, who was the murderer from the beginning.'' This makes any one who has still some moral feeling shudder, and this ought to be enough for any sober-minded man or woman to shun Spiritualistic company.

Mr.~Harrison D. Barrett, president of the National Spiritualists' Association, says that Spiritualism ``steadfastly refused to accept any religious postulates on faith, and at the outset rejected all creeds and dogmatic assumptions of theology.'' This is plain enough. Spiritism rejects the creed of Christianity, and characterizes the saving doctrines of Scriptures as ``dogmatic assumptions.'' By the testimony of its leading exponents, Spiritism is a Christless cult, opposed alike to Christian doctrine and morals. It is one of the false teachings foretold by St.~Paul, when he writes to Timothy: ``Now the Spirit speaketh expressly that in the latter times some shall depart from the faith, giving heed to seducing spirits and doctrines of devils; speaking lies in hypocrisy; having their conscience seared with a hot iron; forbidding to marry\ldots.'' (1 Tim. 4, 1-3).

\subsection{THEOSOPHY}\label{theosophy}

This blasphemous cult is antagonistic to Christianity in three main points.

First, it is pantheistic. Its founder, Madame Blavatsky, says: ``We believe in a universal divine principle, the root All.'' Theosophy rejects a personal God. It believes that God is made up of everything. Horse and star and tree and man are parts of the Theosophists' god.

Secondly, it teaches reincarnation. It says we have three souls, an animal soul, a human soul, and a spiritual soul. The animal soul becomes, after a while, a wandering thing, passing into other human beings. The soul keeps wandering on and on, and may have innumerable different forms. It is simply the old Hindu doctrine of the transmigration of souls, slightly refined to suit European and American tastes. In a country where lizards and cows are not worshipped, it would hardly do to try to proselyte people to the faith that they and their children may be reborn as lizards, cats, or cows! Hence, Theosophy confines reincarnation to the human race, for which merciful limitation I presume we ought all of us simple-minded Christians to be most devoutly thankful!

Their third main point of antagonism to the Christian Religion is the doctrine of the so-called ``Karma,'' or ``The doctrine of consequences.'' It was the doctrine of Buddha and Bob Ingersoll. It is the old heathen fatalism in its barest form. According to the ``Karma,'' you are under the merciless law of cause and effect, to the extent that it is useless to repent; for there is no one to forgive. It is all a question of consequences, -- that's all. Hence there is no place for prayer, repentance, and forgiveness in the Theosophic system.

In Madame Blavatsky's \emph{Key to Theosophy}, a kind of catechism, written evidently for simple-minded Christian people, she makes use of the following dialog: ``Do you believe in God?'' Answer: ``That depends on what you mean by the term.'' ``I mean,'' says the inquirer, ``the God of the Christians, the Father of Jesus, and the Creator, -- the Biblical God of Moses, in short.'' Answer: ``In such a God we do not believe.'' According to the same text-book, Theosophists profess to believe ``in a Universal Divine Principle.'' (p.61.) Other quotations from the \emph{Key}, in which the unchristian character of Theosophy is revealed, are the following: Question: ``Do you believe in prayer, and do you ever pray?'' Answer: ``We do not. We act instead of talking.'' This is consistency, since prayer presupposes a personal and living God. Question: ``Then you also reject resurrection in the flesh?'' Answer: ``Most decidedly we do.''

Theosophy denies that there is ``eternal reward or eternal punishment.'' (p.~108.) It rejects the vicarious atonement of Jesus and the remission of sin. (p.~196.) It is an anti-Christian cult.

Dr.~Talmage said of this sect: ``The most wonderful achievement of the Theosophists is that they keep out of the insane asylum.''

\chapter*{PART II. Doctrines of the Church}\label{part-ii.-doctrines-of-the-church}
\addcontentsline{toc}{chapter}{PART II. Doctrines of the Church}

\section{Concerning the Truth of the Christian Religion}\label{concerning-the-truth-of-the-christian-religion}

\begin{center}
\textsc{Bible Doctrine.}
\end{center}

Of the many religions in the world, ancient as well as modern, the religion of the Bible is the only true one.

Acts 17, 29. 30: ``Forasmuch, then, as we are the offspring of God, we ought not to think that the Godhead is like unto gold, or silver, or stone, graven by art and man's device. And the times of this ignorance God winked at, but now commandeth all men everywhere to repent.''

John 17, 3: ``And this is life eternal, that they might know Thee the only true God, and Jesus Christ, whom Thou hast sent.'' (John 3, 18; 1 Cor. 8, 4; Ex. 20, 3-6.)

\begin{center}
\textsl{Testimony:--}
\end{center}

``Whosoever will be saved, before all things it is necessary that he hold the true Christian faith. Which faith except every one do keep whole and undefiled, without doubt he shall perish everlastingly.'' -- \emph{Athanasian Creed}.

\begin{center}
\textsc{Error.}
\end{center}

\begin{enumerate}
\def\labelenumi{\arabic{enumi}.}
\tightlist
\item
  Carnal mind is in enmity against God. (Rom. 8, 7.) Hence man's unwillingness to be governed by Him. He tries to forgetHim, deny Him; but all to no purpose,-- except, perhaps, for a time. But the devil is always trying some method whereby he may rob man of the Word of God, so that he shall not believe and be saved. (Luke 8, 12.) And one way of accomplishing this is to make him believe that one religion is as good as another, or that Christians ``will not be alone about inheriting heaven.''
\end{enumerate}

The Swiss reformer Zwingli believed ``that all the ancient noble heathens who walked uprightly according to the light they had were also saved.'' (\emph{Expos}. \emph{Fidei}, II, 559.)

The Quaker Barclay, in his \emph{Apology} (10, 2), voices the same thought as regards heathen, Turks, and Jews at the present time.

\begin{center}
\textsl{But God says:--}
\end{center}

Acts 4, 12: ``Neither is there salvation in any other; for there is none other name under heaven given among men whereby we must be saved.'' (See Gospel of St.~John.)

1 Cor. 1, 21: ``For after that, in the wisdom of God, the world by wisdom knew not God, it pleased God by the foolishness of preaching to save them that believe.'' (2 Thess. 1, 7-10: ``punishment for all who do not know and obey the Gospel of our Lord Jesus Christ.'')

2 Thess. 2, 10-12: ``Because they received not the love of the truth that they might be saved, \ldots{} God shall send them strong delusions that they should believe a lie, that they all might be damned who believed not the truth.''

John 14, 6: ``I am the Way, the Truth, and the Life; no man cometh unto the Father but by Me.''

Christ warned against being an ``unbeliever, a heathen and a publican.'' (Matt. 18, 17; Mark 16, 17.)

\begin{enumerate}
\def\labelenumi{\arabic{enumi}.}
\setcounter{enumi}{1}
\tightlist
\item
  A large class of unbelievers call themselves agnostics. This sounds better than atheists and freethinkers, but they are in reality their triplet brother. The ``believ'' of the agnostic is ``We do not know whether there is a God.'' (``The problem of existence has not been solved, and it is insoluble'': (Thomas Huxley.) This dictum has been applied, not only to human existence, but to God Himself, as well as to His revelation.
\end{enumerate}

\begin{center}
\textsl{But God says:--}
\end{center}

John 1, 14: ``And the Word was made flesh, and dwelt among us; and we beheld His glory, the glory as of the Only-begotten of the Father, full of grace and truth.''

1 John 1, 1. 3: ``That which was from the beginning, which we have seen which we have heard, \ldots{} declare we unto you.''

2 Pet. 1, 16: ``For we have not followed cunningly devised fables when we made known unto you the power and coming of our Lord Jesus Christ, but we were eye-witnesses of His majesty.'' (See vv. 17. 18.)

Rom. 1, 19: ``even the heathen acknowledge the existence of God.

\emph{What Is an Atheist?} -- A snare to the simple-minded Christian is the oft repeated assertion that so and so was no atheist, because he had confessed his belief in a personal God or, perhaps, in ``a Higher Power.'' It is true, the words of the Psalmist (14, 1), ``The fool saith in his heart, There is no God,'' describe more nearly the person called ``atheist.'' But all who do not confess the true God are properly to be classed as having no God at all. Christ says: ``He that honoreth not the Son honoreth not the Father.'' (John 5, 23.) ``Whosoever shall deny Me before men, him will I also deny before My Father which is in heaven.'' (Matt. 10, 33.) And the Holy Spirit testifies: ``Whosoever denieth the Son, the same hath not the Father.'' (1 John 2, 23. Compare also 2 John, v. 9.)

\emph{Lodge Religion}. -- Under this head must be ranged all lodges which require a ``belief in a Supreme Being'' from their members, but who do not define this ``Being'' as the Triune God, Father, Son, and Holy Ghost, and studiously avoid using the name of Jesus in their so-called prayers. Silently (and here intentionally) to omit a confession of the Savior is to deny Him, and be classed as heathen. (Matt. 18, 17.) And Christ explicitly teaches that He is the Way, the Life, and the Truth, and that no man cometh unto the Father but by Him. (John 14, 6.)

\emph{Evolution}. -- The notion is also prevalent that Christianity, revealed religion, ``is subject to the laws of evolution.'' Aside from the fact that ``the laws of evolution'' are only imaginings of depraved human nature, which fain would escape just punishment at the hands of God, ``progressive Christianity'' is simply an absurdity. What of a Christianity in which the life, teachings, and deeds of a Christ Jesus, little by little, should be superseded by -- what? Can you think of a moral law which could supersede the Ten Commandments? Of a just God more righteous than Jehovah? more loving than Jesus? more merciful than He who sent His only-begotten Son as the Savior for the whole world? more merciful than He who embraces the Prodigal (Luke 15, 20), pardons the adulteress (John 8, 11), promises heaven to the thief on the cross (Luke 29, 43), and prayed even for His enemies?

No doubt, it is considered the height of wisdom at the present time to assert that ``the law of evolution'' also underlies the Christian religion. But what does history tell us concerning those who have come under its spell? Heathenism is nothing but the result of a trial to improve on the revelation and government of God. The Babylonian captivity of the Jews and the final dispersement of them over all the earth is the outcome of a similar experiment. The early Christian Churches in Asia and elsewhere tried their hands at the same thing, and were rewarded by -- Mohammedanism. And what is popery, Jesuitism, rationalism but the outcome of efforts to apply ``the law of evolution'' to religion! Evolution, indeed, but \emph{always} an evolution downward.

Though the theory on which Darwinian evolution rests has been discarded as insufficient by conscientious scientists, yet the reechoing of it is still heard in some of our higher institutions of learning, not to mention our normal schools, where our common-school teachers are trained. Hence we often read of this or that ``professor,'' in our daily papers, who adjudges the remains of some prehistoric (?) animal as being so or so many millions of years old. These opinions finally find their way into text-books and are henceforth promulgated as ``gospel truths.'' But what reliance may be put upon them the following well-supported facts will show.

\begin{enumerate}
\def\labelenumi{\arabic{enumi}.}
\item
  The Neanderthal Skull, has, since 1856, been offered as one of the chief proofs for the evolution of man from a lower animal, and has been considered about 300,000 years old. Dr.~Meyer, of Bonn, proved that it is the skull of a Russian Cossack killed in 1814.
\item
  The Calaveras Skull, in the California State Museum, was placed in a mine-shaft, 150 feet deep, as a practical joke.
\item
  The ``Colorado Specimen'' was pronounced by a Columbia College ``scientist'' to be ``the missing link,'' one million and a half years old. Cowboys since proved the bones to be that of a pet monkey which they had buried there.
\item
  The Croatian Skeletons (Austria) are the bones of degenerate human beings, -- nothing more.
\item
  The Pithecanthropus Erectur has for some years (since 1891) been the most popular relic with evolutionists. The collection consisted of a skull above the eyes, a leg-bone, and two teeth, -- found in Java at separate places. The age of this ``missing link'' was guessed by various scientists to be from one hundred thousand to one million years! According to measurements the skull is that of an idiot. (See \emph{The Other Side of Evolution}, Alex. Patterson, D. D. Winona Publ. Co., Chicago, 1903.)
\end{enumerate}

\section{The Holy Scriptures}\label{the-holy-scriptures}

\begin{center}
\textbf{A.  The Bible Is God's Revelation to Mankind}

\textsc{Bible Doctrine.}
\end{center}

The Bible, comprising the canonical books of the Old and the New Testament, is God's unerring and only revelation to mankind.

Luke 16, 29: ``They have Moses and the prophets; let them hear them.''

1 Thess. 2, 13: ``When ye received the Word of God which ye heard of us, ye received it not as the word of men, but, \emph{as it is in truth}, the Word of God.''

John 17, 17: ``Sanctify them through Thy truth; Thy Word is truth.''

\begin{center}
\textsl{Lutheran Testimony}
\end{center}

``We believe, teach, and confess that the only rule and standard according to which at once all dogmas and teachers should be esteemed and jusged are nothing else than the prophetic and apostolic Scriptures of the Old and the New Testament, as it is written: `Thy Word is a lamp unto my feet and a light unto my path' (Ps. 119, 105), and St.~Paul: `Though an angel from heaven preach any other gospel unto you, let him be accursed' (Gal. 1, 8). (\emph{Form. Con.}, Epit., Introd., § 1.)

\begin{center}
\textsc{Error.}
\end{center}

\begin{enumerate}
\def\labelenumi{\arabic{enumi}.}
\tightlist
\item
  The evolutionists one and all discredit the narrative in Genesis, first chapter, concerning creation. Of the Bible, as revelation, Haeckel says: ``It is the invention of man's imagination. The so-called truth which believers find in it is a human speculation, and the `childlike' faith in this irrational revelation is only superstition.'' (\emph{The Riddle of the Universe}, 123.)
\end{enumerate}

\begin{center}
\textsl{But God says:--}
\end{center}

Rom. 3, 1. 2: ``What advantage, then hat the Jew? Much every way: chiefly, because that unto them were commiteed \emph{the oracles of God}.''

Heb. 1, 1. 2: ``God, who at sundry times and in diverse manners spake in time past unto the fathers by the prophets, \emph{hath in these last days spoken unto us} by His Son.''

1 Pet. 1, 16: ``For we have \emph{not followed cunningly devised fables} when we made known unto you the power and coming of our Lord Jesus Christ, but were eye-witnesses of His majesty.''

\begin{enumerate}
\def\labelenumi{\arabic{enumi}.}
\setcounter{enumi}{1}
\tightlist
\item
  The so-called higher critics fain would find all kinds of flaws and faults in the Bible: Creation is a myth; the patriarchs never lived, at least not at the time states; Moses, and most of the other authors, never wrote the books ascribed to them; the contents are unreliable; the so-called prophecies are post-event descriptions, etc., etc.
\end{enumerate}

\begin{center}
\textsl{But God says:--}
\end{center}

1 Tim. 4, 1: ``Now \emph{the Spirit speaketh} expressly that in the latter times some shall depart from the faith, giving heed to \emph{seducing spirits} and doctrines of devils.''

John 1, 1-3: ``In the beginning was the Word, and the Word was with God, and the Word was God. All things were made by Him, and without Him was not anything made that was made.''

Luke 20, 37. 38: ``The Lord is the God of Abraham and the God of Isaac and the God of Jacob; for He is not a God of the dead, but of the living.''

Luke 24, 27: ``And \emph{beginning at Moses} and all the prophets, He expounded unto them in all the Scriptures the things concerning Himself.''

John 6, 63: ``The words that I speak unto you, they are spirit, and they are life.''

Matt. 24, 35: ``Heaven and earth shall pass away, but My words shall not pass away.''

John 12, 48: ``He that rejecteth Me, and receiveth not My words, hath one that judgeth him; the word that I have spoken, the same shall judge him in the last day.''

John 10, 35: ``The Scripture cannot be broken.''

\textsc{The Apocryphal Books.} -- Some editions of the Bible contain a number of books called Apocrypha (hidden). The name implies that the authors are unknown, and their writings of doubtful value.

The Apocrypha of the Old Testament comprise some fourteen books and those of the New Testament about twenty-five. The latter have never appeared in any Protestant Bible.

In 1546, the Catholic Church decreed, through the Council of Trent, that the old Testament apocryphal books were to be included among the canonical (rule-giving) books of the Bible. This was done in order to gain certain proof-texts for the doctrines of purgatory, prayers for the dead, and for strengthening the position of the pope generally. However, no council can decree what shall, or shall not, be the Word of God. The canonicity of the Bible is deduced from a multitude of proofs peculiarly its own. The following, among many, may suffice here. 1) The declaration of the different writers that God has spoken through them. 2) The loftiness and sublimity of the Bible-text in comparison with that of secular books. 3) The united testimony of the Church concerning both the Old and the New Testament. 4) The repeated references of Christ and the apostles to numberless statements in the canonical books, but silence as to the apocryphal. 5) The sufficiency of the canonical Scriptures as declared by Christ Himself. (Luke 16, 31; John 5, 39; 10, 35.)

\begin{center}
\textbf{B.  The Bible is Given by Inspiration of God.}

\textsc{Bible Doctrine.}
\end{center}

The contents of the Bible were revealed to the sacred writers by inspiration of the Holy Spirit

2 Tim. 3, 16: ``All Scripture is given \emph{by inspiration of God}, and is profitable for doctrine, for reproof, for correction, for instruction in righteousness.''

2 Pet. 1, 21: ``For the prophecy came not in old time by the will of man, but holy men of God \emph{spake as they were moved by the Holy Ghost}.''

\begin{center}
\textsl{Testimony of the Creed:--}
\end{center}

``And I believe in the Holy Ghost, the Lord and Giver of life; \ldots{} who spake by the prophets.'' -- \emph{Nicene Creed}, § 7.

\begin{center}
\textsc{Error.}
\end{center}

The great majority of modern exegetes, especially the so-called higher critics, maintain that ``science'' forbids the acceptance of an inspiration different from that of a Shakespeare or a Milton. And even if the different authors were under the influence of a higher power, they were children of their times, and hence their opinions were biased, and they were subject to errors in judgment, etc.

\begin{center}
\textsl{But God says:--}
\end{center}

1 Cor. 2, 12. 13: ``NOw we have received, not the spirit of the world, but the Spirit which is of God, that we might known the things that are freely given to us by God. Which things also we speak, not in the \emph{words} which man's wisdom teacheth, but \emph{which the Holy Ghost teacheth}, comparing spiritual things with spiritual.''

1 Cor. 2, 7. 10: ``But we speak the wisdom of God in a mystery, even the hidden wisdom, which God ordained before the world unto our glory; which none of the princes of this world knew. But \emph{God hath revealed them unto us} by His Spirit.''

Matt. 10, 20: ``For it is not ye that speak, but the Spirit of your Father which speaketh in you.'' (2 Sam. 23, 2. 3.)

1 Cor. 1, 23: ``But we preach Christ crucified, unto the Jews a stumbling-block and unto the Greeks foolishness.''

Gal. 1, 20: ``Now the things which I write unto you, behold before God, I lie not.'' (1, 8. 9.)

\textsc{Note.} -- A great many theologians at the present time have given up the belief in verbal inspiration, and contend that God only gave the authors an ``impulse'' to write, and what to write about, or, at best, that the ``concept,'' or ``idea,'' was inspired. But this, together with the assertion that the different authors have been mistaken in several things, is simply destroying the foundation of all faith and hope. If God has not told us in explicit words what His will is, the hope of the Christian is no better than that of a Rationalist.

But a few passages from Scripture will show how the Bible stands on the question.

Ex. 4, 12: ``Now, therefore, go, and I will be with thy mouth, and teach thee what thou shalt speak.''

Ex. 34, 27: ``And the Lord said unto Moses, Write thou these words; for after the tenor of these words I have made a covenant with thee and with Israel.''

2 Sam. 23, 2: ``The Spirit of God spake by me, and His Word was upon my tongue.'' (Of this Christ says in Mark 12, 36: ``David himself said \emph{by the Holy Ghost}.'')

Acts 2, 4: ``And they were filled with the Holy Spirit, and began to speak with other tongues as the Spirit gave them utterance.''

Mark 13, 11: ``For it is not ye that speak, but the Holy Ghost.''

In almost innumerable places we read: ``Thus saith the Lord''; ``The Lord said''' ``The word of the Lord,'' and the like, especially numerous in the writings of the prophets. So also the message of the apostles:--

1 Cor. 2, 13: ``Which things also we speak, not in words which man's wisdom teacheth, but which the Holy Ghost teacheth.''

1 Thess. 2, 13: ``When ye received the Word of God which ye heard of us, ye accepted it not as the word of men, but, as it is in truth, the Word of God.''

\begin{center}
\textbf{C.  The Interpretation of the Bible.}

\textsc{Bible Doctrine.}
\end{center}

The meaning of the words of Scripture must be sought in the Scriptures themselves, not outside of them.

2 Pet. 4, 11: ``If any man speak, let him speak as the oracles of God.''

\begin{center}
\textsl{Lutheran Testimony:--}
\end{center}

``We have, however, another rule, \emph{viz}.: That the Word of God should frame articles of faith; otherwise no one, not even an angel.'' -- \emph{Smalcald Articles}, II, 2, 15.

\begin{center}
\textsc{Error.}
\end{center}

Zwingli and Calvin both declared that human reason must be the judge of Holy Scripture. Zwingli gives this counsel: ``Philosophical argumentation through conclusions of reasoning must not be neglected.'' -- \emph{Expos. Chr. Faith}.

Calvin says: ``Can you prove through reasoning that this contains nothing unreasonable? Yes, if it be conceded that the Lord has instituted nothing that is contrary to reason.'' -- \emph{Geneva Catechism}.

This is the fundamental error of all sects, ancient and modern. And the Reformed Churches, one and all, carry the image of their originators.

\begin{center}
\textsl{But God says:--}
\end{center}

Is. 8, 20: ``To the Law and to the testimony: if they speak not according to this Word, it is because there is no light in them.''

2 Cor. 10, 5: ``We bring into captivity every thought to the obedience of Christ.''

1 Cor. 2, 14: ``But the natural man receiveth not the things of the Spirit of God; for they are foolishness unto him; neither can he know them, because they are spiritually discerned.''

Rom. 9, 20: ``O man, who art thou that repliest against God?''

\textsc{Note.} -- Under the topic ``Means of Grace'' the doctrine of Holy Scripture will be further discussed.

\section{Original Sin}\label{original-sin}

\begin{center}
\textsc{Bible Doctrine and Lutheran Testimony}
\end{center}

``All men, after Adam's fall, begotten after the common course of nature, are born with sin, that is, without the fear of God, without trusting Him, and with fleshly appetites; at that, this disease, or original fault, is truly sin, condemning and bringing eternal death now also upon all that are not born again by the Holy Spirit.'' -- \emph{Augsb. Conf.}, Art. II.)

Gen.~8, 21: ``The imagination of man's heart is evil from his youth.'' (6, 5.)

Eph. 2, 1: ``Ye were dead in trespasses and sin.''

Rom. 3, 22. 23: ``There is no difference; for all have sinned, and come short of the glory of God.''

Eph. 2, 3: ``We were by nature the children of wrath.''

Ps. 51, 5: ``I was shapen in iniquity, and in sin did my mother conceive me.''

Rom. 5, 12: ``By one man sin entered into the world, and death by sin; and so death passed upon all men, for that all have sinned.''

\begin{center}
\textsc{Error.}
\end{center}

The Reformed Churches do not express themselves with sufficient exactness in their definition of Original Sin. The \emph{Methodists}, for instance, say in their \emph{Articles of Religion}: ``Man is very far gone from original righteousness.'' (Art. 7.) To this a Methodist writer makes the remark that ``how far is not stated. Obviously the meaning is that the image of God is not wholly destroyed.'' The practise in the Methodist Church corroborates this assertion. The doctrine of Original Sin is now very rarely taught from Reformed pulpits, and has all but disappeared from the sectarian press.

\begin{center}
\textsl{But God says:--}
\end{center}

Rom. 7, 18: ``For I know that in me, that is, in my flesh, dwelleth no good thing.''

Rom. 8, 7: ``The carnal mind is enmity against God; for it is not subject to the Law of God, neither, indeed, can be.'' (1 Cor. 2, 14; John 3, 6.)

Ps. 53, 3: ``The wicked are estranged from before birth; they go astray as soon as they be born, speaking lies.'' (Is. 48, 8.)

\section{Redemption}\label{redemption}

\begin{center}
\textbf{A.  Redemption Is Through Christ.}

\textsc{Bible Doctrine.}
\end{center}

Gal. 3, 13: ``Christ has redeemed us from the curse of the Law, being made a curse for us.''

Matt. 5, 17: ``Think not that I am come to destroy the Law or the prophets; I am not come to destroy, but to fulfill.'' (3, 15.)

Rom. 8, 3. 4: ``For what the Law could not do, in that it was weak through the flesh, God, sending His own Son in the likeness of sinful flesh, and for sin, condemned sin in the flesh, that the righteousness of the Law might be fulfilled in us, who walk not after the flesh, but after the Spirit.''

\begin{center}
\textsl{Lutheran Testimony:--}
\end{center}

``For since Christ is not alone man, but God and man in one undivided Person, He was as little subject to the Law, because He is the Lord of the Law, as, in His own Person, to suffering and death. Therefore His obedience, not only in suffering and dying, but also that He in our stead was voluntarily subject to the Law, and fulfilled it by His obedience, is imputed to us for righteousness, so that, on account of this complete obedience, which by deed and by suffering, and in life and in death, He rendered His heavenly Father for us, God forgives our sins, regards us godly and righteous, and eternally saves us.'' -- \emph{Form. Conc.}, III, § 15.

\begin{center}
\textsc{Error.}
\end{center}

The \emph{Moravians} (Brethren) and \emph{Methodists} speak in their catechisms, under this head, of Christ, only as a \emph{type}, or \emph{example}. ``How is the life of Christ on earth to be considered? His life on earth was perfectly holy, and is therefore an example for us to follow.'' (\emph{Mor. Cat.}, Qu. 48.) -- ``What does the life of Christ represent? A prototype of perfect goodness and holiness.'' -- \emph{Meth. Cat.}, No.~3, p.~23.

\textsc{Note.--} Present-day Christianity, among the sects and enthusiasts, is all of this variety. ``Be good,'' ``Follow in His steps,'' are slogans heard on every hand. Forsooth, true Christians also strive to be good, and to follow in the footprints of the Savior. But these things do not \emph{make}, but only \emph{adorn} a Christian. We are saved, not by following ``in His steps,'' but by accepting His redemption.

\begin{center}
\textsl{But God says:--}
\end{center}

Gal. 4, 4. 5: ``But when the fulness of the time was come, God sent forth His Son, amde of a woman, made under the Law, to redeem them that were under the Law, that we might receive the adoption of sons.''

Rom. 10, 4: ``For Christ is the end of the Law for righteousness to every one that believeth.'' (Phil. 2, 7. 8)

\begin{center}
\textbf{B.  Redemption Is Universal}

\textsc{Bible Doctrine.}
\end{center}

Christ has through His suffering and death redeemed the whole world.

Rom. 5, 19: ``For as by one man's disobedience many were made sinners, so by the obedience of One shall \emph{many} be made righteous.''

John 1, 29: ``Behold the Lamb of God, which taketh away the sins of the \emph{world}.'' (3, 16; 12, 47.)

Gen.~22, 18: ``In thy seed shall \emph{all nations of the earth} be blessed.''

Luke 2, 10: ``I bring you good tidings of great joy, which shall be to \emph{all people}.''

2 Pet. 3, 9: ``The Lord is not willing that any should perish, but that all should come to repentance.'' (Rom. 11, 32.)

\begin{center}
\textsl{Lutheran Testimony:--}
\end{center}

``The first and chief articles is this, that Jesus Christ, our God and Lord, died for our sins, and was raised again for our justification. Rom. 4, 25. And He alone is the Lamb of God, which taketh away the sins of the world (John 1, 29); and God has laid upon Him the iniquities of us all. Is. 53, 6.'' -- \emph{Smalc. Art.}, II, 1. 2.

\begin{center}
\textsc{Error.}
\end{center}

\emph{Presbyterians, Congregationalists}, and \emph{Calvinistic Baptists} state in their confessions that Christ has redeemed the elect \emph{only}. In the \emph{Westminster Confession} we read: ``As God hath appointed the elect unto glory, He, by the eternal and most free purpose of His will, foreordained all the means thereunto\ldots. \emph{Neither are any other redeemed by Christ, effectually called}, justified adopted, sanctified, and saved, but the elect only.'' -- (III, § 6.) And again: ``The Lord Jesus, by His perfect obedience and sacrifice of Himself, \ldots{} hath fully satisfied the justice of His Father, and purchased not only reconciliation, but an everlasting inheritance in the kingdom of heaven, \emph{for all those whom the Father hath given unto Him}'' (\emph{i. e.}, for the elect \emph{only}).--\emph{Westm. Conf.}, VIII, 5. (See also \emph{Savoy Declaration} and \emph{Bapt. Conf}. of 1688.)

\begin{center}
\textsl{But God says:--}
\end{center}

2 Cor. 5, 15: ``He died \emph{for all}, that they which live should not henceforth live unto themselves, but unto Him which died for them and rose again.''

Matt. 18, 11: ``The Son of Man is come to save \emph{that which was lost}.''

1 Tim. 4, 10: ``We trust in the living God, who is the Savior of \emph{all men}, specially of those that believe.''

1 John 2, 1. 2: ``If any man sin, we have an Advocate with the Father, Jesus Christ the Righteous; and He is the propitiation for our sins, and not for ours only, but also for the sins of the \emph{whole world}.'' (rom. 5, 10; Is. 53.)

2 Pet. 2, 1: ``Denying the Lord that brought them, and bring upon themselves swift destruction.'' (Christ died also for those who by reason of their unbelief are finally lost.)

\section{Conversion}\label{conversion}

\begin{center}
\textsc{Bible Doctrine.}
\end{center}

``Conversion is not an effect of man's own power and strength, but an act and gift of the Holy Spirit.'' -- \emph{Chemnitz's Enchiridion}.

Acts 11, 18: ``God hath also to the Gentiles granted repentance unto life.''

Phil. 2, 13: ``For it is God which worketh in you both to will and to do of His good pleasure.''

Heb. 13, 20. 21: ``Now the God of peace \ldots{} make you perfect in every good work to do His will, working in you that which is well-pleasing in His sight.''

John 15, 5: ``Without Me ye can do nothing.''

2 Cor. 3, 5: ``Not that we are sufficient in ourselves to think anything, as of ourselves, but our sufficiency is of God.''

\begin{center}
\textsl{Lutheran Testimony:--}
\end{center}

``In spiritual and divine things the intellect, heart, and will of the unregenerate man cannot, in any way, by their own natural power, understand, believe, accept, think, will, begin, effect, do, work, or concur in working anything, but they are entirely dead to good and corrupt; so that in man's nature, since the fall, there is, before regeneration, not the least spark of spiritual power remaining still present, by which, of himself, he can prepare himself for God's grace, or accept the offered grace, or, for and of himself, be capable of it, or apply or accommodate himself thereto, or, by his own powers, be able of himself, as of himself, to aid, do, work, or concur in working anything for his conversion, either entirely, or in half, or in even the least or most inconsiderable part, but he is the servant and slave of sin (John 8, 34; Eph. 2, 2; 2 Tim. 2, 26). Hence the natural free will, according to its perverted disposition and nature, is strong and active only with respect to what is displeasing and contrary to God.'' -- \emph{Form. Conc.}, Art. II, § 7.

\begin{center}
\textsc{Error.}
\end{center}

The so-called \emph{Free-will Baptists} tell by their very name what they teach about conversion. But not so the \emph{Methodists}. In section 8 of their \emph{Articles of Religion} they seem to disclaim the concurrence of the Holy Spirit, but their chief teachers are self-confessed \emph{Synergists}, and their whole evangelistic and revivalistic work is based on the old heretical belief in freedom of the will. John Wesley says in his \emph{Scripture Doctrine of Predestination}: ``Oh, you are a defender of the freedom of the will, you assume a free will in man! I assume nothing but what Scripture says, and that you ought to allow me to assume. I do not assume that a man has will or power to do anything good of himself, but by the grace of God we may do everything. We believe that the moment Adam fell, he had no more freedom of the will, but that when God, in His free grace, promised him and his seed a Savior, He then \emph{returned} to humanity a free will and power to accept the proffered salvation.'' Professor W. F. Warner, in his \emph{Systematic Theology}, speaks in similar terms. Also T. N. Ralston, in his \emph{Outlines of Theology}. \emph{The Cumberland Presbyterians} also err here. In their \emph{Origin and Doctrine} we read: ``We object to the idea that man is passive in matters pertaining to salvation until God revives and renews him.'' (p.~86.)

\begin{center}
\textsl{But God says:--}
\end{center}

Jas. 1, 18: ``\emph{Of His own will} begot He us with the Word of Truth, that we should be a kind of first-fruits of His creatures.''

1 Cor. 12, 3: ``No man can say that Jesus is the Lord \emph{but by the Holy Ghost}.''

Jer. 13, 23: ``Can the Ethiopian change his skin, or the leopard his spots? Then may ye also do good that are accustomed to do evil.''

Matt. 7, 16: ``Do men gather grapes of thorns or figs of thistles?'' (12, 34.)

\textsc{Free Will.--} We often hear this expression: ``We have a free will, and can do as we please.'' To this the \emph{Augsburg Confession} makes this reply: ``Concerning free will they teach that man's will has some liberty to work a civil righteousness, and to choose between things that are subject to human reason, but that it has no power to work the righteousness of God, or a spiritual righteousness, without the Spirit of God, because that the natural man receives not the things of the Spirit of God, 1 Cor. 2, 14. But this is wrought in the heart when man receives the Spirit of God through the Word.'' -- Art XVIII.

\section{Predestination}\label{predestination}

\begin{center}
\textsc{Bible Doctrine.}
\end{center}

\begin{enumerate}
\def\labelenumi{\alph{enumi}.}
\tightlist
\item
  God has shown mercy to all mankind, and wishes all to be saved.
\end{enumerate}

1 Tim. 2, 4: ``God will have all men to be saved, and to come unto the knowledge of the truth.''

2 Pet. 3, 9: ``The Lord is not willing that any should perish, but that all should come to repentance.''

Ezek. 33, 11: ``As I live, saith the Lord God, I have no pleasure in the death of the wicked, but that the wicked turn from his way and live.'' (18, 33; John 3, 16.)

\begin{center}
\textsl{Lutheran Testimony:--}
\end{center}

``Therefore, if we wish with profit to consider our eternal election to salvation, we must ine very way hold rigidly and firmly to this, \emph{viz}., that, as the preaching of repentance, so also the promise of the Gospel is universal, \emph{i. e.}, it pertains to all men (Luke 24, 47). Therefore Christ has commanded `that repentance and remission of sins should be preached in His name among all nations.' For God so loved the world, and gave His Son (John 3, 16). Christ bore the sins of the world (John 1, 29), gave His flesh for the life of the world (John 6, 51); His blood is the propitiation for the sins of the whole world (1 John 1, 7; 2, 2),'' etc. -- \emph{Form. Conc.}, Art. XI, 28.

\begin{center}
\textsc{Error.}
\end{center}

To be a \emph{Calvinsit} is to believe what has been properly called the ``monstrous doctrine'' that God does \emph{not} desire the salvation of \emph{all} mankind. IN his \emph{Institutes} Calvin declares that ``by predestination we mean the eternal decree of God, by which He determined with Himself whatever He wished to happen with regard to every man. All are not created on equal terms, but some are preordained to eternal life, \emph{others to eternal damnation}; and, accordingly, as each has been created for one or other of these ends, we say that he has been predestined to life \emph{or death}.'' (III, 21, 5.) ``He arranges all things by His sovereign counsel, in such a way that individuals are born who are doomed from their inception to certain death, and are to glorify Him by their destruction.'' (III, 23, 6.)

\begin{center}
\textsl{But God says:--}
\end{center}

Rom. 11, 32: ``God hath concluded them \emph{all} in unbelief, that He might have mercy \emph{upon all}.''

1 John 2, 2: ``And He is the propitiation for our sins, and not for ours only, but also for the sins of the \emph{whole world}.''

Matt. 23, 37: ``O Jerusalem, Jerusalem, \ldots{} how often would I have gathered thy children together, even as a hen gathereth her chickens under her wings, and \emph{ye would not}.''

\begin{center}
\textsc{Bible Doctrine.}
\end{center}

\begin{enumerate}
\def\labelenumi{\alph{enumi}.}
\setcounter{enumi}{1}
\tightlist
\item
  In choosing His elect, God did not proceed in an arbitrary absolute manner, but chose them in Christ Jesus.
\end{enumerate}

Eph. 1, 3-6: ``Blessed be the God and Father of our Lord Jesus Christ, who hath blessed us with all spriitual blessings in heavenly places in Christ; according as He hath chosen us \emph{in Him} before the foundation of the world, that we should be holy and without blame before Him in love; having predestinated us unto the adoption of children by Jesus Christ to Himself, according to the good pleasure of His will, to the praise of the glory of His grace, wherein He hath \emph{made us accepted in the Beloved}.''

\begin{center}
\textsl{Lutheran Testimony:--}
\end{center}

``Therefore this eternal election of God is to be considered in Christ, and not beyond or without Christ. For `in Christ,' testifies the Apostle Paul (Eph. 1, 4 sq.), `He hath chosen us before the foundation of the world,' as it is written: `He hath made us accepted in the Beloved.'\,'' -- \emph{Form. Conc.}, XI, § 65 (5. 9).

\begin{center}
\textsc{Error.}
\end{center}

\begin{enumerate}
\def\labelenumi{\arabic{enumi}.}
\item
  The fountainhead of modern error concerning this doctrine is \emph{John Calvin}. To him predestination was the central doctrine of the Christian religion. And more, -- predestination he considers strictly an arbitrary act of God. ``I repeat,'' he says, ``that we must ever return to the mere pleasure of the divine will.'' -- \emph{Iinst.}, III, 23, 4.
\item
  The \emph{Reformed Church}, specially so called, expresses its belief thus: ``If we are not to be ashamed of the Gospel, we must confess what it clearly teaches, that God, according to His eternal pleasure and moved by no other cause, has ordained some unto salvation, whereas others were rejected.'' -- \emph{Cons. Genev.}, 224.
\item
  \emph{Congregationalists} and \emph{Baptists}, in their respective confessions, agree with the \emph{Presbyterians} when the latter say: ``By the decree of God, for the manifestation of His glory, some men and angels are predestined unto everlasting life, and others foreordained to everlasting death.'' -- \emph{Westm. Conf.}, III, 3, 7.
\end{enumerate}

\textsc{Note.} -- In 1887 and 1903, the Presbyterian Church (North) added two chapters to their confession (34, 35) and also appended a \emph{Declaratory Statement} concerning predestination. According to this ``Statement,'' Chapter III is to be understood as \emph{not} teaching foreordination to death; and Chapter X, 3, that no children dying in infancy are lost. But as long as the confession is still in force, -- and all did not vote for the ``Statement,'' -- the situation is practically unchanged.

\begin{center}
\textsl{But God says:--}
\end{center}

Hos. 3, 9: ``O Israel, thou hast destroyed thyself, but in Me is thine help.''

John 1, 39: ``Behold the Lamb of God, which taketh away the sins of \emph{the world}.''

Titus 2, 11: ``For the grace of God that bringeth salvation hath appeared to \emph{all men}.'' (2 Pet. 2, 1.)

\textsc{Note.} -- The error of Calvinism does not only consist in denying an election \emph{in Christ}, but especially in teaching an \emph{election to perdition}. The Lutheran Church teaches, according to Scripture, that there is an election unto salvation, but \emph{not} unto condemnation. The \emph{Formula of Concord} (XI, 5) says: ``But the eternal election of God, or predestination, \emph{i. e.}, God's appointment to salvation, pertains not at the same time to the godly and the wicked, \emph{but only to the children of God}.''

On an other phase of this mysterious question we will let \emph{Dr.~Martin Chemintz} speak. He was not only the author of the Eleventh Article in the \emph{Formula of Concord}, but, at the time of its composition, superintendant of the churches of Braunschweig. In his \emph{Enchiridion}, a work written especially for the instructino of the clergy in his diocese, he puts the following question: ``Does God make such election first after a time, as when men have become penitent and believers, or is it made on account of their piety which God foresees?'' \emph{Answer}: ``St.~Paul says in Eph. 1, 4: `He hath chosen us in Christ before the foundation of the world'; and in 2 Tim. 1, 9: `He hath saved us, and called us with an holy calling, not according to our works, but according to His own purpose and grace, which was given us in Christ Jesus before the world began.' Hence, the electino of God does not follow after our faith and righteousness, but goes before them, as a cause of it all. Moreover, whom He did predestinate, them He also called; and whom He called, them He also justified, Rom. 8, 30. And in Eph. 1, 4, Paul does not say that we were elected because we \emph{were} holy or had \emph{become} holy, but He says: He hath chosen us that we should \emph{be} holy. For predestination is a cause of everything pertaining to salvation, as Paul says: `We have obtained an inheritance, being predestinated according to the purpose of Him who workth all things after the counsel fo His own will, that we should be to the praise of His glory, who first trusted in Christ.' (Eph. 1, 11. 12.) And this election is not made in view of our present or future deeds, but according to His own purpose and grace. (Rom. 9; 2 Tim. 1.)''

\section{Justification}\label{justification}

\begin{center}
\textsc{Bible Doctrine.}
\end{center}

Rom. 8, 33: ``Who shall lay anything to the charge of God's elect? It is God that justifieth.''

Gal. 3, 6: ``Even as Abraham believed God, and it was accounted to him for righteousness.''

Eph. 2, 8. 9: ``For by grace are ye saved, through faith, and that not of yourselves, it is the gift of God; not of works, lest any man should boast.''

Rom. 3, 24: ``Being justified freely by His grace through the recemption that is in Christ Jesus.'' 3, 28: ``Therefore we conclude that a man is justified by faith, without the deeds of the Law.''

\begin{center}
\textsl{Lutheran Testimony:--}
\end{center}

``Men cannot be justified before God by their own strength, merits, or works, but are freely justified for Christ's sake, through faith, when they believe that they are received into favor, and that their sins are forgiven for Christ's sake, who, by His death, hath made satisfaction for our sins. This faith God imputes for righteousness in His sight. (Rom. 3 and 4.)'' -- \emph{Augsb. Conf.}, Art. IV.

``This article concerning justification by faith is the chief in the entire Christian doctrine, without which no poor conscience has any firm consolation, or can know aright the riches of the grace of Christ, as Dr.~Luther also has written: `If only this article remain in view pure, the Christian Church also remains pure, and is harmonious and without all sects; but if it do not remain pure, it is not possible to resist any error or fanatical spirit.' And concerning this article Paul especially says that a little leaven leaveneth the whole lump. Therefore in this article he emphasizes with so much zeal and earnestness the exclusive particles, or the word whereby the works of men are excluded (namely, `without Law,' `without works,' `out of grace' `freely,' Rom. 3, 28; 4, 5; Eph. 2, 8. 9), in order to indicate how highly necessary it is that in this article, by the side of the presentation of the pure doctrine, the antithesis, \emph{i. e.}, all contrary dogmas, by this means be separated, exposed, and rejected\ldots{}

``Concerning the righteousness of faith before God we unanimously believe, teach, and confess, according to the comprehensive summary of our faith and convession above presented, \emph{viz.}, that a poor sinful man is justified before God, \emph{i. e.}, absolved and declared free and exempt from all his sins and from the sentence of a well-delivered condemnation, and adopted into sonship and heirship of eternal life, without any merit or worth of his own, also without all preceding, present, or subsequent works, out of pure grace, alone because of the sole merit, complete obedience, bitter suffering, death, and resurrection of our Lord Christ, whose obedience is reckoned to us for righteousness.'' -- \emph{Form. Conc.}, III, 6. 7. 9.

\begin{center}
\textsc{Error.}
\end{center}

\begin{enumerate}
\def\labelenumi{\arabic{enumi}.}
\tightlist
\item
  The \emph{Catholic Church}, the mother of heresies regarding justification, teaches according to its apologist, J. Adam Moehler: ``The Council of Trent represents justification as a renewal of the inward man, by means whereof we become really just, as inherent in the believer, and as a restoration of the primeval state of humanity. On this account, the same council observes that by the act of justification, faith, hope, and charity are infused into the heart of man; and that it is only in this way he is truly united with Christ and becometh a living member of His body. In other words, justification is considered to be sanctification and forgiveness of sins, as the latter is involved in the former, and the former in the latter; it is considered an infusion of the love of God into our hearts through the Holy Spirit; and the interior state of the justified man is regarded as holy feeling, as a sanctified inclination of the will, as habitual pleasure and joy in the divine Law, as a decided and active disposition to fulfil the same in all the occurrences of life, -- in short, as a way of feeling which is in itself acceptable and well pleasing to God.'' -- \emph{Symbolism}, p.~104.
\end{enumerate}

\textsc{Note.} -- It was on account of this grand confusion of ``infusion,'' ``sanctification,'' ``feeling,'' and ``inclination'' that Luther could find no ``joy in the divine Law'' nor peace in the Catholic Church. This came to him only after learning the meaning of Paul's words: ``Justified by faith, we have peace with God through our Lord Jesus Christ.'' (Rom. 5, 1.)

\begin{enumerate}
\def\labelenumi{\arabic{enumi}.}
\setcounter{enumi}{1}
\tightlist
\item
  To receive salvation through grace alone is repugnant to self-seeking, guilt-feeling human nature. Hence not only the pope, but many others have invented schemes of salvation which also include more or less of human efforts. As representatives of these may be mentioned the \emph{Arminians}. Their celebrated teacher Limborch writes: ``Be it known that we, when we state that we are justified through faith, do not exclude works, but rather include them.'' -- \emph{Chr. Theol.}, VI, 4, 22.
\end{enumerate}

The \emph{Quakers}, representing more nearly our every-day, mushroom religious reformers, teach ``that we cannot, as some Protestants unwisely have done, exclude good works from justification. They are absolutely necessary, forsooth, not as a case on account of which we are justified, but as something \emph{in} which we are justified, and without which we cannot be justified.'' -- \emph{Barclay}, \emph{Apol.}, 7, 3. 4.

\textsc{Note.} -- Theoretically, the Reformed sects confess justification by faith alone, but to a large extent their every-day religion is Arminianism pure and simple.

\begin{center}
\textsl{But God says:--}
\end{center}

Gal. 2, 16: ``Knowing that a man is not justified by works of the Law, but by the faith of Jesus Christ.''

Rom. 4, 4. 5: ``Now to him that worketh is the reward not reckoned of grace, but of debt. But to him that worketh not, but believeth on Him that justifieth the ungodly, his faith is counted for righteousness.''

Rom. 11, 6: ``And if by grace, then is it no more of works; otherwise grace is no more grace. But if it be of works, then it is no more grace; otherwise work is no more work.''

Rom. 5, 1: ``Being justified by faith, we have peace with God through our Lord Jesus Christ.''

Rom. 1, 17: ``The just shall live by faith.''

Gal. 3, 26: ``For ye are all the children of God by faith in Christ Jesus.''

\textsc{Note.} -- Though Christ expressly promises all those ``who hunger and thirst after righteousness that they shall be filled'' (Matt. 5, 6), yet the \emph{Catholic Church} would have its members dougt that any one can come to any certainty in the matter. No wonder ``good'' Catholics are willing to pay for indulgences and the reading of masses.

However, the \emph{Methodists} have a touch of the same sickness. Their revival preachers, especially, do not direct the occupants of ``the anxious seat,'' or ``mourners' bench,'' to the precious promises of the Lord in His Word, but to their own feeling. And this feeling is created by ``hard praying,'' shouting, and singing, until the ``subject'' hysterically shouts for joy. He is then declared ``saved.''

\textsl{But God says:--}

John 14, 27: ``Peace I leave with you, My peace I give unto you. Not as the world giveth, give I unto you. Let not your heart be troubled, neither let it be afraid.''

John 20, 29: ``Blessed are they that have not seen, and yet have believed.''

1 John 3, 20: ``For if our heart condemn us, God is greater than our heart, and knoweth all things.''

John 4, 46-53: The nobleman at Capernaum simply clung to the words of Christ: ``Go thy way; thy son liveth.'' ``And himself believed, and his whole house.''

\textsc{Note.} -- \emph{Calvinists} of whatever name (Presbyterian, Congregational, Baptist) do not only teach that Christ died only for the elect, but also that these alone receive the justifying faith, which they can never wholly lose.

This is emphatically disproved by the example of David and Peter. Both certainly fell from grace, and fell deeply, but were again restored.

Paul bemoans the fall of Demas, counsels earnestly the exclusion of the fornicator at Corinth, but recommends him to adoption again after he had become penitent.

\section{Sanctification}\label{sanctification}

\begin{center}
\textsc{Bible Doctrine.}
\end{center}

\begin{enumerate}
\def\labelenumi{\alph{enumi}.}
\tightlist
\item
  Perfect holiness cannot be obtained during our natural life.
\end{enumerate}

Phil. 3, 12: ``Not as though I had already attained, either were already perfect: but I follow after, if I may apprehend that for which also I am apprehended of Christ Jesus.'' (Comp. v. 13.)

1 Thess. 4, 1: ``As ye have received of us how ye ought to walk and to please God, \emph{so ye would abound more and more}.''

Gal. 5, 17: ``For the flesh lusteth against the spirit, and the spirit against the flesh; and these are contrary the one to the other, so that ye cannot do the things that ye would.'' (Heb. 12, 1.)

\begin{center}
\textsl{Lutheran Testimony:--}
\end{center}

``But since in this life believers have not been renewed perfectly or completely (for although their sins are covered by the perfect obedience of Christ, so that they are not imputed to believers for condemnation, and also, through the Holy Ghost, the mortificatino of the old Adam and the renewal in the spirit of their mind in begun), nevertheless the old Adam always clings to them in their nature and all its internal and external powers. Of this the apostle has written: `I know that in me (that is, in my flesh) dwelleth no good thing.' (Rom. 7, 18.) And again: `For that which I do I allow not; for what I would, that do I not; but what I hate, that do I.' Again: `I see another law in my members, warring against the law of my mind, and bringing me into captivity to the law of sin.' Also: `The flesh lusteth against the spirit, and the spirit against the flesh; and these are contrary the one to the other, so that ye cannot do the things that ye would.'

``Therefore, because of these lusts of the flesh, the truly believing, elect, and regenerate children of God require not only the daily instruction and admonition, warning, and threatening, of the Law, but also frequently reproofs, whereby they are roused and follow the Spirit of God. (Ps. 19, 71; 1 Cor. 9, 27; Heb. 12, 8.)'' -- \emph{Form. Conc.}, VI, 7-9.

\begin{center}
\textsc{Error.}
\end{center}

\emph{Catholics} and \emph{Methodists} both teach complete perfection. The former even claim that, by living up to their monastic vows, monks or nuns may attain such perfection that they are able to \emph{sell} to others from their superabundant stock of holiness. This ``stock'' is handled by ``the Church,'' and sold to the needy, according to the Bull \emph{Unigenitus} of Pope Clement VI.

But the \emph{Methodists} do not lag far behind. Wesley has written a special treatise on ``Christian Perfection,'' a standard work among Methodists to this day, in which he maintains that ``a Methodist is one who loves the Lord God with \emph{all} his heart, with \emph{all} his soul, with \emph{all} his mind, with \emph{all} his strength.'' (§ 10.) ``But whom do you mean by `one that is perfect'? We mean one in whom is `the mind which was in Christ,' one `who so walketh as Christ also walked'; a man `that hath clean hands and a pure heart,' or that is `cleansed from all filthiness of flesh and spirit'; one in whom is `no occasion of stumbling,' and who, accordingly, `does not commit sin.'\,'' (§ 15, 4.) In his \emph{Smaller Catechism} Dr.~W. Nast says: ``Can and must a child of God, in this life, be cleansed from all sin? Yes. God's command is: `Ye shall be holy, for I am holy'; and His promise is that, if we confess our sins, He will cleanse us from all unrighteousness, 1 John 1, 9.''

\begin{center}
\textsl{But God says:--}
\end{center}

Jas. 3, 2: ``For in many things we offend all.''

Matt. 6, 12: ``Forgive us our trespasses,'' etc.

Eccl. 7, 20: ``For there is not a just man upon earth, that doeth good and sinneth not.''

Prov. 20, 9: ``Who can say, I have made my heart clean, I am pure from my sin?''

1 John 1, 8: ``IF we say that we have no sin, we deceive ourselves, and the truth is not in us.''

Gal. 5, 17: ``For the flesh lusteth against the spirit, and the spirit against the flesh; and these are contrary the one to the other, so that ye cannot do the things that ye would.''

\begin{center}
\textsc{Bible Doctrine.}
\end{center}

\begin{enumerate}
\def\labelenumi{\alph{enumi}.}
\setcounter{enumi}{1}
\tightlist
\item
  Deeds acceptable to God can be done by believers only.
\end{enumerate}

John 15, 5: ``I am the Vine, ye are the branches. He that abideth in Me, and I in him, the same bringeth forth much fruit; for without Me ye can do nothing.''

Rom. 14, 23: ``Whatsoever is not of faith is sin.''

Heb. 11, 6: ``But without faith it is impossible to please Him; for he that cometh to God must bleieve that He is, and that He is a rewarder of them that diligently seek him.''

Matt. 12, 33: ``For the tree is known by its fruit.''

Gal. 5, 22. 23: ``But the fruit of the Spirit is love, joy, peace, long-suffering, gentleness, goodness, faith, meekness, temperance.''

\begin{center}
\textsl{Lutheran Testimony:--}
\end{center}

``It is God's will, regulation, and command that believers should walk in good works; and that truly good works are not those which every one, with a good intention, himself contrives, or which are done according to human ordinances, but those which God Himself has prescribed and commanded in His Word. Also that truly good works are done, not from our own natural powers, but when by faith the person is reconciled with God, and renewed by the Holy Spirit, or (as Paul says) `created anew in Christ Jesus to good works,' Eph. 2, 10.'' -- \emph{Form. Conc.}, IV, 7.

\begin{center}
\textsc{Error.}
\end{center}

That the semipelagian \emph{Catholic Church}, whose chief hope of salvation rests on human efforts, is the chief offender in this point, goes without saying. At the Council of Trent the following canon was accepted: ``Whoever says that all deeds done before justification - be they of one kind of another - are truly sins, or merit the wrath of God; or who teaches that the more earnestly one strives to make himself qualified for grace, the more he sins; may he be cursed.'' (Sess. 6, 7.) - To this \emph{Chemnitz} in his \emph{Examen}, replies: ``If the deeds in themselves are not sinful, how do they become so? I answer: The heart as GOd's creation is good, but is, on account of sin, nevertheless called evil. In like manner, deeds which not in themselves are evil become befouled and defiled with the unbeliever, because theya re done by people who are not reconciled with God, nor born again, but are full of guilt and sin. For God does not judge according to outward appearance, but according to the state of the heart.'' ``The so-called virtues of unbelievers are not good deeds, but sins, for two reasons: First, as to their \emph{intention}. It is a notable saying of St.~Augustine that not the deed, but the purpose of the deed, decides virtue and vice. For, he says, true virtues are services rendered God in the children of man and the gifts of God in the children of men. Secondly, there is no good fruit which has not grown on the root of love. But love, according to Gal. 5, 22, is the fruit of the Spirit. But this is lacking in them that are not born again. Therefore their good deeds cannot be rated as virtues. A foul tree can bear only foul fruit, that is, sins only. Or according to St.~Augustine: `Whatsoever is not of faith is sin, Rom. 14, 23. This is not to be understood of meats alone (he says) but it is a general truth.'\,'' - Bendixen Ed., 92. 93.

This elucidation is at the present time of special value, since we are so often told of ``moral life,'' ``moral uplift,'' ``moral force,'' etc., where the condition is only common decency - or hidden indecency.

\begin{center}
\textsc{Bible Doctrine.}
\end{center}

\begin{enumerate}
\def\labelenumi{\alph{enumi}.}
\setcounter{enumi}{2}
\tightlist
\item
  That Christ might fulfil all righteousness, He also \emph{preached} the Law as well as the Gospel; but He was \emph{not a new lawgiver}.
\end{enumerate}

John 1, 17: ``For the Law was given by Moses, but grace and truth came by Jesus Christ.''

Luke 4, 18. 19. 21 (Jesus reads Is. 61, 1-3): ``\,`The Spirit of the Lord is upon Me, because He hath annointed Me to preach the Gospel to the poor; He hath sent Me to heal the broken-hearted, to preach deliverance to the captives, and recovering of sight to the blind, to set at liberty them that are bruised, to preach the acceptable year of the Lord.' This day is the scripture fulfilled in your ears.''

\begin{center}
\textsl{Lutheran Testimony:--}
\end{center}

``Some theologians or monks ahve taught us to seek remission of sins, grace, and righteousness through our own works and new forms of worship, which have obscured the office of Christ, and have made out of Christ, not a propitiator and justifier, but only a legislator.'' -- \emph{Apol}., III, 271.

``The adversaries feign that Paul abolishes the Law of Moses, and that Christ succeeds as a new lawgiver in such a way that He does not freely grant the remission of sins, but on account of the works of other laws, if any are now desired\ldots{} Christ has not succeeded Moses in such a way as on account of our works to remit our sins, but so as to set His own merits and His own propitiation on our behalf over against God's wrath, that we may be freely forgiven.'' \emph{Apol}. XIII, 15, 17.

``The adversaries feign that Paul abolishes the Law of Moses, and that Christ succeeds as a new lawgiver in such a way that He does not freely grant the remission of sins, but on account of the work of other laws, if any are now desired\ldots. Christ has not succeeded Moses in such a way as on account of our works to remit sins, but so as to set His own merits and His own propitiation on our behalf over against God's wrath, that we may be freely forgiven.'' -- \emph{Apol.}, XIII, 15, 17.

\begin{center}
\textsc{Error.}
\end{center}

It is the general belief in the \emph{Reformed sects}, that Christ, in the Sermon on the count (Matt. 5-7), has given us ``a new code of ethics,'' and that those who live according to it are sure of heaven. (Why then the atonement?)

The \emph{Roman Catholic Church} not only calls Christ a new lawgiver, but also \emph{curses} those who deny it. (\emph{Council of Trent}, Sess. 6, can. 21.)

The \emph{Methodists} claim that Christ made an ``addition to the Ten Commandments.'' (Cat., No.~3.)

\begin{center}
\textsl{But God says:--}
\end{center}

Gal. 3, 24: ``The Law was our schoolmaster to bring us unto Christ, that we might be justified by faith.'' (4, 4. 5.)

John 5, 45: ``Do not think that I will accuse you to the Father; there is one that accuseth you, even Moses, in whom ye trust.'' (3, 17.)

Gal. 3, 10: ``Christ hath redeemed us from the curse of the Law.'' (Why then ``an addition''?)

\begin{center}
\textsc{Bible Doctrine.}
\end{center}

\begin{enumerate}
\def\labelenumi{\alph{enumi}.}
\setcounter{enumi}{3}
\tightlist
\item
  The rule by which all deeds must be judged is the Moral Law.
\end{enumerate}

Deut. 12, 32: ``What thing soever I command you, observe to do it. Thou shalt not add thereto, nor diminish from it.'' (5, 32.)

Prov. 30, 6: ``Add thou not unto His words, lest He reprove thee, and thou be found a liar.''

2 Tim. 3, 16. 17: ``All Scripture is profitable \ldots{} to make the man of GOd perfect, thoroughly furnished unto all good works.''

\begin{center}
\textsl{Lutheran Testimony:--}
\end{center}

``This doctrine of the Law is needful for believers, in order that they may not depend upon their own holiness and devotion, and under the pretext of the Spirit of God establish a self-chosen form of divine worship, without God's Word and command, as it is written: `Ye shall not do \ldots{} every man whatsoever is right in his own eyes,' etc., but `observe and hear all these words which I command thee.' `Thou shalt not add thereunto, nor diminish therefrom.' (Deut. 12, 8. 28. 32.)'' -- \emph{Form Conc.}, VI, 20.

\begin{center}
\textsc{Error.}
\end{center}

Besides the \emph{Catholics}, with their doctrine of evangelical counsels and consequent superabundant works, the worst errorists in this doctrine are the Romanizing, legalistic, and pietistic \emph{Methodists}. Who has not heard the claim that traveling on Sunday, indulging in innocent recreation, smoking, as well as every use of liquor, are reprehensible, sinful acts? (See \emph{Doctrines and Discipline}, chaps. 2. 3.)

Nearly all Reformed sects are prone to legalism, and try to serve GOd through the commandments of men. (Matt. 15, 9.)

\begin{center}
\textsl{But God says:--}
\end{center}

Col. 2, 16-23: ``Let no man, therefore, judge you in meat, or in drink, or in respect of an holy-day, or of the new moon, or of the Sabbath days; which are a shadow of things to come; but the body is of Christ. Let no man beguile you of your reward, in a voluntary humility and worshipping of angels, intruding into those things which he hath not seen, vainly puffed up by his fleshly mind, and not holding the Head, from which all the body by joints and bands have nourishment ministered, and knit together, increaseth with the increase of GOd. Wherefore, if ye be dead with Christ from the rudiments of the world, why, as though living in the world, are ye subject to ordinances (touch not; taste not; handle not; which are to perish with the using), after the commandments and doctrines of men? Which things have indeed a show of wisdom in will-worship, and humility, and neglecting of the body, not in any honor to the satisfying of the flesh.''

1 Tim. 4, 1-5: ``Now the Spirit speaketh expressly that in the latter times some shall depart from the fiaht, giving heed to seducing spirits and doctrines of devils; speaking lies in hypocrisy; having their consciences seared with a hot iron; forbidding to marry, and commanding to abstain from meats, which God hath created to be received with thanksgiving of them which believe and know the truth. For every creature of God is good, and nothing to be refused if it be received with thanksgiving; for it is sanctified by the Word of God and prayer.''

\begin{center}
\textsc{Bible Doctrine.}
\end{center}

\begin{enumerate}
\def\labelenumi{\alph{enumi}.}
\setcounter{enumi}{4}
\tightlist
\item
  No one becomes a Christian through good deeds, but no one can ignore the diving Law and remain a Christian.
\end{enumerate}

Titus 3, 5: ``Not by works of righteousness which we have done, but according to His mercy He saved us.''

Rom. 3, 20. 28: ``Therefore, by the deeds of the Law there shall no flesh be justified in His sight. Therefore we conclude that a man is justified by faith, without the deeds of the Law.''

Matt. 5, 16: ``Let your light so shine before men that they may see your good works, and glorify your Father which is in heaven.''

Eph. 2, 10: ``For we are His workmanship, created in Christ Jesus unto good works, which God hath before ordained that we should walk in them.''

\begin{center}
\textsl{Lutheran Testimony:--}
\end{center}

``We teach that it is necessary to do good works; not that we may trust that we deserve grace by them, but because it is the will of God that we should do them. By faith alone is apprehended remission of sins and grace. And because the Holy Spirit is received by faith, our hearts are now renewed, and so put on new affections, so that they are able to bring forth good works.'' -- \emph{Augsb. Conf.}, XX, 27-29.

The reason for doing good works \emph{Chemnitz} explains as follows: ``Such are not urged or done as though it were necessary to augment or replace the righteousness or salvation of Christ, which we accept by faith. For righteousness and salvation must be present before a single good deed can be done. The tree must be good before it can bring forth good fruit. (Matt. 12, 33.) Thus, necessarily, good fruit will follow where there is true faith, righteousness, and salvation. Where, therefore, no good fruit follows, there we have a sure sign that faith is not of the right kind, and that righteousness and salvation are not present, but have been lost. (1 Tim. 5, 8; 2 Pet. 1, 9; 1 John 3, 10. 24.)

``\emph{Urban Rhegius} summarizes the whole thus: 'We should do good works,--

\begin{enumerate}
\def\labelenumi{\arabic{enumi}.}
\item
  Because it is a debt of obedience which we owe, and are commanded to do, to God; also a thanksgiving for His goodness and an offering pleasing to God.
\item
  That the heavenly Father may be praised by them.
\item
  That our faith through them may be strengthened and increased.
\item
  That our fellow-men through them may be incited to do good, and receive help in distress.
\item
  That our fiath through them may be shown and proved, our calling made sure, and our faith proved not to be false or dead.
\item
  Because our good deeds, though not meriting forgiveness of sin and salvation, have promise both of temporal and eternal reward. (1 Tim. 4, 8.)'\,'' -- \emph{Chemn., Enchiridion}.
\end{enumerate}

\begin{center}
\textsc{Error.}
\end{center}

\begin{enumerate}
\def\labelenumi{\arabic{enumi}.}
\item
  All antichristian sects, such as \emph{Socinians, Unitarians, Mormons, Seventh-day Adventists}. Denying the atonement of Christ, they must have ``something religious'' to fall back on. But why even that? If either all are saved at last, or the so-called wicked utterly destroyed, who would really battle ``to be good''? Their dictum of ``good deeds necessary to salvation'' is, therefore, only a proof of the truth that ``conscience makes cowards of us all.''
\item
  The \emph{Catholic Church}, from from \emph{the average sectarian}, wittingly or unwittingly, has absorbed that pernicious mixture of Law and Gospel which characterizes the Roman system.
\end{enumerate}

The \emph{Decrees of Trent} declare (Sess. 6, c.~16): ``As a constant power flows from Christ, the Head, on the justified, who are His members, as from the vine to its branches, a power which precedes their good works, accompanies the same, and follows them, -- a power without which they can be in no wise agreeable to God and meritorious, so we are bound to believe that the justified are enabled, through works performed in God, to satisfy the divine Law, according to the conditions of this present life, and to \emph{merit eternal life}, when they depart in a state of grace.'' -- \emph{J. A. Moehler's Symbolism}, 158.

\begin{center}
\textsl{But God says:--}
\end{center}

Heb. 2, 4: ``The just shall live \emph{by his faith}.''

Gal. 3, 10. 11: ``For as many as are of the works of the Law are under the curse. But that no man is justified by the Law in the sight of God it is evident; for, The just shall live by faith.''

Rom. 7, 18: ``For I know that in me, that is, in my flesh, dwelleth no good thing; for to will is present with me, but how to perform that which is good I find not.''

Is. 64, 6: ``But we are all as an unclean thing, and all our righteousnesses are as filthy rags.''

Ps. 143, 2: ``Enter not into judgment with Thy servant; for in Thy sight shall no man living be justified.'' (Luke 17, 10.)

1 Cor. 4, 15: ``In Christ Jesus I have begotten you through the Gospel.''

Jas. 1, 18. 21: ``Of His own will begot He us with the Word of Truth, that we should be a kind of first-fruits of His creatures. Wherefore \ldots{} receive with meekness the engrafted Word, which is able to save your souls.''

Jer. 23, 29: ``Is not My Word like as a fire? saith the Lord, and like a hammer that breaketh the rock in pieces?''

1 Cor. 1, 18: ``For the preaching of the Cross is to them that perish foolishness, but unto us which are saved \emph{it is the power of God}.''

\section{The Means of Grace}\label{the-means-of-grace}

\begin{center}
\textbf{A.  What Are the Means of Grace?}

\textsc{Bible Doctrine.}
\end{center}

The only means through which God has promised to bless mankind are the so-called means of grace, His holy Word and the Sacraments.

John 6, 63: ``The words that I speak unto you, they are Spirit, and they are life.''

Heb. 4, 12: ``The Word of God is quick, and powerful, and sharper than any two-edged sword, piercing even to the dividing asunder of soul and spirit, and of the joints and marrow, and is a discerner of the thoughts and intents of the heart.''

Rom. 1, 16: ``The Gospel of Christ is a power of God unto salvation to every one that believeth.''

\begin{center}
\textsl{Lutheran Testimony:--}
\end{center}

``For the obtaining of this faith the ministry of teachign the Gospel and administering the Sacraments was instituted. For by the Word and Sacraments, as by instruments, the Holy Spirit is given, who worketh faith, where and when it pleaseth God, in those that hear the Gospel.'' -- \emph{Augsb. Conf.}, V, 1. 2.

``In those things which concern the spoken outward Word, we must firmly hold that GOd grants His Spirit or grace to no one except through or with the preceeding outward Word. Thereby we are protected against enthusiasts, \emph{i. e.}, spirits who boast that they have the Spirit without and before the Word.'' \emph{Smalc. Art.}, III, 10, 33.

\begin{center}
\textsc{Error.}
\end{center}

\begin{enumerate}
\def\labelenumi{\arabic{enumi}.}
\item
  The whole army of \emph{Reformed} teachers from \emph{Zwingli} to \emph{``Billy'' Sunday}, denies, in word and in deed, that the Scripture and the Sacraments are vehicles through which God bestows His grace on mankind. \emph{Calvin} calls the Word ``a sound only, that has effect, not because of its being spoken (or used), but because it is believed'' (\emph{Inst.}, IV, 14, 4). \emph{Zwingli} says in his \emph{Reckoning of Faith}: ``I believe, yea I know, that all the Sacraments are so far from conferring grace that they do not even convey or distribute it\ldots. Moreover, a channel or vehicle is not necessary to the Spirit; for He Himself is the virtue and energy, whereby all things are borne, and has no need of being borne.'' (§ 7.)
\item
  According to \emph{Methodism} ``The means of grace are either instituted or prudential.'' The instituded are ``prayer, searching the Scriptures, the Lord's Supper, fasting, and Christian conference.'' (\emph{Doctr. and Discipl.}, §§ 118-122.) According to this explanation everything enjoined in Scripture is a means of grace, \emph{e. g.}, the Ten Commandments with their deeper, spiritual explanation, as found in ther Sermon on the Mount, Matt. 5-7! Of the Sacraments the \emph{Methodists} say: ``A Sacrament is an outward and visible \emph{sign} of an inward and spiritual grace.'' -- \emph{Cat,}, No.~3.
\item
  The \emph{Episcopal Church} teaches that ``Sacraments ordained of Christ be not only badges or tokens of Christian men's profession, but rather they be certain sure witnesses, and effectual \emph{signs of grace}, and GOd's good will towards us, by which He doth work invisibly in us, and doth not only quicken, but also strengthen and confirm, our faith in Him.'' (\emph{Art. of Rel.}, XXV.) The \emph{Reformed Episocopal Church} expressly states that it wants the word ``Sacrament to mean only a symbol or sign instituted by God.'' (Art. 25.)
\item
  \emph{Presbyterians} and \emph{Congregationalists} confess: ``Sacraments are holy \emph{signs} and seals of the covenant of grace, immediately instituted by God, to represent Christ and His benefits, and to confirm our interest in Him, as also to put a visible difference between those that belong unto the Church and the rest of the world, and solemnly to engage them to the service of God in Christ, according to His Word.'' -- Chap. XXVII, 1. \emph{Savoy Decl.}, chap.~XXVIII.
\end{enumerate}

\textsc{Note. --} The Lutheran Church also calls the Sacraments ``signs,'' but \emph{not only} that. Primarily they are \emph{means} through which God gives His grace. The Word of God as a whole is such an effective means; why, then, should not the words of institution in the Sacraments be so?

``Christ has instituted the Sacraments for thereby to strengthen and preserve our faith. Our hearts have difficulty in holding fast to the Word alone. So, when it is proclaimed for all in general, our conscience vexes us with questions, whether God also means me, or, if I also, I personally, may accept it and console myself with it. Therefore Christ, being rich in mercy, instituted such outward, visible Sacraments, in and through which, as through visible, public evidences, He wishes to deal with us, that we may have a sure seal and bond that the primises of the Gospel concern eacn and every one personally, and that He may effectively apply, make sure, and seal such truth in the hearts of all those who use the Sacraments in true faith. And there is no doubting that the Son of God, through these means, works mightily to strengthen and preserve the faith in the believers. Consequently, the Sacramentarians are rightly rejected, who hold that the Sacraments are only outward signs that only indicate or signify something.'' -- \emph{Enchiridion}, Chemnitz.

Since the twelfth century, the Catholic Church has held that there are seven Sacraments: Baptism, Confirmation, the Lord's Supper, Penance, Orders (Priesthood, etc.), Marriage, and Extreme Unction. In their consecration of bells, lights, water, etc., formulas are used ``which presuppose a magical effect.''

\begin{center}
\textsl{But God says:--}
\end{center}

Is. 55, 10. 11: ``For as the rain cometh down, and the snow, from heaven, and returneth not thither, but watereth the earth, and maketh it bring forth and bud, that it may give seed to the sower and bread to the eater, so shall My Word be that goeth forth out of My mouth: it shall not return unto Me void, but \emph{it shall accomplish} that which I please, and it shall prosper in the thing whereto I sent it.'' (Luke 11, 28.)

John 6, 68: ``Lord, to whom shall we go? Thou hast the \emph{words of eternal life}.''

1 Pet. 1, 23: ``being born again, not of corruptible seed, but of incorruptible, \emph{by the Word of God}, which liveth and abideth forever.'' (Rom. 10, 6-8.)

\begin{center}
\textbf{B.  Efficacy of the Means of Grace.}

\textsc{Bible Doctrine.}
\end{center}

The means of grace are effective regardless of the attitude of the administrator towards God.

Matt. 23, 2. 3: ``The scribes and the PHarisees sit in Moses' seat. All, therefore, whatsoever they bid you observe, that observe and do; but do not ye after their words; for they say, and do not.'' (Rom. 2, 21. 23.)

\begin{center}
\textsl{Lutheran Testimony.}
\end{center}

``Though the Church be properly the congregation of saints and true believers, yet, seeing that in this life many hypocrites and evil persons are mingled with it, it is lawful to use the Sacraments administered by evil men, according to the voice of Christ: `The scribes and the Pharisees sit in Moses' seat,' and the words following. And the Sacraments of the Word are effectual by reason of the institution and commandment of Christ, though they be delivered by evil men. They condemn the Donatists and such like, who denied that it was lawful to use the ministry of evil men in the Church and held that the ministry of evil men is useless and without effect.'' -- \emph{Aubsb. Conf.}, VIII

\begin{center}
\textsc{Error.}
\end{center}

The \emph{Donatists}, like the Pharisees, are an ever-present race. And they are not confined to any particular denomination. Churches of the \emph{Methodistic} order suffer most, but all have their share, even the Lutheran Church. The usual charge is that the poor state of the Church, as well as society, must be charged to the account of the unconverted clergy; for ``if ministers spake from the heart, their message would go to the heart.''

But history and experience tell a different story, as the following shows.

\begin{enumerate}
\def\labelenumi{\arabic{enumi}.}
\item
  Ministers are only deliverers of a message. The message, not the messenger, is the instrument of power.
\item
  God alone judges the heart rightly. The error committed by the prophet Elijah in passing judgment on his fellow-man should make us all pause.
\item
  Noah, the mighty ``preacher of righteousness'' (2 Pet. 2, 5.), was certainly zealous in the cause of God, but he did not convert even the builders of the Ark.
\item
  The prophets of old were burning in their zeal for the conversion of sinners, but were stoned for their labors.
\item
  John the Baptist labored largely in vain. And the fate of Christ's apostles is familiar to all. And yet, no one since their day has had the Spirit in such a measure as they.
\end{enumerate}

It is true: ``There is no more contemptible creature on earth than an unconverted preacher.'' -- But the efficacy of the means of grace does not depend upon the converted condition of the clergyman.

\begin{center}
\textsl{But God says:--}
\end{center}

Phil. 1, 15-18: ``Some, indeed, preach Christ even of envy and strife, and some also of good will. The one preaches Christ of contention, not sincerely, supposing to add afflictions to my bonds; but the other of love, knowing that I am set for the defense of the Gospel. What then? Notwithstanding, ever way, whether in pretense or in truth, Christ is preached; and I therein do rejoice, yes, and will rejoice.''

\textsc{Note. 1. --} The doctrine concerning the means of grace is not only clearly taught in the Scriptures, but is also of \emph{vital importance} to the penitent sinner. He hears much about the ``testimony of the Spirit,'' the ``baptism of fire,'' and that the ``Spirit'' is the cause of every religious movement, from Zwinglianism to Dowieism. But as all these ``isms'' in many instances bear witness \emph{against} one another, the ``Spirit' is either unreliable, or it is not the \emph{right} Spirit. Hence a \emph{rule} of faith and a \emph{means} through which God \emph{gives} faith, \emph{according} to this rule, is of the highest importance. This we have in the Gospel.

\textsc{Note. 2. --} The reading of the Bible among the lay people of the Protestant Church is not only permitted, but urgently recommended. in the Catholic Church, however, lay people must have a ``permit'' before they can get a Bible, and then only the Catholic version, or edition, which has several incorrectly translated passages (favorable to Rome, and hence never righted) and numerous explanatory notes (also favorable to Rome). Catholic laymen are never allowed to read or own a Protestant Bible. Hence we to this day hear of Bibles being burned by Catholic priests.

\begin{center}
\textbf{Baptism.}

\textsl{1.  What Is Baptism?}

\textsc{Bible Doctrine.}
\end{center}

Baptism is a sacramental institution of God, through which He effects regeneration and grants forgiveness of sin.

Matt. 28, 19. 20: ``Go ye, therefore, and \emph{make disciples of all the nations}, baptizing them in the name of the Father and of the Son and of the Holy Ghost; teaching them to observe all things whatsoever I have commanded you. And lo, I am with you alway, even unto the end of the world.''

Gal. 3, 26: ``For ye are all the children of God by faith in Christ Jesus. For as many of you as have been baptized into Christ have \emph{put on Christ}.''

Acts 2, 38: ``Repent ye, and be baptized, every one of you, in the name of Jesus Christ \emph{unto the remission of your sins}, and ye shall receive the gift of the Holy Ghost.''

John 3, 5: ``Verily, verily, I say unto thee, Except a man be born of water and of the Spirit, he cannot \emph{enter into the kingdom of God}.''

Titus 3, 5-7: ``Not by works of righteousness which we have done, but according to His mercy \emph{He saved us, by the washing of regeneration} and renewing of the Holy Ghost, which He shed on us abundantly through Jesus Christ, our Savior, that, being justified by His grace, we should be made heirs according to the hope of eternal life.''

Eph. 5, 25. 26: ``Christ loved the Church, and gave Himself for it, that He might sanctify and \emph{cleanse it} with the washing of water by the word.''

1 Pet. 3, 20. 21: ``Noah prepared the ark, wherein few, that is, eight souls, \emph{were saved by water; the like figure} whereunto even \emph{Baptism doth also now save us} (not the putting away of the filth of the flesh, but the answer of a good conscience toward God), by the resurrection of Jesus Christ.''

Col. 2, 11. 12: ``In whom also ye are circumcised with the circumcision made without hands, in putting off the body of the sins of the flesh by the circumcision of Christ; buried with him in Baptism, \emph{wherein also ye are risen with Him through the faith of the operation of God}; who hath raised Him from the dead.'' (Also v. 13, The argument here is: Baptism has taken the place of circumcision, hence the latter is not necessary. But as circumcision was administered to children, so Baptism also must be. And as circumcision converred ``burying of sins,'' so also Baptism.)

Acts 22, 16. On reciting at Jerusalem the story of his conversion, Paul also relates how the prophet Ananias, at God's command, came to him, saying among other things: ``And now, why tarriest thou? Arise, and be baptized, and \emph{wash away thy sins}, calling on the name of the Lord.'' NB. When Calvin interprets this passage by saying (\emph{Inst.}, IV, 15, 15): ``All, then, that Ananias meant to say was: Be baptized, Paul, that you may be assured that your sins are forgiven you,'' this is a fair example of Reformed perversion of Scripture.

\begin{center}
\textsl{Lutheran Testimony:--}
\end{center}

``In these words {[}of institution{]} we must notice, in the first place, that here stands God's commandment and institution that we shall not doubt that Baptism is difine, and not devised and invented by men. For as truly as I can say no man has spun the Ten Commandments, the Creed, and the Lord's Prayer out of his head, but they are revealed and given by God Himself, so also I can boast that Baptism is no human trifle, but that it is instituted by God Himself, and that it is most solemnly and rididly commanded that we must permit ourselves to be baptized, or we cannot be saved.'' -- \emph{Large Cat.}, IV, 6.

``Since we know now what Baptism is, and how it is to be administered, we must, in the second place, also learn why and for what purpose it is instituted; that is, what it avails, gives, and produces. And this also we cannot discern better than from the words of Christ above quoted: `He that believeth and is baptized shall be saved.' Therefore we state it most simply thus, that the power, work, profit, fruit, and end of Baptism is this, \emph{viz.}, to save. For no one is baptized in order that he may become a prince, but, as the words declare, that he be saved. But to be saved, we know, is nothing else than to be delivered from sin, death, and the devil, and to enter into the kingdom of Christ, and to live with Him forever.'' -- \emph{Large Cat.}, IV, 23 f.

\begin{center}
\textsc{Error.}
\end{center}

With the Word of God in this doctrine the Lutheran Church stands alone. \emph{All the others} follow blind reason, and \emph{declare that Baptism does not effect regeneration, but is only a sign of regeneration, or the new birth}.

\emph{Episcopalians}, \emph{Presbyterians}, \emph{Baptists}, \emph{Congregationalists}, and \emph{Methodists} differ in their confessions about Baptism in language only. The Presbyterian may be taken as a model: ``Baptism is a Sacrament of the New Testament, ordained by Jesus Christ, not only for the solemn admission of the party baptized into the visible Church, but also to be unto him a \emph{sign and a seal} of the covenant of grace, of his ingrafting into Christ, of regeneration, of remission of sins, and of his giving up unto God, through Jesus Christ, to walk in newness of life: which Sacrament is, by Christ's own appointment, to be continued in His Church until the end of the world.'' (XXVIII.)

\begin{center}
\textsl{But God says:--}
\end{center}

Rom. 6, 3. 4: ``So many of us as were baptized into Jesus Christ were baptized into His death. Therefore we are buried with Him by Baptism into death.''

1 Cor. 12, 13: ``For in one Spirit were all baptized into one body'' (\emph{i. e.}, Christ, v. 12).

\textsc{Note. --} The Nicene Creed, § 9, reads: ``I acknowledge one Baptism for the \emph{remission of sins}.''

\begin{center}
\textsl{2.  For Whom Is Baptism Instituted?}

\textsc{Bible Doctrine.}
\end{center}

The Sacrament of Baptism is instituted not only for adults, but also for the children.

Matt. 28, 19: ``Go ye, therefore, and teach {[}make disciples of{]} \emph{all nations}, baptizing them,'' etc.

John 3, 5: ``Verily, verily, I say unto thee, Except a man be born of water and the Spirit, he cannot enter into the Kingdom of God.''

Acts 16, 14. 15: ``Lydia, a seller of purple, of the city of Thyatira,\ldots was baptized, \emph{and her household}.''

Acts 16, 32. 33: The jailer at Philippi, when converted, was baptized, ``\emph{and all his}.''

\begin{center}
\textsl{Lutheran Testimony:--}
\end{center}

``Of Baptism they teach that it is necessary to salvation, and that by Baptism the grace of God is offered, and that children are to be baptized, who, by Baptism being offered to God, are received into God's favor. They condemn the Anabaptists, who allow not the baptism of children, and affirm that children are saved without Baptism.'' -- \emph{Augsb. Conf.}, Art. IX.

``It is manifest that God approves of the baptism of little children. Therefore the Anabaptists, who condemn the baptism of little Children, believe wickedly. That God, however, approves of the baptism of little children is shown by this, \emph{viz.}, that God gives the Holy Ghost to those thus baptized. For if this baptism would be in vain, the Holy Ghost would be given to none, none would be saved, and finally there would be no Church. This reason, even taken alone, can sufficiently establish good and godly minds against the godless and fanatical opinions of the Anabaptists.'' -- \emph{Apol. A. C.}, IX, 53.

\begin{center}
\textsc{Error.}
\end{center}

The chief offenders against the truth in the doctrine of Baptism are the so-called \emph{Baptists}. In their confession of 1688 they declare that \emph{only} such are to be baptized as can be truly penitent before God, confess their faith in Christ Jesus, and show their obedience towards Him. (XZXIX, 2.) This, they argue, children cannot do, hence they are not to be baptized.

\textsc{Note. --} The doctrine of Baptism has been variously assailed, from the days of the early Church down to the present time. That Baptism was unnecessary, especially for children, and that only grown persons could have any benefit from it, were early heresies. Many delayed being baptized until they were on their deathbed, while others repeated the rite several times. Especially the practice of infant baptism has been assailed by various sects, and is condemned as unscrpitural and unnecessary by some to the present day.

\begin{center}
\textsc{Objections to Infant Baptism Answered.}
\end{center}

\emph{First Objection}. -- When Christ commanded His apostles to '' make disciples of all nations,'' He did not take children into account.

\emph{Answer}. -- A nation is composed of old and young. Christ does not especially mention children in the command to baptize, because the apostles knew that such were included, even as they were included in the command to circumcise under the Old Testament dispensation. The Savior emphasizes only two things, \emph{viz}.: Baptism is to take the place of circumcision, and instead of their working among \emph{one} nation they should go out among \emph{all} nations

\emph{Second Objection}. -- In connection with Baptism Christ also speaks of faith. But to believe is to perform a conscious act, something which infants obviously cannot do.

\emph{Answer}. -- This is not only human reasoning, but faulty reasoning at that. To believe is to perform a conscious act, \emph{but not always}. A person who with fear and trepidation presents a petition to his dread monarch has often no hope at all. His faith is only an ardent desire. The publican with his petition, Luke 18, 13, the father of the son with a dumb spirit, Mark 9, 24, are examples of such unconscious believers. And to such Christ ministered, according to the promise of old: ``But to this man will I look, even to him that is poor and of a contrite spirit, and trembleth at My Word.'' (Is. 66, 2.) Again: Without faith it is impossible to please God (Heb. 11, 6), and he that believeth not shall be damned (Mark 16, 16). But if faith is always a conscious act, how could a sleeping person be a Christian? Or how, then, could a person dying while asleep, or during unconsciousness, escape condemnation?

Christ solves the seeming mystery best of all. He says of certain small children that they \emph{believe on Him} (Matt. 18, 6), and that, if we would enter the kingdom of heaven, we must become as they (v. 3). At His triumphant entry into Jerusalem He was hailed as King, not only by the adult population, but also by ``babes and sucklings'' (Matt. 21, 15. 16); and He reproved those who tried to stop them.

\emph{Third Objection}. -- In conjunction with Baptism Christ also enjoins teaching. And the apostles seem to have followed this practise. In Acts 2, 41 it is related how only those were baptized who \emph{received the teachings} of the apostle. ``Babes and sucklings'' cannot thus be instructed.

\emph{Answer}. -- The practise of the apostles is neither here nor elsewhere fully described. But it is preposterous to conclude that among the families baptized on Pentecost there should have been no children. Likewise, when it is related in Acts 16 that ``Lydia and her household were baptized,'' also the keeper of the prison at Philippi ``and all his,'' we rightly conclude that children were included. And certainly the contrary cannot be established. (See also Acts 18, 8; 1 Cor. 1, 16.)

The older translation of the Bible is much to blame for the view that baptism must follow teaching only. If the reading, ``Go ye and teach all nations,'' in Matt. 28, 19, had been rendered, ``Go ye and make disciples of all nations, baptizing them,'' etc., as the original Greek has it, many a sincere truth-seeker would not have fallen a prey to designing proselyters.

\emph{Fourth Objection}. -- There is no way in which we can communicate with, or exert influence on, the soul of a child; hence anything that we may do is of no avail.

\emph{Answer}. -- A bud is not the flower in full bloom, and yet the whole flower is in the bud. The suckling is not very old before, like a bud, it unfolds its latent powers. Besides, it is not man who influences the soul of a child, -- or an adult's, for that matter, -- but God, through His means of grace.

Of Jeremiah God says: ``Before I formed thee, I knew thee; before thou camest forth, I sanctified thee.'' (Jer. 1, 5.) And of John the Baptist: ``He shall be filled with the Holy Ghost, even from his mother's womb.'' (Luke 1, 15.)

The point here is not that the Lord gave these two children the Holy Ghost, but that they received it \emph{earlier than others}, who received it on the eighth day after their birth, at \emph{circumcision}. For as Abraham received the sign of circumcision as a seal of righteousness of faith (Rom. 4, 11), so this same sign became the same seal unto the \emph{children} who were circumcised. In like manner GOd now, through Baptism, als influences the child's spiritual life, and gives it the Holy Ghost.

\emph{Fifth Objection}. -- The children at birth have no need of being influenced by God or man; for according to Rom. 5, 18, ``as by the offense of one judgement came upon all men to condemnation, even so by the righteousness of One the free gift came upon all men unto justification of life''; and this can apply fully only when innocent childhood is taken into account.

\emph{Answer}. -- Only one child born into this world was born without sin. All others are conceived in sin and shapen in iniquity (Ps. 51, 5); all have sinned, and come short of the glory of God (Rom. 3, 23); and except a man be born again, born of water and of the Spirit, he cannot enter into the kingdom of God (John 3, 3. 5).

As to Rom. 5, 18, the apostle explains in the preceeding verse how he wishes the declaration, ``Free gift came upon all men,'' understood. He says: ``For if by the trespasses of the one, death reigned through the one, much more shall they that \emph{receive} the abundance of grace and of the gift of righteousness reign in life through the One, even Jesus Christ.'' There is abundance of grace for all, but only they profit by it who \emph{``receive''} it, \emph{i. e.}, believe it.

The Jews believed that they, as a matter of course, would surely be saved because they were the descendants of Abraham; but the apostle says explicitly: ``They which are of faith, the same are the children of Abraham.'' (Gal. 3, 7; Rom. 9, 8.)

\emph{Sixth Objection.} -- Baptism of the Spirit is the only regenerating baptism. Water-baptism is of no avail.

\emph{Answer}. -- To argue thus is to put asunder what God has joined together. God is not bound to certain means, it is true, but we are: and the few examples of God's extraordinary workings are only exceptions that attest the rule. The apostles on Pentecost promised the remission of sins and the gift of the Holy Ghost only to those who would be baptized. (Acts 2, 38.) God could, undoubtedly, kindle faith in man's heart without means, but He has bound us to the use of His Word, and promised salvation only to those who use it and accept it. (Rom. 10, 8-18.)

To call Christian Baptism ``water-baptism'' is to talk disrespectully of a divine ordinance.

Naaman the Syrian found fault with the treatment which the prophet Elisha told him to use for his leprosy; but when he listened to counsel and used it, he became clean. (2 Kings 5.) When God has ordained the Sacrament of Baptism, and through it gives salvation and the Holy Ghost (Titus 3, 5), it behooves us to praise and give thanks, instead of finding fault. To say with Calvin that ``Spirit and water (John 3, 5) is the same as Spirit which is water'' is doing violence to the Scriptures. (See his \emph{Institutio Rel.}, IV, 16, 25.)

\emph{Seventh Objection}. -- The baptism of infants is only a relic of popery, and not sanctioned by apostolic practise.

\emph{Answer}. -- The beginning of popery has no exact date. Popery is a growth whose roots reach back into the time of the apostles themselves, who often speak of antichristian tendencies. But historians of merit generally say that ``the worshiping of the Beast and his image'' (Rev.~14, 9) became unmistakable in the sixth century. But Baptism was not one of its signs, as the following excerpts clearly show.

\emph{St.~Augustine}, bishop of Hippo, † 430, says: ``The whole church practises infant baptism. It was not instituted by councils, but was always practised.''

At the instance of Bishop \emph{St.~Cyprian}, † 258, the \emph{Council of Carthage} (254) decreed that ``no one ought to be hindered from Baptism and from the grace of God, who is merciful and kind and loving to all; which, since it is to be observed and maintained in respect of all, we think it is to be even more observed in respect of infants and newly born persons, who on this very account deserve more from our help and from the divine mercy, that immediately ,on the very beginning of their birth, lamenting and weeping, they do nothing else than entreat.'' -- \emph{Ante-Nicene Fathers}, V, 354.

\emph{Origen} of Alexandria, † 254, not only testifies that it was according to ancient usage of the Church to ``baptize even infants\ldots for the remission of sins,'' but that such custom had come down ``\emph{from the apostles themselves}.''

\emph{Tertullian}, † 245, who on account of his heresies really is no church-father, fought infant-baptism, \emph{not} on account of its unscrpituralness, but because of the danger of falling away from the baptismal vow. He wanted it done as late in life as possible. -- \emph{Ante-Nicene Fathers}, III, 661 f.

In the so-called \emph{Constitutions of the Holy Apostles} (erroneously ascribed to Clement of Rome, a disciple of the Apostle Peter), a collection of church rules of not later than the fourth century, we read: ``Do you also baptize your infants, and bring them up in the nurture and admonition of God. For He says: `Suffer the little children to come unto Me, and forbid them not.'\,'' -- \emph{Ante-Nicene Fathers}, VII, 457.

\emph{Irenaeus}, bishop of Lyons, who died as a martyr 202, in \emph{Against Heretics}, writes: ``He came to save all through means of Himself -- all, I say, who through Him are born again to God -- \emph{infants and children}, and boys, and youths, and old men.'' (\emph{Ante-Nicene Fathers}, I, 391.) What he means by ``born again to God'' he tells us further on: ``And again, giving to the disciples \emph{the power or regeneration} into God, He said to them, `Go and teach all nations, baptizing them in the name of the Father and of the Son and of the Holy Ghost.'\,'' (444.)

\emph{Justin Martyr}, † 179, in his \emph{First Apology} (15), speaks of ``many, both men and women, who have been Christ's disciples from childhood.'' This he explains more fully in chapter 61, on ``Christian Baptism'': ``Then they are brought by us where there is water, and are regenerated in the same manner in which we ourselves were regenerated.'' -- \emph{Ante-Nicene Fathers}, I, 167. 183.

During the Pelagian controversy the doctrine of infant baptism also came up for consideration. After -- or at -- the Council of Carthage, 416, \emph{sixty-eight} bishops wrote to Bishop Innocent of Rome, and, among other things, pointed out to him that the error of the Pelagians really meant ``that infants need not be baptized at all, that they were perfectly innocent and needed no redemption.''

\emph{Eighth Objection}. -- Baptism is valid only when done by immersion.

\emph{Answer}. -- This argument from the first illiterate adversaries of infant baptism is, wonderful to relate, still in vogue, especially among Baptists. At first the Greek word, which has been Anglicized into ``baptize,'' was held to mean ``immerse,'' and nothing else. Scholars among them must now, of course, acknowledge that such is not the case (as can readily be seen by any one who has but a smattering of Greek, in referring to Matt. 7, 4. 8; 3, 11; Acts 1, 5; 2, 17, etc., etc.), but they cling to the theory just the same.

Baptism by immersion was undeniably practised in the first few centuries after the apostles, but not by any means exclusively. \emph{The Teachings of the Twelve Apostles}, a document dating from the second century of the Christian era, has this to say on the mode of baptizing: ``And concerning Baptism, thus baptize ye: Having first said all these things, baptize into the name of the Father and of the Son and of the Holy Spirit, in living water. But if thou have not living water, baptize into other water; and if thou canst not in cold, in warm. But if thou have not either, \emph{pour out water} thrice upon the head into the name of the Father and Son and Holy Ghost.'' -- \emph{Ante-Nicene Fathers}, VII, 379.

\emph{Ninth Objection}. -- The repetition of Baptism is not a sin, but a blessing.

\emph{Answer}. -- Since there is no warrant in the Scriptures for the Catholic doctrine of an ``indelible character'' conferred by Baptism, some fanatics have gone to the other extreme and taught that Baptism ought to be repeated now and then, even as the other Sacrament, the Lord's Supper, is repeated.

The Lord's Supper is to be repeated because \emph{Christ expressly comands it}. In the institution of Baptism no such command is added. IN the whole Bible no instance of repetition is recorded. The covenant made in Baptism may be broken by man, but not by the Lord. He who repents and returns to GOd will find the baptismal covenant still in force.

\begin{center}
\textbf{D.  The Office of the Keys, or Absolution.}

\textsc{Bible Doctrine.}
\end{center}

\emph{Definition}. -- ``Through the office of the ministry God has given to the Church the command and power to pass judgment on the unbelieving and impenitent sinner, and, for Christ's sake, to absolve the believing and penitent.'' -- \emph{Chemnitz}, \emph{Enchir. and Ex.}

Matt. 16, 19: ``Whatsoever thou shalt bind on earth shall be bound in heaven, and whatsoever thou shalt lose on earth shall be loosed in heaven.''

John 20, 23 (to all the disciples): ``Whosoever sins ye remit, they are remitted unto them, and whosoever sins ye retain, they are retained.''

\begin{center}
\textsl{Lutheran Testimony:--}
\end{center}

``Men are taught that they should highly regard absolution, inasmuch as it is God's voice, and pronounced by God's command.'' -- \emph{Augsb. Conf.}, Art. 25, 3.

``Now, their judgment is this: that the power of the keys, or the power of the bishops, by the rule of the Gospel, is a power, or commandment, from God of preaching the Gospel, of remitting or retaining sins, and of administering the Sacraments.'' -- \emph{Augsb. Conf.}, Art. 28, 5.

``In addition to this it is necessary to confess that the keys pertain not to the person of a particular man, but to the Church, as many most clear and firm arguments testify. For Christ, speaking concerning the keys (Matt. 18, 19), adds: `If two of you shall agree on earth, as touching anything that they shall ask, it shall be done for them of My Father which is in heaven.' Therefore He ascribes the keys to the Church principally and immediately, just as also for this reason the Church has principally the right of calling. For just as the promise of the Gospel belongs certainly and immediately to the entire Church, because the keys are nothing else than the office whereby this promise is communicated to every one who desires it, just as it is actually manifest that the Church has the power to ordain ministers of the Church. And Christ speaks in these words, `Whatsoever ye shall bind,' etc., and means that to which He has given the keys, namely, the Church: `Where two or three are gathered together in My name' (Matt. 18, 20). Likewise Christ gives supreme and final jurisdiction to the Church when He says: `Tell it to the Church.'\,'' -- \emph{Smalc. Art.}, App., § 24.

``In case of necessity even a layman absolves, and becomes the minister and pastor of another.'' -- \emph{Ibid}., § 67.

\begin{center}
\textsc{Error.}
\end{center}

\begin{enumerate}
\def\labelenumi{\arabic{enumi}.}
\item
  The well-known position of the \emph{Catholic Church}, that ``no one can enter heaven unless a \emph{priest} has opened the door'' (\emph{Rom. Cat.}, II, 5, 43), need here only be mentioned. And the oft-repeated saying that St.~Peter is the doorkeeper of heaven is only a monastic fable.
\item
  The \emph{Reformed Churches}, one and all, practise neither private nor public absolution. Zwingli ridiculed it, and Calvin says: ``The power of the keys is simply the preaching of the Gospel,\ldots and in so far as men are concerned, it is not so much power as ministry. Properly speaking, Christ did not give this power to men, but to His Word, of which He made men the ministers.'' -- \emph{Instit.}, IV, 11, 1.
\item
  The \emph{Episcopalians} have substituted for absolution this formula in \emph{Warning for the Celebration of the Holy Communion}: ``If there be any of you who by this means cannot quiet his own conscience herein, but requireth further comfort or counsel, let him come to me, or to some other minister of God's Word, and open his grief that he may receive such godly counsel and advice as may tend to the quieting of his conscience and the removing of all scruple and doubtfulness.'' -- \emph{Book of Com. Prayer}, Exhort.
\item
  The \emph{Presbyterians} speak only of ``Church Censure'' (\emph{Westm. Conf.}, XXX), and the \emph{Methodists} have invented the ``class-meeting'' with its ``class-leader'' as a substitute for the ordinance of God. -- \emph{Ch. Discipl.}, II, 28.
\end{enumerate}

\textsc{Note. --} The Scriptures speak of the Office of the Keys in a two-fold manner.

\begin{enumerate}
\def\labelenumi{\arabic{enumi}.}
\item
  When the Word of God is preached, His wrath and condemnation is pronounced on sin and impenitent sinners. Though the preacher does not know of every sin, nor has any particular sinner in view, yet the sins of the impenitent and unbelieving ar bound and retained in heaven through his preaching. (Matt. 24: The Jews were doomed for not accepting the salvation offered. Luke 10: Woe was pronounced upon Chorazin, Bethsaida, Capernaum, and the places that would not accept the preaching of the apostles. Also chap.~13; Acts 13, 40. 41; Eph. 4, 4: ``For this ye know, that no whoremonger, nor unclean person, nor covetous man, who is an idolater, hath any inheritance in the kingdom of Christ and of God.'' 2. Thess. 1, 8: ``The Lord Jesus shall come to''take vengance on them that know not God, and that obey not the Gospel of our Lord Jesus Christ.'')
\item
  When one who is conscience-striken and weak in faith cannot find full consolation in the Word as preached to the Christian congregation, but anxiously asks whether he, the great, unworthy sinner, is included among those towards whom God is gracious or not, or what decree may have been passed upon him in the council of the Most High, to such a one Christ not only in general causes the Gospel to be preached, but in order to augment and strengthen his faith and give him an abiding consolation, He also privately and personally assures him of the forgiveness of sin. (Cf. Matt. 9; Luke 7, 19. 23.) And this loosing key, together with its attendant gracious promise, He has given unto His servants to use. This is commonly called \emph{private absolution}. (Adapted from Chemnitz's \emph{Enchiridion}.)
\end{enumerate}

\begin{center}
\textbf{E.  The Lord's Supper}

\textsl{1.  What is the Lord's Supper?}

\textsc{Biblical Lutheran Doctrine.}
\end{center}

``The Lord's SUpper is the true body and blood of our Lord Jesus Christ, under the bread and wine, given unto us Christians to eat and to drink, as it was instituted by Christ Himself.'' -- Matt. 26, 26-28; Mark 14, 22-24; Luke 22, 19. 20; 1 Cor. 11, 23-25.

The Tenth Article in the \emph{Augsburg Confession} is substantially the same as the above. In his \emph{Apology}, Melancthon adds: ``We defend the doctrine received in the entire Church, that in the Lord's Supper the body and blood of Christ are truly and substantially present, and are truly tendered with those things which are seen, bread and wine. And we speak of the presence of the living Christ, knowing that `death hath no more dominion over Him.'\,'' -- Art. X.

\begin{center}
\textsc{Error.}
\end{center}

\begin{enumerate}
\def\labelenumi{\arabic{enumi}.}
\tightlist
\item
  \emph{Zwingli}, the father of Reformed rationalism, says in his preface to \emph{A Short Christian Instruction to the Clergy}: ``The Lord's Supper is nothing more than the feast of the soul; and Christ instituted it as a remembrance of Himself. When man trusts in the suffering and redemption of Christ, he shall be saved. Of this He has left us a \emph{sure and visible sign} in the emblems of His body and blood, and entreats us to eat and to drink both in remembrance of Him.'' -- Winer, Clark's ed., p.~269. (Compare \emph{Darst.}, XVI, 2.)
\end{enumerate}

In his \emph{Reckoning of Faith} (presented at Augsburg, 1530) this is his standpoint: ``eighthly, I believe that in the Holy Eucharist, \emph{i. e.}, the supper of thanksgiving, the true body of Christ is present by the contemplation of faith; \emph{i. e.}, that they who thank the Lord for the kindness conferred on us in His Son, acknowledge that He assumed true flesh, in it truly suffered, truly washed away our sins in His own blood; and thus everything done by Christ becomes present to them by the contemplation of faith. But that the body of Christ in essense and really -- \emph{i. e.}, the natural body itself -- is either present in the Supper or masticated with our mouth or teeth, as the papists and some who long for the fleshpots of Egypt assert, we not only deny, but firmly maintain is an error opposed to God's Word.'' -- Jacobs, \emph{Book of Concord}, I, 170.

\begin{enumerate}
\def\labelenumi{\arabic{enumi}.}
\setcounter{enumi}{1}
\item
  \emph{Calvin} says in his \emph{Institutes}: ``How, then, could they {[}the disciples{]} have been so ready to believe \emph{what is repugnant to all reason, viz.}, that Christ was seated at table under their eye, and yet was contained invisibly under the bread?'' -- IV. C. XVII, § 23.)
\item
  \emph{The Heidelberg Catechism} (by Ursinus and Olevianus, 1563) is the mother confession of all later Reformed confessions. Of the Eucharist it says: ``The Lord's Supper is a distributing and receiving of bread and wine commanded of Christ unto the faithful, that by \emph{these signs} He might testify that He has delivered and yielded His body unto death, and has shed His blood for them, and does give them these things to eat and drink, that they might be unto them the meat and drink of eternal life, and that thereby also He might testify that He would dwell in them, nourish and quicken them forever.'' (Quest. 75, Ursin's \emph{Explication}; Henry Parry's translation.) -- Again: ``To eat is to believe, to receive remission of sin by faith, to be united to Christ, to be made partakers of the life of Christ.'' (Qu. 76.) Again: ``The literal sense, if it be properly taken, can be \emph{no other wise} understood than thus: The substance of this bread is the substance of My body. But so to understand it is an \emph{undoubted absurdity}.'' (Qu. 77.)
\item
  \emph{The Book of Common Prayer (Episcopal)} says in ``a catechism\ldots to be learned\ldots before confirmation'': ``Why was the Sacrament of the Lord's Supper ordained? \emph{Ans.}: For the continual remembrance of the sacrifice of the death of Christ, and of the benefits which we receive thereby. What is the inward part, or thing signified? \emph{Ans.}: The body and blood fo Christ, which are \emph{spiritually taken and received} by the faithful in the Lord's Supper.'' (p.~228.) -- In the Episcopal \emph{Articles of Religion}, Art. XXVIII speaks in the first part of ``partaking of the body and blood of Christ,'' but concludes by stating that ``The body of Christ is given, taken, and eaten, in the Supper, only after \emph{an heavenly and spiritual manner}. And the means whereby the body of Christ is received and eaten in the Supper is faith.'' (Reformed Episcopals the same in substance.)
\item
  The \emph{Presbyterians} say: ``Worthy receivers, outwardly partaking of the visible elements in this Sacrament, do then also inwardly by faith, really and indeed, yet not carnally and corporally, but spiritually, receive and feed upon Christ crucified, and all benefits of His death: the body and blood of Christ being there not corporally or carnally in, with, or under the bread and wine; yet as really, but spiritually, present to the faith of believers in that ordinance, as the elements themselves are to their outward senses.'' -- \emph{Westm. Conf.}, Art. XXIX, 7.
\item
  The \emph{Methodist Episcopal Church} in its \emph{Articles of Religion}, Art. XVIII, is one with the Episcopalians. The \emph{Baptist} and \emph{Congregationalist} position is that of the \emph{Westminster Confession}.
\end{enumerate}

\begin{center}
\textsl{But God says:--}
\end{center}

Matt. 26, 26-28: ``Jesus took bread, and blessed it, and brake it, and gave it to the disciples, and said, Take, eat: \emph{this is My body}. And He took the cup, and gave thanks, and gave it to them, saying, Drink ye all of it: \emph{For this is My blood} of the New Testament, which is shed for many for the remission of sins.''

1 Cor. 10, 16: ``The cup of blessing which we bless, is it not \emph{the communion of the blood of Christ?} The bread which we break, is it not \emph{the communion of the body of Christ?}''

1 Cor. 11, 27: ``Wherefore whosoever shall eat this bread, and drink this cup of the Lord, unworthily, shall be \emph{guilty of the body and blood of the Lord}.'' (Cf. v. 29.)

Matt. 27, 20: ``Lo, I am with you alway, even unto the end of the world.''

\textsc{Note. --} When the Reformed deny the real presence in the Lord's Supper and say that Christ is only \emph{spiritually present}, how can they explain the situation at the \emph{first} Communion? At that time the participants did not need to ``soar to heaven on the wings of faith'' to get His blessing. The Savior was bodily and spiritually right in the midst of them. But they received a blessing, surely, and that was what it can only be if Christ meant what He said, His body and blood, for the remission of sins.

\begin{center}
\textsl{2.  What Benefits Do Christians Derive from the Lord's Supper?}

\textsc{Bible Doctrine.}
\end{center}

The Lord's Supper, being a means of grace, confers upon the believing partaker forgiveness fo sins, life, and salvation.

Matt. 26, 28: ``For this is My blood of the New Testament, which is shed for many \emph{for the remission of sins}.'' (Luke 22, 19. 20.)

1 Cor. 11, 29: ``The \emph{unworthy} partaker''eateth and drinketh damnation to himself.'' Hence, the \emph{worthy partaker} must get ``forgiveness of sins, life, and salvation.''

\begin{center}
\textsl{Lutheran Testimony:--}
\end{center}

``We have briefly considered the first part, namely, the essence of this Sacrament. We now come to its power and blessing, which is a most important part, as we should know what we go for, and what we receive. This is plainly evident from the words just quoted: `This is My body and blood, given and shed for you for the remission of sins.' In other words, we go to the Communion because we receive there a treasure through and in which we obtain the forgiveness of sins. How so? There stand the words through which this is imparted! When He bids me go to eat and to drink, it is with the intent that it should be mine, and be a source of blessing to me as a pledge and earnest thereof, yea, as the very gift in which I am to find shelter against sin, death, and every misfortune.'' -- Luther's \emph{Large Cat.}, V, 20 f.; Lenker's Tr.

\begin{center}
\textsc{Error.}
\end{center}

\begin{enumerate}
\def\labelenumi{\arabic{enumi}.}
\item
  As the \emph{Reformed Churches} deny the real presence in the Lord's Supper, so they also deny the real benefit thereof. But this makes it a Sacrament in name only. \emph{Presbyterians}, \emph{Congregationalists}, and \emph{Baptists} unite in confessing that ``The Lord Jesus instituted the Sacrament of His body and blood\ldots for the perpetual \emph{remembrance} of the sacrifice of Himself in His death,'' etc. -- \emph{Westm. Conf.}, XXIX, 1.
\item
  The \emph{Methodists} speak ambiguously in their \emph{Articles of Religion} of the Lord's Supper. (Cf. XVI and XVIII. And what is memant by the following declaration: ``The Supper of the Lord\ldots is a Sacrament of our redemption by Christ's death''?) But they are unmistakably Reformed in practise and public teaching. Dr.~W. Nast has the following in his \emph{Smaller German Catechism}: ``What does the Lord's Supper signify? It is to the believer a \emph{memorial feast} of Christ's atoning passion and death and a \emph{token} of his participation in the benefits of Christ.''
\end{enumerate}

\textsc{Note 1. --} No Christian dogma has a larger literature to its credit than the Lord's Supper. The theologians of the early Church as a whole, decidedly taught the real presence; but their lack of precision, at times, has given fanatics and Romanists arguments for the defense of their own strange opinions concerning the Sacrament of the Altar.

Like Popery itself, the present doctrine of the Lord's Supper in the Catholic Church is a rather late development. In 844, the French monk Paschasius Radbertus published a work wherein the change of bread and wine into the flesh and blood of Christ was vigorously defended (metabolism, transubstantiation). In 1214, at the Fourth Lateran Council, this unscriptural opinion was made the doctrine of the Church (reiterated at the Council of Trent, in 1551); and in 1415, at the Council of Constance (where John Hus was burned), it was finally decreed that the consecrated \emph{bread alone} should be distributed to the lay people (concomitance), the wine to be drunk by the officiating priest, to insure against any of the ``precious drops'' being spilled in any manner! And finally, the consecrated elements are sacrificed as a ``bloodless offering,'' espeically for the dead in purgatory by being elevated in the mass; or the consecrated wafer is carried about in solemn procession to be \emph{worshipped} by the populace.

These abuses were soon discarded by the Reformers (\emph{Augsb. Conf.}, XXIV) but not all reverted to apostolic practise. Zwingli and Calvin differ somewhat in their definitions of the Sacrament of the Altar, but \emph{agree} in denying the real presence. And their \emph{chief} reason for doing so is the rationalistic argument: human reason cannot understand the real presence of Christ's body and blood in the Sacrament.

\textsc{Note 2. --} Reformed writers on the Lord's Supper generally call the Lutheran doctrine of the real presence ``consubstantiation.'' This term means, ``of the same substance,'' \emph{viz.}, that in the Sacrament the bread and the body of Christ form one substance. Those who apply the term ``consubstantiation'' to the Lutheran doctrine of the Lord's Supper misprepresent our position. We do not teach that the bread and the body of Christ, or the wine and His blood, form one substance, but that in, with, and under the bread we receive, not in a natural, but supernatural (sacramental) manner, the true body of Christ, and with the wine His true blood. This is the plain doctrine of Scripture, though too deep for human reason to fathom.

\begin{center}
\textsc{A Few Objections Answered.}
\end{center}

\emph{First Objection.} -- The word ``is,'' in the words of institution, must be taken to mean ``\emph{signifies}.''

\emph{Answer.} -- The words of institution are alike wherever quoted. If any figurative sense were intended, some author would surely have mentioned it. If the words of institution are to be understood figuratively, then no scripture can be taken literally. For no doctrine is more plainly stated than that of the Lord's Supper.

Some texts quoted by Reformed writers to support their doctrine that ``is'' may mean ``signifies'' prove the contrary: John 14, 6: ``I am the Way, the Truth and the LIfe''; 10, 9: ``I am the Door.'' Here Christ does not \emph{signify} these things, but He verily \emph{is} what He claims to be.

It is agreed on every hand that the Old Testament paschal lamb has been superseded by the Lord's Supper. This lamb signified Christ, who as a Lamb should be sacrificed for the sin of the whole world. (John 1, 29.) Now that Christ has come, ``signifies'' has been, yeah, \emph{must} be, superseded by ``is.'' As the body is more than the shadow, so the Eucharist is more than the paschal lamb.

\emph{Second Objection.} -- As Christ's body has ascended into heaven, His presence in the Eucharist can be only spiritual.

\emph{Answer.} -- Though Christ ascended into heaven, this does not prevent him from being present in the Eucharist. First: He is almighty, and can do whatsoever He pleases. Secondly: He is truthful. He has promised to be present, and He surely will. (1 Cor. 10, 16.) Thirdly: He is everywhere present. The right hand of God is everywhere, so Christ also is everywhere. Fourthly: Though Christ is invisible, His presence since the incarnation is always real. Or does he leave His body at the right hand of the Father while He is present spiritually somewhere? No! His presence is the presence of the \emph{personal, undivided God-man}.

To partake of Christ spiritually is to believe on Him. Of this partaking Christ speaks in John 6. Zwingli, and the Reformed generally, use this chapter to prove their side of the case. But how can Jesus there refer to the Sacrament, which was not yet instituted? Besides, 1 Cor. 11, 27-29 forbids us to bleieve that there is only a spiritual body, a spiritual eating and drinking, in the Lord's Supper, since according to this chapter even unbelievers partake of Him.

\emph{Third Objection.} -- The body of Christ cannot be present in the Eucharist because it has not the required attribute of omnipresence.

\emph{Answer.} -- Christ's body participates in the attributes of the divine nature. He endured fasting forty days and forty nights (Matt. 4, 2); did not sink in the water (Matt. 14, 22); became invisible when the Jews counseled murder (John 8, 59); came out of the sealed grave; was visible and, in an instant, invisible (John 20, 14 f.); came through closed doors (John 20, 19. 26). Christ's human nature also participates in the attribute of omnipresence. According to His last will and testament, in which He has expressed Himself clearly and fully concerning our heritage, we believe in the real presence of His body and blood in the sacrament.

\textsc{Note. --} The Reformed insist on breaking the bread at Communion, 1) because Christ did so at the Last Supper; and 2) because it signifies the sufferings which Christ had to undergo in order to redeem us. We do not hold that this breaking of the bread is essential. Christ broke the bread because the loaves were too large for the purpose in view; and there is no special significance connected with the act. He certainly did not break the bread in order to signify the treatment He was to receive; for the Scriptures state explicitly that His body \emph{must not be broken} (John 19, 63), and the bones of the paschal lamb were not broken (Ex. 12, 46).

When the Lutheran Church uses unleavened wafers at Communion, it does so because they are convenient to distribute, and because the bread used by Christ was unleavened. In like manner it also uses \emph{grape-wine}, not -juice, nor a juice of some other fruit, or merely water, as some temperance fanatics prescribe, because this was the wine of the Passover, and because Christ explicitly mentions ``the fruit of the vine'' (Matt. 26, 29) as the proper element.

\section{After Death}\label{after-death}

\begin{center}
\textsc{Bible Doctrine.}
\end{center}

At death the soul enters on its eternal existence: bliss or torment. OF this the body will also partake after Judgement Day.

Gen.~3, 19: ``In the sweat of thy face shalt thou eat bread, till thou return unto the ground, for out of it wast thou taken; for dust thou art, and unto dust shalt thou return.''

Eccl. 12, 7: ``Then shall the dust return to the earth as it was; and the spirit shall return unto God, who gave it.''

Luke 23, 43: ``To-day thou shalt be with Me in paradise.''

Luke 16, 22. 23: ``And it came to pass that the beggar died, and was carried by the angels into Abraham's bosom. The rich man also died, and was buried; and in hell he lift up his eyes, being in torment.''

Rev.~14, 13: ``Blessed are the dead which die in the Lord from henceforth; yea, saith the Spirit, that they may rest from their labors; and their works do follow them.''

Heb. 9, 27: ``It is appointed unto men once to die, but after this the Judgement.''

Phil. 1, 23: ``I desire to depart, and \emph{to be with Christ}, which is far better.''

Acts 1, 25: ``Judas by transgression fell that he might go to \emph{his own place}.''

\begin{center}
\textsl{Lutheran Testimony:--}
\end{center}

``I believe in Jesus Christ,\ldots who ascended into heaven, and sitteth on the right hand of God the Father Almighty; from thence He shall come again to judge the quick and the dead.'' -- \emph{Apostles' Creed}, Sec. Art. \emph{Athanasian Creed}, § 37. \emph{Augsb. Conf.}, Arts. III and XVII.

``We pray in this petition\ldots that when, at last, the hour of death shall arrive (He would) grant us a happy end, and graciously take us from this world of sorrow to Himself in heaven.'' -- \emph{Small Cat.}, III, 7.

\begin{center}
\textsc{Error.}
\end{center}

\begin{enumerate}
\def\labelenumi{\arabic{enumi}.}
\item
  The \emph{Catholic Church} teaches that ``there exists in the next life a middle state of temporary punishment, allotted for those who have died in venial sin, or who have not satisfied the justice of God for sins already forgiven\ldots. This intermediate state is commonly called purgatory.'' (Gibbons, \emph{Faith of Our Fathers}, p.~248.) This declaration is substantially the same as that contained in canon 30, session 6, of the Council of Trent, and in question 3, part I, 6, of the \emph{Roman Catechism}. -- When the cardinal says that the doctrine of purgatory ``is clearly taught in the Old Testament; that it is, at least, insinuated in the New Testament; that it is unanimously proclaimed by the Fathers of the Church,'' but includes in the Old Testament the apocryphal books of the Maccabees (see 2 Macc. 12, 43-46), and mentions that the Monatist Tertullian, A. D. 160-220, as the first testifying ``Father,'' his ``clear'' and ``unanimous'' proofs do not appear convincing.
\item
  Though the representative \emph{Reformed} Churches stand for retribution in their confessions, the prevailing opinion among their members is that ``we get all the hell there is here on earth.'' Such as opinion is not even \emph{rationalism}. The conception of a just God forbids it. Yet the REformed sects in their preaching and in their literature very generally ignore the Bible-doctrine concerning a state of punishment after death.
\end{enumerate}

Lodgism is to a great extent the cause for disbelief in eternal punishment. Every lodge-man buried, even a suicide or a profligate Elk, is declared saved, ``removed to the Grand Lodge above'' or to ``the happy hunting-ground'' of the Red Man.

\begin{center}
\textsl{But God says:--}
\end{center}

2 Cor. 5, 10: ``We must all appear before the judgment-seat of Christ, that every one may receive the things done in his body, according to that he hath done, whether it be good or bad.'' (1 Cor. 15.)

Dan. 12, 2: ``And many of them that sleep in the dust of the earth shall awake, some to everlastin life and some to shame and everlasting contempt.''

Rom. 14, 10: ``We shall all stand before the judgment-seat of Christ.''

Matt. 25, 41. 46: ``Then shall He say unto them on the left hand: Depart from Me, ye cursed, into everlasting fire, prepared for the devil and his angels. And these shall go away into everlasting punishment, but the righteous into life eternal.''

\end{document}
