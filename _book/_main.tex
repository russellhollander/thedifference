% Options for packages loaded elsewhere
\PassOptionsToPackage{unicode}{hyperref}
\PassOptionsToPackage{hyphens}{url}
%
\documentclass[
]{book}
\usepackage{amsmath,amssymb}
\usepackage{iftex}
\ifPDFTeX
  \usepackage[T1]{fontenc}
  \usepackage[utf8]{inputenc}
  \usepackage{textcomp} % provide euro and other symbols
\else % if luatex or xetex
  \usepackage{unicode-math} % this also loads fontspec
  \defaultfontfeatures{Scale=MatchLowercase}
  \defaultfontfeatures[\rmfamily]{Ligatures=TeX,Scale=1}
\fi
\usepackage{lmodern}
\ifPDFTeX\else
  % xetex/luatex font selection
\fi
% Use upquote if available, for straight quotes in verbatim environments
\IfFileExists{upquote.sty}{\usepackage{upquote}}{}
\IfFileExists{microtype.sty}{% use microtype if available
  \usepackage[]{microtype}
  \UseMicrotypeSet[protrusion]{basicmath} % disable protrusion for tt fonts
}{}
\makeatletter
\@ifundefined{KOMAClassName}{% if non-KOMA class
  \IfFileExists{parskip.sty}{%
    \usepackage{parskip}
  }{% else
    \setlength{\parindent}{0pt}
    \setlength{\parskip}{6pt plus 2pt minus 1pt}}
}{% if KOMA class
  \KOMAoptions{parskip=half}}
\makeatother
\usepackage{xcolor}
\usepackage{longtable,booktabs,array}
\usepackage{calc} % for calculating minipage widths
% Correct order of tables after \paragraph or \subparagraph
\usepackage{etoolbox}
\makeatletter
\patchcmd\longtable{\par}{\if@noskipsec\mbox{}\fi\par}{}{}
\makeatother
% Allow footnotes in longtable head/foot
\IfFileExists{footnotehyper.sty}{\usepackage{footnotehyper}}{\usepackage{footnote}}
\makesavenoteenv{longtable}
\usepackage{graphicx}
\makeatletter
\def\maxwidth{\ifdim\Gin@nat@width>\linewidth\linewidth\else\Gin@nat@width\fi}
\def\maxheight{\ifdim\Gin@nat@height>\textheight\textheight\else\Gin@nat@height\fi}
\makeatother
% Scale images if necessary, so that they will not overflow the page
% margins by default, and it is still possible to overwrite the defaults
% using explicit options in \includegraphics[width, height, ...]{}
\setkeys{Gin}{width=\maxwidth,height=\maxheight,keepaspectratio}
% Set default figure placement to htbp
\makeatletter
\def\fps@figure{htbp}
\makeatother
\setlength{\emergencystretch}{3em} % prevent overfull lines
\providecommand{\tightlist}{%
  \setlength{\itemsep}{0pt}\setlength{\parskip}{0pt}}
\setcounter{secnumdepth}{5}
\ifLuaTeX
  \usepackage{selnolig}  % disable illegal ligatures
\fi
\IfFileExists{bookmark.sty}{\usepackage{bookmark}}{\usepackage{hyperref}}
\IfFileExists{xurl.sty}{\usepackage{xurl}}{} % add URL line breaks if available
\urlstyle{same}
\hypersetup{
  pdftitle={The Difference},
  pdfauthor={I.G. Monson},
  hidelinks,
  pdfcreator={LaTeX via pandoc}}

\title{The Difference}
\usepackage{etoolbox}
\makeatletter
\providecommand{\subtitle}[1]{% add subtitle to \maketitle
  \apptocmd{\@title}{\par {\large #1 \par}}{}{}
}
\makeatother
\subtitle{A Popular Guide to Denominational History and Doctrine}
\author{I.G. Monson}
\date{1915}

\begin{document}
\maketitle

{
\setcounter{tocdepth}{1}
\tableofcontents
}
\hypertarget{preface}{%
\chapter*{Preface}\label{preface}}

What is the difference between the various Christian sects? The common answer is that there is no difference, that they all have the same Bible, that all are trying to reach the same place, etc. For the ignorant this is the easiest way to get away from this question. The lazy, indifferent, and the unbeliever will find the same answer the most convenient. But any one interested in truth and in Christianity will gladly accept information concerning the most important questions in life.

Of the many good works on the subject none are suited for general distribution among the several Lutheran church-bodies in this country. In the theological seminaries the works of Winer, Guericke, Guenther, and Gumlich are, of course, indispensible; but in the preparatory schools, parochial and Sunday-schools, as well as in the average Lutheran home, such learned works are impractical, if not wholly useless.

The aim has, therefore, been in the present volume to be more suggestive than exhaustive, to avoid all technicalities, and to treat the different dogmas in such a manner that the common, every-day Christian could understand them.

In order to attin this end, the dogmatic, positive, Biblical, Lutheran method was found most serviceable. And for this we make no apology, the cry of ``represtination of Reformation theology'' by rationalists and unionists to the contrary notwithstanding. As God at the last day will judge man according to His Word (John 12, 48), the only right way is to ask for the Old Paths, where is the good way, and walk therein (Jer. 6, 16).

Now and then names and phrases will be found inclosed in brackets. This has been done in order to stimulate further research by both teacher and student. Books of reference of some kind are now in nearly every home.

To write on Symbolism in our unionistic and irreligious age is, indeed, a thankless task. And yet \emph{the full truth was never popular}. Now more than in the last century the words of Stahl have full force: ``The powers that be are against us. the masses are against us. The tendency of the times is against us. The powerful errorists within the Church are against us.'' To this may here be added: The Reformed civilization under which we live is against us.

But God has a remnant yet to save. \emph{And it will be saved, but only through the Word of Truth}.

\textsc{The Author.}

\hypertarget{part-i.}{%
\chapter*{PART I.}\label{part-i.}}
\addcontentsline{toc}{chapter}{PART I.}

\hypertarget{origin-of-divided-protestantism}{%
\section{Origin of Divided Protestantism}\label{origin-of-divided-protestantism}}

When Luther in Germany opposed Roman Catholicism and popery, it was because the Roman Church had perverted the Gospel of Christ, taught righteousness by works, and prohibited the circulation of the Bible among the people. Strive as he would, his troubled soul could find no consolation in following the precepts of the Church. It was only after years of fruitless toil, in and out of the cloister, that he had at last found peace and consolation in the precious promise of the Lord: ``The just shall live by his faith'' (Hab. 2, 4)

The Swiss Reformers, Calvin and Zwingli, on the other hand, do not seem to have been thus troubled. Though earnest in their reform work, they worked more in the direction of emancipating their Churches from the yoke of the Roman organization, in eradicating superstition, correcting outward, flagrant abuses, and instilling into the minds of the people a spirit of independence, political as well as religious.

Luther contended that an effectual reformation must come from within; for if the tree is made good, the fruit will also be good. The Swiss made the branches, instead of the root, their point of attack.

Luther's weapon in the battle of the Reformation was the Word of God only. With this means, he said, the Savior repelled even Satan himself. With this means the apostles conquered the heathen world and laid it at the feet of Christ. To the Swiss this method was too slow; they were not content with letting the leaven of the Gospel work in its peculiar way; they must needs also use physical force and political power, a trait to this day characteristic of the Reformed Church.

Closely allied to this was the different views taken by the Reformers as to the value of the Scriptures. Luther contended that the Word of God was not merely a message to mankind, but the vehicle of the Holy Spirit through which He gives the proffered grace, a truth attested by the Apostle Paul, when he says that the Holy Spirit was given through the preaching of the Gospel (Gal. 3, 2), and faith through the Word of God (Rom. 10, 17).

Zwingli, on the other hand, believed that the Spirit needed no means, that He worked independently of the Scriptures. The preaching of the Word was to him only ``an outward sound,'' outside of which the Holy Spirit must work regeneration and salvation. Hence, he hoped ``that good men in all ages, who had walked uprightly according to the light they had, were also saved.'' (\emph{Expos. Fidei}, 2, 559.)

But the real difference between the Reformers came to light in their discussion concerning the authority of the Scriptures. Luther contended with the Apostle Paul that the natural man receiveth not the things of the Spirit of God, for they are foolishness unto him, neither can he know them, because they are spiritually discerned. (1 Cor. 2, 14) Hence, if he would not be adjudged wise in his own conceit (Rom. 12, 16), he must not think above that which is written (1 Cor. 4, 6), but bring into captivity every thought to the obedience of Christ (2 Cor. 10, 5). History as well as his own experience had taught him that it was futile to worship God, teaching for doctrines the commandments of men (Matt. 15, 9), and therefore he set up this formula in the Smalcald Articles: ``It is certain that the Word of God alone shall set up articles of faith, and none else, not even an angel.'' (II, 15.)

No doubt, the Swiss would have said Yes and Amen to this, adding the rationalistic phrase: ``rightly understood.'' (John Calvin declared that he had signed the Augsburg Confession in this manner.) They did not deny the authenticity of any part of the canonical Scriptures, but ever and anon they would follow human reason, and assert that the language of the Holy Spirit must not be taken in a literal, but in a figurative sense, as, for instance, in the doctrine of the Lord's Supper. Any other interpretation, they said, would be ``absurd'' and ``unreasonable.'' What becomes of the Scriptures under such treatment we shall observe as we enter more fully into our subject.

That such opposite views concerning the Scriptures should be followed by Different methods of church-work is not surprising.

To Luther only such changes seemed necessary as would promote true piety. Any ceremony, therefore, which had come down from antiquity, and was not connected with any idolatrous practice, or could be cleansed from such, was retained. Church interiors, with altars, pulpits, sacristies, organs, paintings, and statuary, were retained, also clerical vestments and the observance of many holidays; in short, anything and everything which could be used properly, and was not prohibited by the Word of God.

To the Swiss everything Catholic was ``a heathenish abomination.'' What the Scriptures did not directly command was to them unscriptural, and hence they not only countenanced the ruthless destruction of statuary and paintings in churches, but even counseled to do so. Church interiors were entirely remodeled, and ceremonies of the most simple kind instituted. This difference between the Reformed and the Lutheran position may best be illustrated by an anecdote. In discussing the retention or rejection of the old liturgies, Zwingli said, ``Show me a passage in the Scriptures wherein they are enjoined.'' To which Luther replied, ``Show me a passage wherein they are condemned.''

The different stand taken by the Reformers concerning Church and State must not be overlooked. The Lutheran Church soon became a State Church, it is true, but in spite of the teachings of Luther. ``It is with the Word we must contend,'' he observed, ``and by the Word we must refute and expel what has gained a footing by violence. I would not resort to force against such as are superstitious, nor even against unbelievers! Whosoever believeth let him draw nigh, and whoso believeth not, stand afar off. Let there be no compulsion. Liberty is of the very essence of faith. God does more by the simple power of His Word than you and I and the whole world could effect by all our efforts put together! God arrests the heart, and that once taken, all is won.''

To the Swiss Reformers the Jewish Church of the Old Testament was the only correct model. Zwingli looked upon the Council as representative both of Church and State, and the theocratic State Church organized by Calvin at Geneva is a fact well known to all students of history. Even in America the Reformed Church commenced as a State Church, and the roots of it are still sprouting such shoots as Sunday-laws, chaplains for conventions, Congress, legislatures, and for the army and navy, and the fanatical activities of the National Reform Association.

The Reformed Churchis split up into a multitude of sects. They have no confession to which they all subscribe; many of them have no confession at all, and others modify their beliefs as occasion demands. This is because a belief founded partly on ``reason,'' ``views,'' ``science,'' ``feeling,'' and the like, can have no stability. Views change; the rational way of one is irrational to another; the science of to-day is a fable to-morrow. In order, therefore, to gain recruits for this or that sect, innumerable schemes, methods, and measures are resorted to, -- camp-meetings, revivals, the Y. M. C. A., and of late the Laymen's Movements, Sociali Service, etc.

We shall now pass in review the principal denomination which have a following in the United States. Our next chapter briefly summarizes the history of these denominations, and succeeding chapters will take up, one at a time, the chief doctrines of Scripture, stating the agreement of our Lutheran Confessions with the same, and the differences which exist between us and the various denominations.

\hypertarget{historical-summaries}{%
\section{Historical Summaries}\label{historical-summaries}}

\hypertarget{the-anglican-church}{%
\subsection{THE ANGLICAN CHURCH}\label{the-anglican-church}}

Many dignitaries of the Anglican Church do not, at the present time, like to date the beginning of their denomination from the time of the Reformation. And some would like to ignore the Reformation altogether. Says one: ``England has never had but one religion; it has never changed that religion, and it has that religion still.''

From the sixth century until the year 1534, when Henry VIII was refused the sanction of the pope to a divorce from his first wife (he had six in all), England was as much a Catholic country as any other of the European states. In that year, to avenge himself on the pope, Henry ``denied the papal supremacy, and declared himself to be the one protector of the English Church, its only and supreme lord, and, so far as might be by the laws of Christ, its supreme head.'' And this Parliament later confirmed. This was the beginning of the revolt which carried England outside the Catholic fold.

Writings of Luther and, later, of the Swiss Reformers were early and eagerly read in England. Henry, who had written against Luther before the breach with the pope, continued in the Church, and, as far as Roman doctrine is concerned, died a Catholid; but when his son, Edward VI, ascended the throne, in 1547, reform ideas were so well rooted that Archbishop Thomas Cranmer, counseled mainly by Reformed theologians (Martin Bucer, Peter Martyr), in 1552 set up a confession of faith containing forty-two articles, later cut down to thirty-nine at the Synod of London, in 1562. This confession was ratified by Parliament in 1571, and thereby became a part of the law of the land. ``The Book of Common Prayer,'' which is still in use, contains not only this document, but also a full ritual for all occasions during the church-year.

The Thirty-nine Articles, the confession of the Anglican Church, are decidedly Reformed in tone and tendency, though in many respects taking a middle course between the Lutheran and the Reformed. But when Episcopal missionaries say to Lutheran immigrants that their Church is the Lutheran, excepting the language, they ought to know that they are not telling the truth. The Episcopal Church in the United States is only the Anglican Church without state-connections. It is officially known as The Protestant Episcopal Church. As such it dates from 1787, when it received its first bishop, Dr.~Seabury, and constituted itself independent of England.

In 1873, it suffered a small loss through a few priests, who left it on account of a ritualistic (Romish) tendency, which was rapidly gaining ground in the Church. They reduced the confession to thirty-five articles, and changed a few expressions here and there. They are known as The Reformed Episcopal Church.

Though composed of three more or less antagonistic elements, the ``high,'' the ``low,'' and the ``broad,'' the Episcopal Church has escaped any serious division, and has remained virtually one united Church up to this time. It is governed by bishops, -- hence its name, -- and believes in the so-called Apostolic Succession, an invention inherited from popery.

In late years the Episcopal Church has become very lax in many ways. Higher criticism and rationalism dominate its theologians, and ritualism opens the way back to Rome for many every year.

In many Episcopal churches, both in our country and in England, prayers are said for the dead, the saints are worshipped, mass is read, and auricular confession to the priest is practised. Indeed, nothing but the claims of the supremacy and infallibility of the pope seems to stand in the way of a large section of the Episcopal Church going bodily into the Roman Catholic communion.

\hypertarget{presbyterians}{%
\subsection{PRESBYTERIANS}\label{presbyterians}}

The attitude of Henry VIII to the REformation, the instability of the government under his first successors, and the double influence of Lutheran and Reformed theologians during the stirring times of the sixteenth century did not allow the English people to study the problems of the Reformation properly. They were not only continually crossing the English Channel to and fro on account of persecutions of one kind or another, but they were also ``tossed hither and thither by all sorts of winds of learning,'' and as ``it is ever to the injury of essentials that the midn of man is preoccupied with secondary matters,'' the English church people suffered on this account more than any other nation. ``Dissenters,'' ``Independents,'' and ``Non-conformists,'' and ``Puritans'' are names which remind the serious-minded Evangelical Christian of to-day of the futile and unnecessary strifes of centuries ago in England, religious, of course, in a way, but mostly about ceremonies, vestments, church furnishings, and church government.

The principles underlying Presbyterianism came originally from Geneva, where John Calvin held sway in Church and State. This cold, unbending, and rigoristic reformer had patterned his church government mostly on Old Testament lines, and, through strict discipline, sought to attain, if possible, an immaculate Church on earth. Through John Knox, the Scotch reformer, these and other Calvinistic peculiarities were transplanted on English soil, where they not only became the source of endless troubles, but also the cause of an un-evangelical rigorism and persecutions.

Presbyterians are called so on account of their mode of church government. When a church is organized, a board of elders is chosen. The elders elect the minister, and these, called a ``session,'' rule in all temporal as well as spiritual affairs. Over the session, as a court of appeals, is the presbytery, composed of churches within a given area; a number of presbyteries make up a Synod, a still higher court, and the General Assembley, which meets every three years, constitutes the court of last resort.

But the Presbyterian Church has also been noted for its Calvinistic doctrine concerning predestination, the belief according to which God has loved, not the whole world, as Christ explicitly declared (John 3, 16), but only a few, namely the elect. To their Westminster Confession, adopted in 1647, a ``declarative statement'' was added a few years ago, which, to some degree, softens the language employed in the confession, but as the old confession still stands, the situation is practically unchanged.

The Presbyterians are divided into numerous sects. A few may be enumerated: ``The Associated Presbyterian Church''; ``The Reformed Presbyterian Church'' (Covenanters, Cameronians); ``The Associate Reformed Presbyterian Church''; ``The United Presbyterian Church''' ``The Cumberland Presbyterian Church,'' etc.

Some of the questions which divide these Presbyterian church-bodies are altogether trivial. The Southern Presbyterians and the United Presbyterian Church, for instance, remain separate mainly because they cannot agree on the question whether any hymns outside of the metrical versions of the Psalms should be sung in public worship. While the Southern Presbyterians use hymns and ``Gospel songs,'' The United Presbyterians restrict themselves to the use of the Psalms, in agreement with the strict Zwinglian idea that ``nothing is permitted except what the Bible commands.'' The correct, Lutheran, doctrine on this point is that only that is prohibited which the Bible forbids, and that all else is a matter of Christian liberty, and must not keep the Churches separate.

\hypertarget{congregationalists}{%
\subsection{CONGREGATIONALISTS}\label{congregationalists}}

It is a question whether Congregationalists may properly be called a denomination, as the different churches of this body have nothing in common but the name.

When Congregationalists maintain, ``That every Christian church is entitled to elect its own officers, to manage all its own affairs, and to stand independent of, and irresponsible to, all authority, saving that only of the Supreme and Divine Head of the CHurch, the Lord Jesus Christ,'' they so far utter a precious Scriptural truth. But when they condemn the formation of even \emph{advisory} church-bodie (synods), the acceptance of a common confession or creed by such bodies, and when in the confessions of their several local churches they sometimes use terms so vague that even a Brahmin could subscribe to them, then it is no wonder that the Universlaist and the Unitarian churches, \emph{which are not Christian}, originated in the Congregational Church in this country, and that even within the Congregational Church there is a large group of congregations in which even the elements of Christianity are no longer taught nor confessed.

Originally the Congregationalists were one with the Presbyterians in fighting Episcopalism in England, and with them accepted the Westminster COnfession, in 1643. Later they broke with their allies on account of the Presbyterian church government, and, chiefly through the fanatical Robert Brown (1586), constituted themselves as Independent churches.

In 1658, the Independents of London changed the Westminster Confession in a few minor points. (The Savoy Declaration.) THis altered confession was accepted by the Independents of this country at a synod in Boston, in 1680. But this acceptance they do not consider binding on anybody, not even on those who subscribe to it.

Both before and after Robert Brown took a hand in shaping the destiny of the Independents, they suffered severe persecutions in England. But, sad to relate, when, in 1620, they landed at Plymouth Rock, Mass., none became more intolerate than they (Pilgrim Fathers). Only their own church-members could become citizens. Church and State were one. Churchgoing was compulsory. Those who disagreed with them were sorely persecuted, unmercifully banished, or condemned to death.

The Congregationalists make special effort to draw Lutherans into their fold, and are not very particular about the means which they employ toward this end. In strong Lutheran communities they will, for instance, introduce a ceremony which closely resembles the Lutheran rite of confirmation, and some of their congregations have even adopted the Lutheran name (``Congregationalist Lutheran Church,'' etc.), with the sam epurpose in view.

\hypertarget{the-baptists}{%
\subsection{THE BAPTISTS}\label{the-baptists}}

Baptists are generally not very anxious to discuss the beginning of their denomination. They try to show at all times that Baptists, as a Church, never have persecuted other believers. Lately, however, teachers in their higher schools have been making researches into historical archives of the Old World, and one after the other concedes that ``early Baptists were called by their enemies Anabaptists.'' (\emph{Bapt. Congr.}, 1911, 106.)

During great political and religious upheavals there are always some connected with the movement who commit excesses or are dissatisfied with the final settlement. Of such there were not a few during the beginning of the Reformation. Some, as Muenzer, Storch, and others, called ``the heavenly prophets of Zwickau,'' found the Scriptures inadequate to their needs, and maintained that GOd had given them additional revelations. Among the first things which they discarded was Infant Baptism. After the death of Muenzer on the battlefield in the Peasants' War, 1524, John of Leyden gathered the prophet's followers, took possession of the city of Muenster, in Westphalia, and inaugurated a government along socialistic lines. Not only community of goods, but also polygamy was practiced. After the storming of the city by the neighboring princes, and execution of the chief leaders, their adherents were scattered throughout Switzerland, Germany, the Netherlands, and England, where they henceforth became known by the name Anabaptists, because, not acknowledging Infant Baptism, they rebaptized those who had been baptized as children. Anabaptists means ``those who baptize again.''

The man who succeeded in creating order out of this mess is the Frieslander Menno Simmons. Zealous, yet prudent, he succeeded in collecting the scattered members into congregations both in the Netherlands and Germany, and gave them instructions in church government and doctrine. From that time on they became more tractable, repudiated the excesses of their fanatical ancestors, and, after the name of their leader, called themselves Mennonites. Later they divided into numerous groups.

That the Baptists of our day do not look with pride on the historical beginning of their denomination is only human. It is nothing to be proud of. And yet it is a historical fact that the persecuted Anabaptists from Holland came to England, where they first made common cause with the dissenters (Independents, etc.), and only since the middle of the seventeenth centiry they have called themselves Baptists. By this name they wish to emphasize their belief in immersion as ``the only proper mode of administering Baptism,'' and that only to adults. Until recently it has been generally believed that immersion was practiced among them from the very beginning. Such is not the case however. Up to 1641 Baptism was administered in England by sprinkling or pouring, and no other method was set forth ``as the only correct way.''

In church organization the Baptists agree with the Congregationalists. There are Baptist churches, but there is no Baptist Church, except in anme; hence, no common confession. In 1677, a ``Confession of Faith'' was set up in London, and again in 1688, but this is not binding on all who call themselves Baptists. Their old belief concerning immersion as a necessary mode of Baptism is also undergoing a marked change. Many churches hold ``that the character and confession of the candidate gives sufficient validity to the ordinance, and receive baptized believers from whatever alliance they may come.'' (\emph{Bapt. Layman's Book}, p.~130.) Hence, Baptists ``are coming to have almost nothing to stand out for in separation from other denominations, except some matters of external form.'' (\emph{Bapt. Congr.}, 1911, p.~91.)

Like all Reformed, the Baptists are divided into numerous sects. ``Old-School Baptists,'' also called ``Primitive Baptists,'' are Calvinistic in doctrine. Some of the others tell their own story with their name: ``Free-Will Baptists,'' ``Anti-Mission Baptists,'' ``Six-Principle Baptists'' (Heb. 6, 1.2), ``Seventh-Day Baptists,'' ``Campbellites'' (or ``Disciples of Christ''), ``Christians,'' ``Weinbrennerians'' (``Church of God''), ``Dunkers,'' ``River Brethren,'' etc., etc.

Their home missionaries are as fanatical as Muenzer, and respect no denominational lines. The Lutherans especially are bothered by these zealots. We can truly say of them as the Baptist professor, McGlothlin, says of the Campbellites: ``They have continued their predatory habits of proselyting down to the present time.'' (\emph{Bapt Congr.}, 1911, p.~104.) Their errors are treated in the second part of this treatise.

\hypertarget{the-methodists}{%
\subsection{THE METHODISTS}\label{the-methodists}}

The Methodist Church is the youngest church-body among the larger Reformed denominations, dating from about 1739. In 1729, two brothers, John and Charles Wesley, were studying at Oxford University, and, becoming troubled about the welfare of their souls, commenced to study the Bible more earnestly, and to practise the Christian virtues.

On account of their strict methodical ways of performing these things a student, in derision, called them Methodists, a name which the Wesleys themselves shortly afterwards applied to their different ``societies,'' since organized into independent churches.

The Wesleys did not at first intend to found a new denomination. They only wished to inject more real Christian life and energy into the Anglican Churhc, of which they were ordained presbyters. But very soon the ``societies'' which they had organized for seeking and practising holiness thought that they could no longer worship together with the rest of the church-members, and therefore organized themselves into separate congregations. Though both John and Charles Wesley had a like share in starting the pietistic movement, yet John soon took the leading part, and is considered the real father of Methodism.

The Wesley brothers were zealous workers, it is true, but when they returned from America, in 1737, they confessed that they had labored to save others, but were not converted themselves. This they first became, so they confessed, when John had absorbed the truths proclaimed by Luther in his introduction to the Epistle to the Romans, and Charles, through reading Luther's exposition of the Epistle to the Galatians.

For a time the Wesleys were assisted by George Whitefield in their evangelistic efforts. But they soon parted company, as Whitefield was a strict Calvinist in the doctrine of predestination, and became the founder of the Calvinistic Methodists, or ``Huntingdon Connection.''

The confession of the Methodist Church, edited by John Wesley, is only the Thirty-nine Articles of the Anglican Church slightly modified and cut down to twenty-five. IN these we look in vain for specific Methodist doctrine and practises. These are inferred from the writings of John Wesley mainly. Characteristic practises and belifs of the various Methodist bodies are: revivals (camp-meetings, protracted meetings, etc.); condemnation of every use of alcohol and tobacco; a belief that conversion, to be genuine, must be an experience of which the believer knows the exact time, and experience brought about by an immediate action of the Holy Spirit, and which is not complete unless it has been \emph{felt} by the Christian; a belief that man may attain perfection (perfect sanctification) in this present life; the doctrine, held in common with all Reformed sects, that Baptism and the Lord's Supper are not means of grace, but only signs of God's grace. The Methodists wish to be styled evangelical, \emph{par excellence}, among the Reformed Churches, but there are certainly none more legalistic than they. The doctrine of Christian liberty is a sealed book to them.

In England the Methodist Church calls itself the Wesleyan Methodist Church, and dates its beginning from 1739, in which year John Wesley started the first congregation in Bristol, and dedicated ``The Old Foundry'' in London as a house of worship. In this country the first Methodist church was started in New York, 1766, by a Wesleyan local preacher from Ireland. Others followed and started ``societies'' in many places, so that in 1784 Dr.~Coke, whom Wesley consecrated bishop, met sixty preachers in Baltimore for the first American conference. This conference organized the Methodist Episcopal Church, and adopted the articles of faith edited by John Wesley.

The church government of the Methodist Episcopal Church is hierarchical to a high degree. The congregations have very little to say. Their ministers are appointed by the bishop, and after from two to five years are transferred to another charge by the same authority.

When the name Methodist Church is used, the official name Methodist Episcopal, is usually understood. But the number of Methodist connections is very great. Some insert ``Methodist'' in their corporate name, others omit this term. To enumerate a few: ``The Wesleyan Methodist New Connection''; ``The Primitive Methodist Connection''; ``Bible Christians''; ``Methodist Episcopal Church South''; ``The Protestant Methodist Church''; ``The Canada Wesleyan Methodist''; ``African Methodist Episcopal Church''; ``United Brethren''; ``The Albright Methodists'' (The Evangelical Association); ``Jumpers''; ``Howling Methodists''; and ``The Salvation Army.''

\hypertarget{unchristian-cults}{%
\section{Unchristian Cults}\label{unchristian-cults}}

How many sects there are in this country alone no one knows, not even the Director of the Census. Catholics would fain tell us that this state of affairs dates back only to the time of the Reformation. There may, at the present time, be more sects than formerly, but divisions in the Church have always existed, and the Pope of Rome has not only had his hands full in pacifying and judging between warring factions of monks and their so-called ``eminent teachers,'' but he has constantly had on hand a war of extermination against dissenters, who would not acquiesce in his teachings. The attitude of the pope towards the Modernists of our own day is only the old story over again. But their silence after his condemnation is the slence of the dead, -- and of the same value.

But there are sects and sects. Some who retain the fundamental thuths of Christianity are called Christian sects, for in them there is a possibility of acquiring saving truths, -- though the danger of not finding them is also very great. The most important of these we have discussed in the preceeding chapters. But where even fundamentals are denied, there no salvation is possible, and such sects are at best only clubs for entertainment, if not, indeed, wholly anti-Christian and gates of hell. Among the chief of these the following may be mentioned: --

\hypertarget{unitarians-and-universalists}{%
\subsection{UNITARIANS AND UNIVERSALISTS}\label{unitarians-and-universalists}}

The late Senator Hoar of Massachusetts used to ask this conundrum, ``What is the difference between Universalists and Unitarians?'' THe answer was, ``The Universalists believe that God is too good to damn them, and the Unitarians believe that they are too good to be damned.'' The characterization is so striking and true that little more need be said.

These churches accept the Bible as their guide in spiritual matters, but only so far as it accords with human reason. They do not believe in the Trinity, original sin, redemption, atonement, endless punishment, etc., but believe that all will be saved, or -- some of them -- that the wicked will be annihilated.

\hypertarget{the-adventists}{%
\subsection{THE ADVENTISTS}\label{the-adventists}}

It would not be proper to lay the heresy of the Advent Church at the door of the Baptists, and yet a Baptist clergyman is responsible for its existence.

William Miller, a licensed Baptist preacher, was neither an astronomer nor a scholarly theologian, but believed that he could foretell the day and year of the second coming of Christ. This was in the year 1833. People became hysterical, sold their belongings, and in ``ascension robes'' prepared for the event. The Day of Judgment was set for April 14, 1844. That day is long past, and yet the Adventists revere Miller as a great prophet. The day has since been set time and again by other would-be prophets, but, of course, with the same results. And yet the Adventists work with might and main to secure converts for their faith. THe Baptists have furnished them the mode of baptizing, and the Methodists camp-meeting ideas.

The larger body among them, the Seventh-Day Adventists, denies the Holy Trinity, teaches the mortality of the soul, the final destruction of the wicked, and, especially, that Saturday must be observed as the weekly day of rest.

On account of their persistent missionary work, in which they respect no denominational lines, and their juggling with the words ``Christ,'' ``faith,'' ``sacrifice,'' etc., a goodly number of simple-minded Christians are ensnared by them every year.

Adventism is represented by a great number of sects. Among the most active are, at present, the Russellites. The leader of the Russellites is ``Pastor'' (he has never been ordained) Charles T. Russell. The work of spreading the erroneous teaching of the ``Pastor'' is carried on under a confusing and misleading variety of titles, the best known of which is ``The International Bible Students' Association'' of Brooklyn. Other names are: ``Watch Tower Bible and Tracts Society,'' formerly ``Zion Watch Tower and Tract Society''; also ``The Laymen's Home Missionary Movement,'' in the interests of which are published \emph{Everybody's Paper} and \emph{the People's Pulpit}. Most important of all the publications of ``Pastor'' Russell are six large volumes, entitled \emph{Studies in Scripture}. IN order to even mislead the blind, books and articles are prepared in raised letters, under the title of ``The Gould Free Library for the Blind,'' South Boston, Mass. There are also issued what are called \emph{International Sunday-school Lessons}. It has been proved in court that ``Pastor'' Russell is totally ignorant either of Greek or Hebrew, while claiming to be a great Bible interpreter; that he has been ordained by no Church or society; that he has been divorced by his wife; that he is connected with various business corporations, all of which are under his absolute control, and which hold stock valued at millions of dollars. We append a brief summary of the false teachings assiduously being spread by Russell and his followers, the Millennial Dawn religion.

\begin{enumerate}
\def\labelenumi{\arabic{enumi}.}
\item
  Russellism denies the doctrine of the Trinity.
\item
  It denies that Jesus Christ was God before His incarnation.
\item
  It teaches that Christ was only a created spirit.
\item
  In incarnation He ceased to be a spirit, and became the second Adam.
\item
  As the second Adam He had only one nature.
\item
  His nature of humanity was annihilated on the cross.
\item
  He did not rise in the body in which he died.
\item
  The body in which He died may have been dissolved into gas.
\item
  The body in which He appeared after death was nothing more than a momentarily materialized appearance, which was finally dissolved
\item
  Jesus Christ is not now a man.
\item
  The ``Man Christ Jesus'' no longer exists.
\item
  Jesus Christ is now an invisible spirit-being.
\item
  He came to the world in 1874 as an invisible spirit-being.
\item
  The millennium will begin in 1914. (!)
\item
  All the dead out of Christ will be raised at that time.
\item
  All the unrighteous and wicked dead will be raised and made perfect and innocent like Adam before the Fall.
\item
  All the unrighteous and wicked dead will be given a second chance.
\item
  The more wicked they have been in this life, the more likely they will be, through the ``experience'' of sin, to accept the Gospel of the second chance.
\item
  Those who accept the second chance will have everlasting life.
\item
  Those who get everlasting life will sustain it by eating food.
\item
  Those who do not want to live forever will have the privilege of being asphyxiated in the lake of fire.
\item
  The assurance given to the wicked and sinful is that there is no suffering for sin.
\item
  Those who do not care for heaven need not be afraid of hell.
\item
  The finally impenitent are extinguished here, and annihilated.
\end{enumerate}

\hypertarget{quakers}{%
\subsection{QUAKERS}\label{quakers}}

``Quiet as a Quaker meeting'' is a byword. The reason for this is that George Fox, the founder of the sect, discarded the Bible as an insufficient message from God, and sought consolation in ``inner promptings, whisperings, and moving of the spirits.'' In their so-called services no Bible-text is expounded, but ``when the Spirit moves'' a man or woman, that person speaks; but until then there is silence. And when, after waiting a reasonable time, no one is ``moved,'' the congregation goes home.

Quakers also wish to be known as the ``Society of Friends.'' They hate ostentation, never remove the hat in saluting any one, and call everybody ``thou.'' They will neither take an oath nor serve in the army.

The theology of the Quakers may be somewhat more definite now than at the time of their founder, but not sufficiently so as to characterize them as Christians. Their founder's everlasting refrain ran like this: ``Not Scripture, but Spirit! Not Christ for us, but in us! Not churches (`tower buildings') and bells, not Sacraments and dogmas, but only the inner light which GOd ignites in the conscience of every one, and through which the SPirit of Christ regenerates and comforts man.'' This sounds plausible, but as long as they deny the atonement, Christ will not send the regerating Spirit. But without regeneration no one shall see God. (John 3, 3.) The Quakers are not the only ones relying on the ``Spirit'' and the ``inner light.'' But they seem to rely on a ``Spirit'' and ``Light'' which is not always to be relied on. Thus we find them divided into ``orthodox,'' ``dry,'' ``wet,'' ``Hicksite,'' ``fighting,'' ``progressive,'' ``primitive,'' ``Baptist,'' ``Christian,'' and ``Keithian'' Quakers.

Quakerism is only a type of a large class of cults, both ancient and modern, the adherents of which, like the rich man (Luke 16, 27), find, as they think, the Bible insufficient to the needs of mankind. But whatever their achievements may be in our eyes, the Holy Spirit says of them that ``other foundation can no man lay than that is laid, which is Jesus Christ.'' (1 Cor. 3, 11.)

\hypertarget{the-mormons}{%
\subsection{THE MORMONS}\label{the-mormons}}

This old world of ours has seen many queer and brazen humbugs, but not till Mary Baker-Eddy proclaimed Christian Science has anything appeared quite equal to Mormonism, or, as its adherents wish to be known, The Church of Jesus Christ of Latter-Day Saints, founded by Joseph Smith.

ne Solomon Spaulding (died 1818) amused himself, after retiring from the ministry, by writing a book, in Biblical style, purporting to be the history of the peopling of of America by the ten lost tribes of Israel.

This manuscript Joseph Smith secured, and, after altering a little here and there (without, however, improving its style, for he was very poorly educated) he published it in 1830 under the name of \emph{The Book of Mormon}, and proclaimed it to be of equal authority with the Bible.

But the real author's name does not appear on its pages. Instead, it purports to be a revelation. An angel, so he said, pointed out to Joseph Smith that ``on the west side of a hill, not far from the top, about four miles from Palmyra, N.Y., near the road to Manchester,'' he would find plates of gold inscribed with hieroglyphs in ``reformed Egyptian,'' and a pair of eye-glasses called ``Urim and Thummim.'' And Joseph found them, so he says, and by sitting behind a curtain, dictating to a confederate on the other side, he gave to a benighted world the greatest light in its history: \emph{The Book of Mormon}. What became of the plates is still a mystey. Forsooth, certain persons bear witness, on a front page in the book, that they have seen them, but it is only the testimony of ill-reputed people in favor of one of the same kind.

The plates are said to have been hidden in the hill about A. D. 420. Yet the inscriptions mention Calvinism, Universalism, Methodism, Millenarianism, and Roman Catholicism!

Though polygamy is one of the main tenets of Mormonism, still it is condemned in \emph{The Book of Mormon}. It was an after-thought, and was revealed to the churhc later!

To those who calmly compare Mormonism with Christianity there is not much danger. But those who listen only to the glib-tongued ``missionaries,'' who conceal more than they divulge, Mormonism is as alluring a scheme as the devil has yet concocted. Mormonism is by its missionaries pictured as a paradise on earth; hence the rule of the ignorant and the sensual to Salt Lake City and other Mormon centers.

\hypertarget{christian-science}{%
\subsection{CHRISTIAN SCIENCE}\label{christian-science}}

Christian Science is one of the latest cults to bid for the favor of those whom Christ characterizes as ``an evil and adulterous generation seeking after a sign.'' (Matt. 12, 39.)

Mary Baker-Glover-Eddy, divorcee and widow, a small, frail, hysterical, and nervous woman, consulted during an illness, in 1862, a noted mesmerist by the name of P. P. Quimby. From that time on she commenced to teach, and write on, mental healing. People who knew Quimby have testified that she, at least to begin with, copied his teachings verbatim, but without giving him credit.

However, Quimby would probably not make the charge of plagiarism against Mrs.~Eddy if he were to read her \emph{Science and Health, with Key to the Scriptures}, the Scientists' Bible, as it reads to-day. No man in his right mind would wish to be accredited with even having started the old lady on her Bible-producing career, as the twaddle she wrote can only emanate from a brain disordered.

That seemingly sane people can revere her as a god, and freely spend millions in furthering the cause of the cult, finds its explanation only in one cause, the one which St.~Paul mentions when he says: ``Because they receive not the love of truth, that they might be saved. And for this cause God shall send them strong delusion, that they should believe a liel; that they all might be damned who believed not the truth, but had pleasure in unrighteousness.'' (2 Thess. 2, 10-12).

Christian Science -- officially, ``The Church of Christ, Scientist'' -- has rightly been characterized as ``neither Christian nor scientific.'' In corroboration hereof the following from \emph{Science and Health} will suffice: --

~~``Got is all, and therefore matter, sickness, sin, and death have no reality; they are illusions of mortal mind.''

~~``All disease is belief.''

~~``All is mind; matter is but a belief and error.''

~~``Matter and mortal body are illusions of human belief.''

~~``Man is coexistent with God.''

~~``Man never fell into sin or death; he is forever happy, harmonious, and immortal.''

~~``God never forgives sin.''

~~``Jesus never ransomed man by paying the debt that sin incurs.''

~~``Man has neither birth nor death.''

~~``Destroy sin, sickness, and death in the mind, and they are gone forever.''

The fundamental propositions of Christian Science are stated thus: --

\begin{enumerate}
\def\labelenumi{\arabic{enumi})}
\tightlist
\item
  God is all in all; 2) God is good, Good is mind; 3) God, Spirit, being all, nothing is matter; 4) Life, God, omnipotent GOod, deny death, evil, sin disease; -- disease, sin, evil, death, deny God, omnipotent Good, Life.
\end{enumerate}

To believe such tommmy-rot together with: ``No matter in mind and no mind in matter; no matterin life and no life in matter; no matter in good and no good in matter,'' indicates that there is something the ``matter'' with the minds of Mrs.~Eddy and her followers.

\hypertarget{spiritualism}{%
\subsection{SPIRITUALISM}\label{spiritualism}}

Spiritualism (Spiritism) is an unchristian cult based on a real or pretended intercourse with the souls of the dead. For the greater part, Spiritistic mediums are tricksters and frauds. In so far as they may commune with the spirits of the departed, they fall under the condemnation of the word of God: ``There shall not be found among you any one\ldots that useth divination, or an observer of times, or an enchanter, or a witch, or a charmer, or a consulter with familiar spirits, or a wizard, or a necromancer.'' (Deut. 18, 10. 11.)

Since 1848, modern Spiritualism has had adherents in the United States. Many are enticed by its trickery and love for the unknown. The Spiritualists do advise their uninitiated to read the Bible, thereby gaining the victims' confidence. These advisors then bring the reader to question certain parts of the Bible. Satan's procedure while he tempted Jesus was similar to the Spiritualists' use of the Bible now. Spiritualism denies that our Lord Jesus is divine; it denies the existence of the devil, demons, and angels. Exponents of Spiritualism say the following about our Bible: ``To assert that it is a holy and divine book, that God inspired the writers to make known His divine will, is a gross outrage on, and misleading, the public.'' (\emph{Outlines of Spiritualism for the Young}.) ``The New Testament is made up of traditions and theological speculations by unknown persons.'' (\emph{Outlines}, p.~13. 14.)

In the Spiritistic book \emph{Whatever Is, Is Right} we find the following information: ``What is evil? Evil does not exist; evil is good. What is a lie? A lie is the truth intrinsically; it holds a lawful place in creation; it is a necessity. What is vice? Vice, and virtue, too, are beautiful in the eyes of the soul. What is murder? Murder is good. Murder is a perfectly natural act.''

Spiritualism leads to infidelity and immorality. According to Mrs.~Woodhull, elected three years in succession as president of the Spiritist societies in America, it is ``the sublime mission of Spiritism to deliver humanity from the thraldom of marriage.'' Dr.~Day, of Montville, Conn., writes: ``It is a fact, and no honest, intelligent Spiritualist can deny its truth, that nine-tenths of modern Spiritualists are, either openly or secretly (as far as they dare), practically \emph{Free Lovers}, in the broadest sense of the word. I am familiar with many of the most prominent leaders, teachers, and mediums of Spiritualism, who are secret agents of Free Love secret circles.'' The same Dr.~Day quotes ``a prominent author and teacher of Spiritualism'' as saying: ``Free Love is the central doctrine of Spiritualism. The new social order is a \emph{social harmony based upon passional attractions}, or the harmony of the varied and developed passional or impulsive nature of man. Attraction is our only law.'' According to Spiritist doctrine, marriate is not a divine institution, in which in reality God joins together one man and one woman, but it is based on the laws of human nature, and is the result from ``natural and spiritual affinities.'' The two parties united are not so much united into one flesh as virtually into one spirit and one soul. Divorces are to be freely granted when desired by both, or even by only one party. ``The marriage vow imposes no obligation on the Spiritualistic husband.'' (\emph{T. L. Harris}.)

Modern Spiritualism emphatically denies the fall of man through the temptation of the devil. This denial is publicly made by the author of \emph{Outlines}. Others deny the existence of the devil; and still another makes a statement, so blasphemous that we can hardly bear repeating it. ``Whom then,'' says he, ``can we believe, God or Satan? The facts justify us in believing Satan. It was not the devil, but God, who made the mistake in the Garden of Eden. It was God, and not the devil, who was the murderer from the beginning.'' This makes any one who has still some moral feeling shudder, and this ought to be enough for any sober-minded man or woman to shun Spiritualistic company.

Mr.~Harrison D. Barrett, president of the National Spiritualists' Association, says that Spiritualism ``steadfastly refused to accept any religious postulates on faith, and at the outset rejected all creeds and dogmatic assumptions of theology.'' This is plain enough. Spiritism rejects the creed of Christianity, and characterizes the saving doctrines of Scriptures as ``dogmatic assumptions.'' By the testimony of its leading exponents, Spiritism is a Christless cult, opposed alike to Christian doctrine and morals. It is one of the false teachings foretold by St.~Paul, when he writes to Timothy: ``Now the Spirit speaketh expressly that in the latter times some shall depart from the faith, giving heed to seducing spirits and doctrines of devils; speaking lies in hypocrisy; having their conscience seared with a hot iron; forbidding to marry\ldots.'' (1 Tim. 4, 1-3).

\hypertarget{theosophy}{%
\subsection{THEOSOPHY}\label{theosophy}}

This blasphemous cult is antagonistic to Christianity in three main points.

First, it is pantheistic. Its founder, Madame Blavatsky, says: ``We believe in a universal divine principle, the root All.'' Theosophy rejects a personal God. It believes that God is made up of everything. Horse and star and tree and man are parts of the Theosophists' god.

Secondly, it teaches reincarnation. It says we have three souls, an animal soul, a human soul, and a spiritual soul. The animal soul becomes, after a while, a wandering thing, passing into other human beings. The soul keeps wandering on and on, and may have innumerable different forms. It is simply the old Hindu doctrine of the transmigration of souls, slightly refined to suit European and American tastes. In a country where lizards and cows are not worshipped, it would hardly do to try to proselyte people to the faith that they and their children may be reborn as lizards, cats, or cows! Hence, Theosophy confines reincarnation to the human race, for which merciful limitation I presume we ought all of us simple-minded Christians to be most devoutly thankful!

Their third main point of antagonism to the Christian Religion is the doctrine of the so-called ``Karma,'' or ``The doctrine of consequences.'' It was the doctrine of Buddha and Bob Ingersoll. It is the old heathen fatalism in its barest form. According to the ``Karma,'' you are under the merciless law of cause and effect, to the extent that it is useless to repent; for there is no one to forgive. It is all a question of consequences, -- that's all. Hence there is no place for prayer, repentance, and forgiveness in the Theosophic system.

In Madame Blavatsky's \emph{Key to Theosophy}, a kind of catechism, written evidently for simple-minded Christian people, she makes use of the following dialog: ``Do you believe in God?'' Answer: ``That depends on what you mean by the term.'' ``I mean,'' says the inquirer, ``the God of the Christians, the Father of Jesus, and the Creator, -- the Biblical God of Moses, in short.'' Answer: ``In such a God we do not believe.'' According to the same text-book, Theosophists profess to believe ``in a Universal Divine Principle.'' (p.61.) Other quotations from the \emph{Key}, in which the unchristian character of Theosophy is revealed, are the following: Question: ``Do you believe in prayer, and do you ever pray?'' Answer: ``We do not. We act instead of talking.'' This is consistency, since prayer presupposes a personal and living God. Question: ``Then you also reject resurrection in the flesh?'' Answer: ``Most decidedly we do.''

Theosophy denies that there is ``eternal reward or eternal punishment.'' (p.~108.) It rejects the vicarious atonement of Jesus and the remission of sin. (p.~196.) It is an anti-Christian cult.

Dr.~Talmage said of this sect: ``The most wonderful achievement of the Theosophists is that they keep out of the insane asylum.''

\hypertarget{part-ii.-doctrines-of-the-church}{%
\chapter*{PART II. Doctrines of the Church}\label{part-ii.-doctrines-of-the-church}}
\addcontentsline{toc}{chapter}{PART II. Doctrines of the Church}

\hypertarget{concerning-the-truth-of-the-christian-religion}{%
\section{Concerning the Truth of the Christian Religion}\label{concerning-the-truth-of-the-christian-religion}}

\begin{center}
\textsc{Bible Doctrine.}
\end{center}

Of the many religions in the world, ancient as well as modern, the religion of the Bible is the only true one.

Acts 17, 29. 30: ``Forasmuch, then, as we are the offspring of God, we ought not to think that the Godhead is like unto gold, or silver, or stone, graven by art and man's device. And the times of this ignorance God winked at, but now commandeth all men everywhere to repent.''

John 17, 3: ``And this is life eternal, that they might know Thee the only true God, and Jesus Christ, whom Thou hast sent.'' (John 3, 18; 1 Cor. 8, 4; Ex. 20, 3-6.)

\begin{center}
\textsl{Testimony:--}
\end{center}

``Whosoever will be saved, before all things it is necessary that he hold the true Christian faith. Which faith except every one do keep whole and undefiled, without doubt he shall perish everlastingly.'' -- \emph{Athanasian Creed}.

\begin{center}
\textsc{Error.}
\end{center}

\begin{enumerate}
\def\labelenumi{\arabic{enumi}.}
\tightlist
\item
  Carnal mind is in enmity against God. (Rom. 8, 7.) Hence man's unwillingness to be governed by Him. He tries to forgetHim, deny Him; but all to no purpose,-- except, perhaps, for a time. But the devil is always trying some method whereby he may rob man of the Word of God, so that he shall not believe and be saved. (Luke 8, 12.) And one way of accomplishing this is to make him believe that one religion is as good as another, or that Christians ``will not be alone about inheriting heaven.''
\end{enumerate}

The Swiss reformer Zwingli believed ``that all the ancient noble heathens who walked uprightly according to the light they had were also saved.'' (\emph{Expos}. \emph{Fidei}, II, 559.)

The Quaker Barclay, in his \emph{Apology} (10, 2), voices the same thought as regards heathen, Turks, and Jews at the present time.

\begin{center}
\textsl{But God says:--}
\end{center}

Acts 4, 12: ``Neither is there salvation in any other; for there is none other name under heaven given among men whereby we must be saved.'' (See Gospel of St.~John.)

1 Cor. 1, 21: ``For after that, in the wisdom of God, the world by wisdom knew not God, it pleased God by the foolishness of preaching to save them that believe.'' (2 Thess. 1, 7-10: ``punishment for all who do not know and obey the Gospel of our Lord Jesus Christ.'')

2 Thess. 2, 10-12: ``Because they received not the love of the truth that they might be saved, \ldots{} God shall send them strong delusions that they should believe a lie, that they all might be damned who believed not the truth.''

John 14, 6: ``I am the Way, the Truth, and the Life; no man cometh unto the Father but by Me.''

Christ warned against being an ``unbeliever, a heathen and a publican.'' (Matt. 18, 17; Mark 16, 17.)

\begin{enumerate}
\def\labelenumi{\arabic{enumi}.}
\setcounter{enumi}{1}
\tightlist
\item
  A large class of unbelievers call themselves agnostics. This sounds better than atheists and freethinkers, but they are in reality their triplet brother. The ``believ'' of the agnostic is ``We do not know whether there is a God.'' (``The problem of existence has not been solved, and it is insoluble'': (Thomas Huxley.) This dictum has been applied, not only to human existence, but to God Himself, as well as to His revelation.
\end{enumerate}

\begin{center}
\textsl{But God says:--}
\end{center}

John 1, 14: ``And the Word was made flesh, and dwelt among us; and we beheld His glory, the glory as of the Only-begotten of the Father, full of grace and truth.''

1 John 1, 1. 3: ``That which was from the beginning, which we have seen which we have heard, \ldots{} declare we unto you.''

2 Pet. 1, 16: ``For we have not followed cunningly devised fables when we made known unto you the power and coming of our Lord Jesus Christ, but we were eye-witnesses of His majesty.'' (See vv. 17. 18.)

Rom. 1, 19: ``even the heathen acknowledge the existence of God.

\emph{What Is an Atheist?} -- A snare to the simple-minded Christian is the oft repeated assertion that so and so was no atheist, because he had confessed his belief in a personal God or, perhaps, in ``a Higher Power.'' It is true, the words of the Psalmist (14, 1), ``The fool saith in his heart, There is no God,'' describe more nearly the person called ``atheist.'' But all who do not confess the true God are properly to be classed as having no God at all. Christ says: ``He that honoreth not the Son honoreth not the Father.'' (John 5, 23.) ``Whosoever shall deny Me before men, him will I also deny before My Father which is in heaven.'' (Matt. 10, 33.) And the Holy Spirit testifies: ``Whosoever denieth the Son, the same hath not the Father.'' (1 John 2, 23. Compare also 2 John, v. 9.)

\emph{Lodge Religion}. -- Under this head must be ranged all lodges which require a ``belief in a Supreme Being'' from their members, but who do not define this ``Being'' as the Triune God, Father, Son, and Holy Ghost, and studiously avoid using the name of Jesus in their so-called prayers. Silently (and here intentionally) to omit a confession of the Savior is to deny Him, and be classed as heathen. (Matt. 18, 17.) And Christ explicitly teaches that He is the Way, the Life, and the Truth, and that no man cometh unto the Father but by Him. (John 14, 6.)

\emph{Evolution}. -- The notion is also prevalent that Christianity, revealed religion, ``is subject to the laws of evolution.'' Aside from the fact that ``the laws of evolution'' are only imaginings of depraved human nature, which fain would escape just punishment at the hands of God, ``progressive Christianity'' is simply an absurdity. What of a Christianity in which the life, teachings, and deeds of a Christ Jesus, little by little, should be superseded by -- what? Can you think of a moral law which could supersede the Ten Commandments? Of a just God more righteous than Jehovah? more loving than Jesus? more merciful than He who sent His only-begotten Son as the Savior for the whole world? more merciful than He who embraces the Prodigal (Luke 15, 20), pardons the adulteress (John 8, 11), promises heaven to the thief on the cross (Luke 29, 43), and prayed even for His enemies?

No doubt, it is considered the height of wisdom at the present time to assert that ``the law of evolution'' also underlies the Christian religion. But what does history tell us concerning those who have come under its spell? Heathenism is nothing but the result of a trial to improve on the revelation and government of God. The Babylonian captivity of the Jews and the final dispersement of them over all the earth is the outcome of a similar experiment. The early Christian Churches in Asia and elsewhere tried their hands at the same thing, and were rewarded by -- Mohammedanism. And what is popery, Jesuitism, rationalism but the outcome of efforts to apply ``the law of evolution'' to religion! Evolution, indeed, but \emph{always} an evolution downward.

Though the theory on which Darwinian evolution rests has been discarded as insufficient by conscientious scientists, yet the reechoing of it is still heard in some of our higher institutions of learning, not to mention our normal schools, where our common-school teachers are trained. Hence we often read of this or that ``professor,'' in our daily papers, who adjudges the remains of some prehistoric (?) animal as being so or so many millions of years old. These opinions finally find their way into text-books and are henceforth promulgated as ``gospel truths.'' But what reliance may be put upon them the following well-supported facts will show.

\begin{enumerate}
\def\labelenumi{\arabic{enumi}.}
\item
  The Neanderthal Skull, has, since 1856, been offered as one of the chief proofs for the evolution of man from a lower animal, and has been considered about 300,000 years old. Dr.~Meyer, of Bonn, proved that it is the skull of a Russian Cossack killed in 1814.
\item
  The Calaveras Skull, in the California State Museum, was placed in a mine-shaft, 150 feet deep, as a practical joke.
\item
  The ``Colorado Specimen'' was pronounced by a Columbia College ``scientist'' to be ``the missing link,'' one million and a half years old. Cowboys since proved the bones to be that of a pet monkey which they had buried there.
\item
  The Croatian Skeletons (Austria) are the bones of degenerate human beings, -- nothing more.
\item
  The Pithecanthropus Erectur has for some years (since 1891) been the most popular relic with evolutionists. The collection consisted of a skull above the eyes, a leg-bone, and two teeth, -- found in Java at separate places. The age of this ``missing link'' was guessed by various scientists to be from one hundred thousand to one million years! According to measurements the skull is that of an idiot. (See \emph{The Other Side of Evolution}, Alex. Patterson, D. D. Winona Publ. Co., Chicago, 1903.)
\end{enumerate}

\hypertarget{the-holy-scriptures}{%
\section{The Holy Scriptures}\label{the-holy-scriptures}}

\begin{center}
\textbf{A.  The Bible Is God's Revelation to Mankind}

\textsc{Bible Doctrine.}
\end{center}

The Bible, comprising the canonical books of the Old and the New Testament, is God's unerring and only revelation to mankind.

Luke 16, 29: ``They have Moses and the prophets; let them hear them.''

1 Thess. 2, 13: ``When ye received the Word of God which ye heard of us, ye received it not as the word of men, but, \emph{as it is in truth}, the Word of God.''

John 17, 17: ``Sanctify them through Thy truth; Thy Word is truth.''

\begin{center}
\textsl{Lutheran Testimony}
\end{center}

``We believe, teach, and confess that the only rule and standard according to which at once all dogmas and teachers should be esteemed and jusged are nothing else than the prophetic and apostolic Scriptures of the Old and the New Testament, as it is written: `Thy Word is a lamp unto my feet and a light unto my path' (Ps. 119, 105), and St.~Paul: `Though an angel from heaven preach any other gospel unto you, let him be accursed' (Gal. 1, 8). (\emph{Form. Con.}, Epit., Introd., § 1.)

\begin{center}
\textsc{Error.}
\end{center}

\begin{enumerate}
\def\labelenumi{\arabic{enumi}.}
\tightlist
\item
  The evolutionists one and all discredit the narrative in Genesis, first chapter, concerning creation. Of the Bible, as revelation, Haeckel says: ``It is the invention of man's imagination. The so-called truth which believers find in it is a human speculation, and the `childlike' faith in this irrational revelation is only superstition.'' (\emph{The Riddle of the Universe}, 123.)
\end{enumerate}

\begin{center}
\textsl{But God says:--}
\end{center}

Rom. 3, 1. 2: ``What advantage, then hat the Jew? Much every way: chiefly, because that unto them were commiteed \emph{the oracles of God}.''

Heb. 1, 1. 2: ``God, who at sundry times and in diverse manners spake in time past unto the fathers by the prophets, \emph{hath in these last days spoken unto us} by His Son.''

1 Pet. 1, 16: ``For we have \emph{not followed cunningly devised fables} when we made known unto you the power and coming of our Lord Jesus Christ, but were eye-witnesses of His majesty.''

\begin{enumerate}
\def\labelenumi{\arabic{enumi}.}
\setcounter{enumi}{1}
\tightlist
\item
  The so-called higher critics fain would find all kinds of flaws and faults in the Bible: Creation is a myth; the patriarchs never lived, at least not at the time states; Moses, and most of the other authors, never wrote the books ascribed to them; the contents are unreliable; the so-called prophecies are post-event descriptions, etc., etc.
\end{enumerate}

\begin{center}
\textsl{But God says:--}
\end{center}

1 Tim. 4, 1: ``Now \emph{the Spirit speaketh} expressly that in the latter times some shall depart from the faith, giving heed to \emph{seducing spirits} and doctrines of devils.''

John 1, 1-3: ``In the beginning was the Word, and the Word was with God, and the Word was God. All things were made by Him, and without Him was not anything made that was made.''

Luke 20, 37. 38: ``The Lord is the God of Abraham and the God of Isaac and the God of Jacob; for He is not a God of the dead, but of the living.''

Luke 24, 27: ``And \emph{beginning at Moses} and all the prophets, He expounded unto them in all the Scriptures the things concerning Himself.''

John 6, 63: ``The words that I speak unto you, they are spirit, and they are life.''

Matt. 24, 35: ``Heaven and earth shall pass away, but My words shall not pass away.''

John 12, 48: ``He that rejecteth Me, and receiveth not My words, hath one that judgeth him; the word that I have spoken, the same shall judge him in the last day.''

John 10, 35: ``The Scripture cannot be broken.''

\textsc{The Apocryphal Books.} -- Some editions of the Bible contain a number of books called Apocrypha (hidden). The name implies that the authors are unknown, and their writings of doubtful value.

The Apocrypha of the Old Testament comprise some fourteen books and those of the New Testament about twenty-five. The latter have never appeared in any Protestant Bible.

In 1546, the Catholic Church decreed, through the Council of Trent, that the old Testament apocryphal books were to be included among the canonical (rule-giving) books of the Bible. This was done in order to gain certain proof-texts for the doctrines of purgatory, prayers for the dead, and for strengthening the position of the pope generally. However, no council can decree what shall, or shall not, be the Word of God. The canonicity of the Bible is deduced from a multitude of proofs peculiarly its own. The following, among many, may suffice here. 1) The declaration of the different writers that God has spoken through them. 2) The loftiness and sublimity of the Bible-text in comparison with that of secular books. 3) The united testimony of the Church concerning both the Old and the New Testament. 4) The repeated references of Christ and the apostles to numberless statements in the canonical books, but silence as to the apocryphal. 5) The sufficiency of the canonical Scriptures as declared by Christ Himself. (Luke 16, 31; John 5, 39; 10, 35.)

\begin{center}
\textbf{B.  The Bible is Given by Inspiration of God.}

\textsc{Bible Doctrine.}
\end{center}

The contents of the Bible were revealed to the sacred writers by inspiration of the Holy Spirit

2 Tim. 3, 16: ``All Scripture is given \emph{by inspiration of God}, and is profitable for doctrine, for reproof, for correction, for instruction in righteousness.''

2 Pet. 1, 21: ``For the prophecy came not in old time by the will of man, but holy men of God \emph{spake as they were moved by the Holy Ghost}.''

\begin{center}
\textsl{Testimony of the Creed:--}
\end{center}

``And I believe in the Holy Ghost, the Lord and Giver of life; \ldots{} who spake by the prophets.'' -- \emph{Nicene Creed}, § 7.

\begin{center}
\textsc{Error.}
\end{center}

The great majority of modern exegetes, especially the so-called higher critics, maintain that ``science'' forbids the acceptance of an inspiration different from that of a Shakespeare or a Milton. And even if the different authors were under the influence of a higher power, they were children of their times, and hence their opinions were biased, and they were subject to errors in judgment, etc.

\begin{center}
\textsl{But God says:--}
\end{center}

1 Cor. 2, 12. 13: ``NOw we have received, not the spirit of the world, but the Spirit which is of God, that we might known the things that are freely given to us by God. Which things also we speak, not in the \emph{words} which man's wisdom teacheth, but \emph{which the Holy Ghost teacheth}, comparing spiritual things with spiritual.''

1 Cor. 2, 7. 10: ``But we speak the wisdom of God in a mystery, even the hidden wisdom, which God ordained before the world unto our glory; which none of the princes of this world knew. But \emph{God hath revealed them unto us} by His Spirit.''

Matt. 10, 20: ``For it is not ye that speak, but the Spirit of your Father which speaketh in you.'' (2 Sam. 23, 2. 3.)

1 Cor. 1, 23: ``But we preach Christ crucified, unto the Jews a stumbling-block and unto the Greeks foolishness.''

Gal. 1, 20: ``Now the things which I write unto you, behold before God, I lie not.'' (1, 8. 9.)

\textsc{Note.} -- A great many theologians at the present time have given up the belief in verbal inspiration, and contend that God only gave the authors an ``impulse'' to write, and what to write about, or, at best, that the ``concept,'' or ``idea,'' was inspired. But this, together with the assertion that the different authors have been mistaken in several things, is simply destroying the foundation of all faith and hope. If God has not told us in explicit words what His will is, the hope of the Christian is no better than that of a Rationalist.

But a few passages from Scripture will show how the Bible stands on the question.

Ex. 4, 12: ``Now, therefore, go, and I will be with thy mouth, and teach thee what thou shalt speak.''

Ex. 34, 27: ``And the Lord said unto Moses, Write thou these words; for after the tenor of these words I have made a covenant with thee and with Israel.''

2 Sam. 23, 2: ``The Spirit of God spake by me, and His Word was upon my tongue.'' (Of this Christ says in Mark 12, 36: ``David himself said \emph{by the Holy Ghost}.'')

Acts 2, 4: ``And they were filled with the Holy Spirit, and began to speak with other tongues as the Spirit gave them utterance.''

Mark 13, 11: ``For it is not ye that speak, but the Holy Ghost.''

In almost innumerable places we read: ``Thus saith the Lord''; ``The Lord said''' ``The word of the Lord,'' and the like, especially numerous in the writings of the prophets. So also the message of the apostles:--

1 Cor. 2, 13: ``Which things also we speak, not in words which man's wisdom teacheth, but which the Holy Ghost teacheth.''

1 Thess. 2, 13: ``When ye received the Word of God which ye heard of us, ye accepted it not as the word of men, but, as it is in truth, the Word of God.''

\begin{center}
\textbf{C.  The Interpretation of the Bible.}

\textsc{Bible Doctrine.}
\end{center}

The meaning of the words of Scripture must be sought in the Scriptures themselves, not outside of them.

2 Pet. 4, 11: ``If any man speak, let him speak as the oracles of God.''

\begin{center}
\textsl{Lutheran Testimony:--}
\end{center}

``We have, however, another rule, \emph{viz}.: That the Word of God should frame articles of faith; otherwise no one, not even an angel.'' -- \emph{Smalcald Articles}, II, 2, 15.

\begin{center}
\textsc{Error.}
\end{center}

Zwingli and Calvin both declared that human reason must be the judge of Holy Scripture. Zwingli gives this counsel: ``Philosophical argumentation through conclusions of reasoning must not be neglected.'' -- \emph{Expos. Chr. Faith}.

Calvin says: ``Can you prove through reasoning that this contains nothing unreasonable? Yes, if it be conceded that the Lord has instituted nothing that is contrary to reason.'' -- \emph{Geneva Catechism}.

This is the fundamental error of all sects, ancient and modern. And the Reformed Churches, one and all, carry the image of their originators.

\begin{center}
\textsl{But God says:--}
\end{center}

Is. 8, 20: ``To the Law and to the testimony: if they speak not according to this Word, it is because there is no light in them.''

2 Cor. 10, 5: ``We bring into captivity every thought to the obedience of Christ.''

1 Cor. 2, 14: ``But the natural man receiveth not the things of the Spirit of God; for they are foolishness unto him; neither can he know them, because they are spiritually discerned.''

Rom. 9, 20: ``O man, who art thou that repliest against God?''

\textsc{Note.} -- Under the topic ``Means of Grace'' the doctrine of Holy Scripture will be further discussed.

\hypertarget{original-sin}{%
\section{Original Sin}\label{original-sin}}

\begin{center}
\textsc{Bible Doctrine and Lutheran Testimony}
\end{center}

``All men, after Adam's fall, begotten after the common course of nature, are born with sin, that is, without the fear of God, without trusting Him, and with fleshly appetites; at that, this disease, or original fault, is truly sin, condemning and bringing eternal death now also upon all that are not born again by the Holy Spirit.'' -- \emph{Augsb. Conf.}, Art. II.)

Gen.~8, 21: ``The imagination of man's heart is evil from his youth.'' (6, 5.)

Eph. 2, 1: ``Ye were dead in trespasses and sin.''

Rom. 3, 22. 23: ``There is no difference; for all have sinned, and come short of the glory of God.''

Eph. 2, 3: ``We were by nature the children of wrath.''

Ps. 51, 5: ``I was shapen in iniquity, and in sin did my mother conceive me.''

Rom. 5, 12: ``By one man sin entered into the world, and death by sin; and so death passed upon all men, for that all have sinned.''

\begin{center}
\textsc{Error.}
\end{center}

The Reformed Churches do not express themselves with sufficient exactness in their definition of Original Sin. The \emph{Methodists}, for instance, say in their \emph{Articles of Religion}: ``Man is very far gone from original righteousness.'' (Art. 7.) To this a Methodist writer makes the remark that ``how far is not stated. Obviously the meaning is that the image of God is not wholly destroyed.'' The practise in the Methodist Church corroborates this assertion. The doctrine of Original Sin is now very rarely taught from Reformed pulpits, and has all but disappeared from the sectarian press.

\begin{center}
\textsl{But God says:--}
\end{center}

Rom. 7, 18: ``For I know that in me, that is, in my flesh, dwelleth no good thing.''

Rom. 8, 7: ``The carnal mind is enmity against God; for it is not subject to the Law of God, neither, indeed, can be.'' (1 Cor. 2, 14; John 3, 6.)

Ps. 53, 3: ``The wicked are estranged from before birth; they go astray as soon as they be born, speaking lies.'' (Is. 48, 8.)

\hypertarget{redemption}{%
\section{Redemption}\label{redemption}}

\begin{center}
\textbf{A.  Redemption Is Through Christ.}

\textsc{Bible Doctrine.}
\end{center}

Gal. 3, 13: ``Christ has redeemed us from the curse of the Law, being made a curse for us.''

Matt. 5, 17: ``Think not that I am come to destroy the Law or the prophets; I am not come to destroy, but to fulfill.'' (3, 15.)

Rom. 8, 3. 4: ``For what the Law could not do, in that it was weak through the flesh, God, sending His own Son in the likeness of sinful flesh, and for sin, condemned sin in the flesh, that the righteousness of the Law might be fulfilled in us, who walk not after the flesh, but after the Spirit.''

\begin{center}
\textsl{Lutheran Testimony:--}
\end{center}

``For since Christ is not alone man, but God and man in one undivided Person, He was as little subject to the Law, because He is the Lord of the Law, as, in His own Person, to suffering and death. Therefore His obedience, not only in suffering and dying, but also that He in our stead was voluntarily subject to the Law, and fulfilled it by His obedience, is imputed to us for righteousness, so that, on account of this complete obedience, which by deed and by suffering, and in life and in death, He rendered His heavenly Father for us, God forgives our sins, regards us godly and righteous, and eternally saves us.'' -- \emph{Form. Conc.}, III, § 15.

\begin{center}
\textsc{Error.}
\end{center}

The \emph{Moravians} (Brethren) and \emph{Methodists} speak in their catechisms, under this head, of Christ, only as a \emph{type}, or \emph{example}. ``How is the life of Christ on earth to be considered? His life on earth was perfectly holy, and is therefore an example for us to follow.'' (\emph{Mor. Cat.}, Qu. 48.) -- ``What does the life of Christ represent? A prototype of perfect goodness and holiness.'' -- \emph{Meth. Cat.}, No.~3, p.~23.

\textsc{Note.--} Present-day Christianity, among the sects and enthusiasts, is all of this variety. ``Be good,'' ``Follow in His steps,'' are slogans heard on every hand. Forsooth, true Christians also strive to be good, and to follow in the footprints of the Savior. But these things do not \emph{make}, but only \emph{adorn} a Christian. We are saved, not by following ``in His steps,'' but by accepting His redemption.

\begin{center}
\textsl{But God says:--}
\end{center}

Gal. 4, 4. 5: ``But when the fulness of the time was come, God sent forth His Son, amde of a woman, made under the Law, to redeem them that were under the Law, that we might receive the adoption of sons.''

Rom. 10, 4: ``For Christ is the end of the Law for righteousness to every one that believeth.'' (Phil. 2, 7. 8)

\begin{center}
\textbf{B.  Redemption Is Universal}

\textsc{Bible Doctrine.}
\end{center}

Christ has through His suffering and death redeemed the whole world.

Rom. 5, 19: ``For as by one man's disobedience many were made sinners, so by the obedience of One shall \emph{many} be made righteous.''

John 1, 29: ``Behold the Lamb of God, which taketh away the sins of the \emph{world}.'' (3, 16; 12, 47.)

Gen.~22, 18: ``In thy seed shall \emph{all nations of the earth} be blessed.''

Luke 2, 10: ``I bring you good tidings of great joy, which shall be to \emph{all people}.''

2 Pet. 3, 9: ``The Lord is not willing that any should perish, but that all should come to repentance.'' (Rom. 11, 32.)

\begin{center}
\textsl{Lutheran Testimony:--}
\end{center}

``The first and chief articles is this, that Jesus Christ, our God and Lord, died for our sins, and was raised again for our justification. Rom. 4, 25. And He alone is the Lamb of God, which taketh away the sins of the world (John 1, 29); and God has laid upon Him the iniquities of us all. Is. 53, 6.'' -- \emph{Smalc. Art.}, II, 1. 2.

\begin{center}
\textsc{Error.}
\end{center}

\emph{Presbyterians, Congregationalists}, and \emph{Calvinistic Baptists} state in their confessions that Christ has redeemed the elect \emph{only}. In the \emph{Westminster Confession} we read: ``As God hath appointed the elect unto glory, He, by the eternal and most free purpose of His will, foreordained all the means thereunto\ldots. \emph{Neither are any other redeemed by Christ, effectually called}, justified adopted, sanctified, and saved, but the elect only.'' -- (III, § 6.) And again: ``The Lord Jesus, by His perfect obedience and sacrifice of Himself, \ldots{} hath fully satisfied the justice of His Father, and purchased not only reconciliation, but an everlasting inheritance in the kingdom of heaven, \emph{for all those whom the Father hath given unto Him}'' (\emph{i. e.}, for the elect \emph{only}).--\emph{Westm. Conf.}, VIII, 5. (See also \emph{Savoy Declaration} and \emph{Bapt. Conf}. of 1688.)

\begin{center}
\textsl{But God says:--}
\end{center}

2 Cor. 5, 15: ``He died \emph{for all}, that they which live should not henceforth live unto themselves, but unto Him which died for them and rose again.''

Matt. 18, 11: ``The Son of Man is come to save \emph{that which was lost}.''

1 Tim. 4, 10: ``We trust in the living God, who is the Savior of \emph{all men}, specially of those that believe.''

1 John 2, 1. 2: ``If any man sin, we have an Advocate with the Father, Jesus Christ the Righteous; and He is the propitiation for our sins, and not for ours only, but also for the sins of the \emph{whole world}.'' (rom. 5, 10; Is. 53.)

2 Pet. 2, 1: ``Denying the Lord that brought them, and bring upon themselves swift destruction.'' (Christ died also for those who by reason of their unbelief are finally lost.)

\hypertarget{conversion}{%
\section{Conversion}\label{conversion}}

\begin{center}
\textsc{Bible Doctrine.}
\end{center}

``Conversion is not an effect of man's own power and strength, but an act and gift of the Holy Spirit.'' -- \emph{Chemnitz's Enchiridion}.

Acts 11, 18: ``God hath also to the Gentiles granted repentance unto life.''

Phil. 2, 13: ``For it is God which worketh in you both to will and to do of His good pleasure.''

Heb. 13, 20. 21: ``Now the God of peace \ldots{} make you perfect in every good work to do His will, working in you that which is well-pleasing in His sight.''

John 15, 5: ``Without Me ye can do nothing.''

2 Cor. 3, 5: ``Not that we are sufficient in ourselves to think anything, as of ourselves, but our sufficiency is of God.''

\begin{center}
\textsl{Lutheran Testimony:--}
\end{center}

``In spiritual and divine things the intellect, heart, and will of the unregenerate man cannot, in any way, by their own natural power, understand, believe, accept, think, will, begin, effect, do, work, or concur in working anything, but they are entirely dead to good and corrupt; so that in man's nature, since the fall, there is, before regeneration, not the least spark of spiritual power remaining still present, by which, of himself, he can prepare himself for God's grace, or accept the offered grace, or, for and of himself, be capable of it, or apply or accommodate himself thereto, or, by his own powers, be able of himself, as of himself, to aid, do, work, or concur in working anything for his conversion, either entirely, or in half, or in even the least or most inconsiderable part, but he is the servant and slave of sin (John 8, 34; Eph. 2, 2; 2 Tim. 2, 26). Hence the natural free will, according to its perverted disposition and nature, is strong and active only with respect to what is displeasing and contrary to God.'' -- \emph{Form. Conc.}, Art. II, § 7.

\begin{center}
\textsc{Error.}
\end{center}

The so-called \emph{Free-will Baptists} tell by their very name what they teach about conversion. But not so the \emph{Methodists}. In section 8 of their \emph{Articles of Religion} they seem to disclaim the concurrence of the Holy Spirit, but their chief teachers are self-confessed \emph{Synergists}, and their whole evangelistic and revivalistic work is based on the old heretical belief in freedom of the will. John Wesley says in his \emph{Scripture Doctrine of Predestination}: ``Oh, you are a defender of the freedom of the will, you assume a free will in man! I assume nothing but what Scripture says, and that you ought to allow me to assume. I do not assume that a man has will or power to do anything good of himself, but by the grace of God we may do everything. We believe that the moment Adam fell, he had no more freedom of the will, but that when God, in His free grace, promised him and his seed a Savior, He then \emph{returned} to humanity a free will and power to accept the proffered salvation.'' Professor W. F. Warner, in his \emph{Systematic Theology}, speaks in similar terms. Also T. N. Ralston, in his \emph{Outlines of Theology}. \emph{The Cumberland Presbyterians} also err here. In their \emph{Origin and Doctrine} we read: ``We object to the idea that man is passive in matters pertaining to salvation until God revives and renews him.'' (p.~86.)

\begin{center}
\textsl{But God says:--}
\end{center}

Jas. 1, 18: ``\emph{Of His own will} begot He us with the Word of Truth, that we should be a kind of first-fruits of His creatures.''

1 Cor. 12, 3: ``No man can say that Jesus is the Lord \emph{but by the Holy Ghost}.''

Jer. 13, 23: ``Can the Ethiopian change his skin, or the leopard his spots? Then may ye also do good that are accustomed to do evil.''

Matt. 7, 16: ``Do men gather grapes of thorns or figs of thistles?'' (12, 34.)

\textsc{Free Will.--} We often hear this expression: ``We have a free will, and can do as we please.'' To this the \emph{Augsburg Confession} makes this reply: ``Concerning free will they teach that man's will has some liberty to work a civil righteousness, and to choose between things that are subject to human reason, but that it has no power to work the righteousness of God, or a spiritual righteousness, without the Spirit of God, because that the natural man receives not the things of the Spirit of God, 1 Cor. 2, 14. But this is wrought in the heart when man receives the Spirit of God through the Word.'' -- Art XVIII.

\hypertarget{predestination}{%
\section{Predestination}\label{predestination}}

\begin{center}
\textsc{Bible Doctrine.}
\end{center}

\begin{enumerate}
\def\labelenumi{\alph{enumi}.}
\tightlist
\item
  God has shown mercy to all mankind, and wishes all to be saved.
\end{enumerate}

1 Tim. 2, 4: ``God will have all men to be saved, and to come unto the knowledge of the truth.''

2 Pet. 3, 9: ``The Lord is not willing that any should perish, but that all should come to repentance.''

Ezek. 33, 11: ``As I live, saith the Lord God, I have no pleasure in the death of the wicked, but that the wicked turn from his way and live.'' (18, 33; John 3, 16.)

\begin{center}
\textsl{Lutheran Testimony:--}
\end{center}

``Therefore, if we wish with profit to consider our eternal election to salvation, we must ine very way hold rigidly and firmly to this, \emph{viz}., that, as the preaching of repentance, so also the promise of the Gospel is universal, \emph{i. e.}, it pertains to all men (Luke 24, 47). Therefore Christ has commanded `that repentance and remission of sins should be preached in His name among all nations.' For God so loved the world, and gave His Son (John 3, 16). Christ bore the sins of the world (John 1, 29), gave His flesh for the life of the world (John 6, 51); His blood is the propitiation for the sins of the whole world (1 John 1, 7; 2, 2),'' etc. -- \emph{Form. Conc.}, Art. XI, 28.

\begin{center}
\textsc{Error.}
\end{center}

To be a \emph{Calvinsit} is to believe what has been properly called the ``monstrous doctrine'' that God does \emph{not} desire the salvation of \emph{all} mankind. IN his \emph{Institutes} Calvin declares that ``by predestination we mean the eternal decree of God, by which He determined with Himself whatever He wished to happen with regard to every man. All are not created on equal terms, but some are preordained to eternal life, \emph{others to eternal damnation}; and, accordingly, as each has been created for one or other of these ends, we say that he has been predestined to life \emph{or death}.'' (III, 21, 5.) ``He arranges all things by His sovereign counsel, in such a way that individuals are born who are doomed from their inception to certain death, and are to glorify Him by their destruction.'' (III, 23, 6.)

\begin{center}
\textsl{But God says:--}
\end{center}

Rom. 11, 32: ``God hath concluded them \emph{all} in unbelief, that He might have mercy \emph{upon all}.''

1 John 2, 2: ``And He is the propitiation for our sins, and not for ours only, but also for the sins of the \emph{whole world}.''

Matt. 23, 37: ``O Jerusalem, Jerusalem, \ldots{} how often would I have gathered thy children together, even as a hen gathereth her chickens under her wings, and \emph{ye would not}.''

\begin{center}
\textsc{Bible Doctrine.}
\end{center}

\begin{enumerate}
\def\labelenumi{\alph{enumi}.}
\setcounter{enumi}{1}
\tightlist
\item
  In choosing His elect, God did not proceed in an arbitrary absolute manner, but chose them in Christ Jesus.
\end{enumerate}

Eph. 1, 3-6: ``Blessed be the God and Father of our Lord Jesus Christ, who hath blessed us with all spriitual blessings in heavenly places in Christ; according as He hath chosen us \emph{in Him} before the foundation of the world, that we should be holy and without blame before Him in love; having predestinated us unto the adoption of children by Jesus Christ to Himself, according to the good pleasure of His will, to the praise of the glory of His grace, wherein He hath \emph{made us accepted in the Beloved}.''

\begin{center}
\textsl{Lutheran Testimony:--}
\end{center}

``Therefore this eternal election of God is to be considered in Christ, and not beyond or without Christ. For `in Christ,' testifies the Apostle Paul (Eph. 1, 4 sq.), `He hath chosen us before the foundation of the world,' as it is written: `He hath made us accepted in the Beloved.'\,'' -- \emph{Form. Conc.}, XI, § 65 (5. 9).

\begin{center}
\textsc{Error.}
\end{center}

\begin{enumerate}
\def\labelenumi{\arabic{enumi}.}
\item
  The fountainhead of modern error concerning this doctrine is \emph{John Calvin}. To him predestination was the central doctrine of the Christian religion. And more, -- predestination he considers strictly an arbitrary act of God. ``I repeat,'' he says, ``that we must ever return to the mere pleasure of the divine will.'' -- \emph{Iinst.}, III, 23, 4.
\item
  The \emph{Reformed Church}, specially so called, expresses its belief thus: ``If we are not to be ashamed of the Gospel, we must confess what it clearly teaches, that God, according to His eternal pleasure and moved by no other cause, has ordained some unto salvation, whereas others were rejected.'' -- \emph{Cons. Genev.}, 224.
\item
  \emph{Congregationalists} and \emph{Baptists}, in their respective confessions, agree with the \emph{Presbyterians} when the latter say: ``By the decree of God, for the manifestation of His glory, some men and angels are predestined unto everlasting life, and others foreordained to everlasting death.'' -- \emph{Westm. Conf.}, III, 3, 7.
\end{enumerate}

\textsc{Note.} -- In 1887 and 1903, the Presbyterian Church (North) added two chapters to their confession (34, 35) and also appended a \emph{Declaratory Statement} concerning predestination. According to this ``Statement,'' Chapter III is to be understood as \emph{not} teaching foreordination to death; and Chapter X, 3, that no children dying in infancy are lost. But as long as the confession is still in force, -- and all did not vote for the ``Statement,'' -- the situation is practically unchanged.

\begin{center}
\textsl{But God says:--}
\end{center}

Hos. 3, 9: ``O Israel, thou hast destroyed thyself, but in Me is thine help.''

John 1, 39: ``Behold the Lamb of God, which taketh away the sins of \emph{the world}.''

Titus 2, 11: ``For the grace of God that bringeth salvation hath appeared to \emph{all men}.'' (2 Pet. 2, 1.)

\textsc{Note.} -- The error of Calvinism does not only consist in denying an election \emph{in Christ}, but especially in teaching an \emph{election to perdition}. The Lutheran Church teaches, according to Scripture, that there is an election unto salvation, but \emph{not} unto condemnation. The \emph{Formula of Concord} (XI, 5) says: ``But the eternal election of God, or predestination, \emph{i. e.}, God's appointment to salvation, pertains not at the same time to the godly and the wicked, \emph{but only to the children of God}.''

On an other phase of this mysterious question we will let \emph{Dr.~Martin Chemintz} speak. He was not only the author of the Eleventh Article in the \emph{Formula of Concord}, but, at the time of its composition, superintendant of the churches of Braunschweig. In his \emph{Enchiridion}, a work written especially for the instructino of the clergy in his diocese, he puts the following question: ``Does God make such election first after a time, as when men have become penitent and believers, or is it made on account of their piety which God foresees?'' \emph{Answer}: ``St.~Paul says in Eph. 1, 4: `He hath chosen us in Christ before the foundation of the world'; and in 2 Tim. 1, 9: `He hath saved us, and called us with an holy calling, not according to our works, but according to His own purpose and grace, which was given us in Christ Jesus before the world began.' Hence, the electino of God does not follow after our faith and righteousness, but goes before them, as a cause of it all. Moreover, whom He did predestinate, them He also called; and whom He called, them He also justified, Rom. 8, 30. And in Eph. 1, 4, Paul does not say that we were elected because we \emph{were} holy or had \emph{become} holy, but He says: He hath chosen us that we should \emph{be} holy. For predestination is a cause of everything pertaining to salvation, as Paul says: `We have obtained an inheritance, being predestinated according to the purpose of Him who workth all things after the counsel fo His own will, that we should be to the praise of His glory, who first trusted in Christ.' (Eph. 1, 11. 12.) And this election is not made in view of our present or future deeds, but according to His own purpose and grace. (Rom. 9; 2 Tim. 1.)''

\hypertarget{justification}{%
\section{Justification}\label{justification}}

\begin{center}
\textsc{Bible Doctrine.}
\end{center}

Rom. 8, 33: ``Who shall lay anything to the charge of God's elect? It is God that justifieth.''

Gal. 3, 6: ``Even as Abraham believed God, and it was accounted to him for righteousness.''

Eph. 2, 8. 9: ``For by grace are ye saved, through faith, and that not of yourselves, it is the gift of God; not of works, lest any man should boast.''

Rom. 3, 24: ``Being justified freely by His grace through the recemption that is in Christ Jesus.'' 3, 28: ``Therefore we conclude that a man is justified by faith, without the deeds of the Law.''

\begin{center}
\textsl{Lutheran Testimony:--}
\end{center}

``Men cannot be justified before God by their own strength, merits, or works, but are freely justified for Christ's sake, through faith, when they believe that they are received into favor, and that their sins are forgiven for Christ's sake, who, by His death, hath made satisfaction for our sins. This faith God imputes for righteousness in His sight. (Rom. 3 and 4.)'' -- \emph{Augsb. Conf.}, Art. IV.

``This article concerning justification by faith is the chief in the entire Christian doctrine, without which no poor conscience has any firm consolation, or can know aright the riches of the grace of Christ, as Dr.~Luther also has written: `If only this article remain in view pure, the Christian Church also remains pure, and is harmonious and without all sects; but if it do not remain pure, it is not possible to resist any error or fanatical spirit.' And concerning this article Paul especially says that a little leaven leaveneth the whole lump. Therefore in this article he emphasizes with so much zeal and earnestness the exclusive particles, or the word whereby the works of men are excluded (namely, `without Law,' `without works,' `out of grace' `freely,' Rom. 3, 28; 4, 5; Eph. 2, 8. 9), in order to indicate how highly necessary it is that in this article, by the side of the presentation of the pure doctrine, the antithesis, \emph{i. e.}, all contrary dogmas, by this means be separated, exposed, and rejected\ldots{}

``Concerning the righteousness of faith before God we unanimously believe, teach, and confess, according to the comprehensive summary of our faith and convession above presented, \emph{viz.}, that a poor sinful man is justified before God, \emph{i. e.}, absolved and declared free and exempt from all his sins and from the sentence of a well-delivered condemnation, and adopted into sonship and heirship of eternal life, without any merit or worth of his own, also without all preceding, present, or subsequent works, out of pure grace, alone because of the sole merit, complete obedience, bitter suffering, death, and resurrection of our Lord Christ, whose obedience is reckoned to us for righteousness.'' -- \emph{Form. Conc.}, III, 6. 7. 9.

\begin{center}
\textsc{Error.}
\end{center}

\begin{enumerate}
\def\labelenumi{\arabic{enumi}.}
\tightlist
\item
  The \emph{Catholic Church}, the mother of heresies regarding justification, teaches according to its apologist, J. Adam Moehler: ``The Council of Trent represents justification as a renewal of the inward man, by means whereof we become really just, as inherent in the believer, and as a restoration of the primeval state of humanity. On this account, the same council observes that by the act of justification, faith, hope, and charity are infused into the heart of man; and that it is only in this way he is truly united with Christ and becometh a living member of His body. In other words, justification is considered to be sanctification and forgiveness of sins, as the latter is involved in the former, and the former in the latter; it is considered an infusion of the love of God into our hearts through the Holy Spirit; and the interior state of the justified man is regarded as holy feeling, as a sanctified inclination of the will, as habitual pleasure and joy in the divine Law, as a decided and active disposition to fulfil the same in all the occurrences of life, -- in short, as a way of feeling which is in itself acceptable and well pleasing to God.'' -- \emph{Symbolism}, p.~104.
\end{enumerate}

\textsc{Note.} -- It was on account of this grand confusion of ``infusion,'' ``sanctification,'' ``feeling,'' and ``inclination'' that Luther could find no ``joy in the divine Law'' nor peace in the Catholic Church. This came to him only after learning the meaning of Paul's words: ``Justified by faith, we have peace with God through our Lord Jesus Christ.'' (Rom. 5, 1.)

\begin{enumerate}
\def\labelenumi{\arabic{enumi}.}
\setcounter{enumi}{1}
\tightlist
\item
  To receive salvation through grace alone is repugnant to self-seeking, guilt-feeling human nature. Hence not only the pope, but many others have invented schemes of salvation which also include more or less of human efforts. As representatives of these may be mentioned the \emph{Arminians}. Their celebrated teacher Limborch writes: ``Be it known that we, when we state that we are justified through faith, do not exclude works, but rather include them.'' -- \emph{Chr. Theol.}, VI, 4, 22.
\end{enumerate}

The \emph{Quakers}, representing more nearly our every-day, mushroom religious reformers, teach ``that we cannot, as some Protestants unwisely have done, exclude good works from justification. They are absolutely necessary, forsooth, not as a case on account of which we are justified, but as something \emph{in} which we are justified, and without which we cannot be justified.'' -- \emph{Barclay}, \emph{Apol.}, 7, 3. 4.

\textsc{Note.} -- Theoretically, the Reformed sects confess justification by faith alone, but to a large extent their every-day religion is Arminianism pure and simple.

\begin{center}
\textsl{But God says:--}
\end{center}

Gal. 2, 16: ``Knowing that a man is not justified by works of the Law, but by the faith of Jesus Christ.''

Rom. 4, 4. 5: ``Now to him that worketh is the reward not reckoned of grace, but of debt. But to him that worketh not, but believeth on Him that justifieth the ungodly, his faith is counted for righteousness.''

Rom. 11, 6: ``And if by grace, then is it no more of works; otherwise grace is no more grace. But if it be of works, then it is no more grace; otherwise work is no more work.''

Rom. 5, 1: ``Being justified by faith, we have peace with God through our Lord Jesus Christ.''

Rom. 1, 17: ``The just shall live by faith.''

Gal. 3, 26: ``For ye are all the children of God by faith in Christ Jesus.''

\textsc{Note.} -- Though Christ expressly promises all those ``who hunger and thirst after righteousness that they shall be filled'' (Matt. 5, 6), yet the \emph{Catholic Church} would have its members dougt that any one can come to any certainty in the matter. No wonder ``good'' Catholics are willing to pay for indulgences and the reading of masses.

However, the \emph{Methodists} have a touch of the same sickness. Their revival preachers, especially, do not direct the occupants of ``the anxious seat,'' or ``mourners' bench,'' to the precious promises of the Lord in His Word, but to their own feeling. And this feeling is created by ``hard praying,'' shouting, and singing, until the ``subject'' hysterically shouts for joy. He is then declared ``saved.''

\textsl{But God says:--}

John 14, 27: ``Peace I leave with you, My peace I give unto you. Not as the world giveth, give I unto you. Let not your heart be troubled, neither let it be afraid.''

John 20, 29: ``Blessed are they that have not seen, and yet have believed.''

1 John 3, 20: ``For if our heart condemn us, God is greater than our heart, and knoweth all things.''

John 4, 46-53: The nobleman at Capernaum simply clung to the words of Christ: ``Go thy way; thy son liveth.'' ``And himself believed, and his whole house.''

\textsc{Note.} -- \emph{Calvinists} of whatever name (Presbyterian, Congregational, Baptist) do not only teach that Christ died only for the elect, but also that these alone receive the justifying faith, which they can never wholly lose.

This is emphatically disproved by the example of David and Peter. Both certainly fell from grace, and fell deeply, but were again restored.

Paul bemoans the fall of Demas, counsels earnestly the exclusion of the fornicator at Corinth, but recommends him to adoption again after he had become penitent.

\hypertarget{sanctification}{%
\section{Sanctification}\label{sanctification}}

\hypertarget{the-means-of-grace}{%
\section{The Means of Grace}\label{the-means-of-grace}}

\hypertarget{after-death}{%
\section{After Death}\label{after-death}}

\end{document}
