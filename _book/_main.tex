% Options for packages loaded elsewhere
\PassOptionsToPackage{unicode}{hyperref}
\PassOptionsToPackage{hyphens}{url}
%
\documentclass[
]{book}
\usepackage{amsmath,amssymb}
\usepackage{iftex}
\ifPDFTeX
  \usepackage[T1]{fontenc}
  \usepackage[utf8]{inputenc}
  \usepackage{textcomp} % provide euro and other symbols
\else % if luatex or xetex
  \usepackage{unicode-math} % this also loads fontspec
  \defaultfontfeatures{Scale=MatchLowercase}
  \defaultfontfeatures[\rmfamily]{Ligatures=TeX,Scale=1}
\fi
\usepackage{lmodern}
\ifPDFTeX\else
  % xetex/luatex font selection
\fi
% Use upquote if available, for straight quotes in verbatim environments
\IfFileExists{upquote.sty}{\usepackage{upquote}}{}
\IfFileExists{microtype.sty}{% use microtype if available
  \usepackage[]{microtype}
  \UseMicrotypeSet[protrusion]{basicmath} % disable protrusion for tt fonts
}{}
\makeatletter
\@ifundefined{KOMAClassName}{% if non-KOMA class
  \IfFileExists{parskip.sty}{%
    \usepackage{parskip}
  }{% else
    \setlength{\parindent}{0pt}
    \setlength{\parskip}{6pt plus 2pt minus 1pt}}
}{% if KOMA class
  \KOMAoptions{parskip=half}}
\makeatother
\usepackage{xcolor}
\usepackage{longtable,booktabs,array}
\usepackage{calc} % for calculating minipage widths
% Correct order of tables after \paragraph or \subparagraph
\usepackage{etoolbox}
\makeatletter
\patchcmd\longtable{\par}{\if@noskipsec\mbox{}\fi\par}{}{}
\makeatother
% Allow footnotes in longtable head/foot
\IfFileExists{footnotehyper.sty}{\usepackage{footnotehyper}}{\usepackage{footnote}}
\makesavenoteenv{longtable}
\usepackage{graphicx}
\makeatletter
\def\maxwidth{\ifdim\Gin@nat@width>\linewidth\linewidth\else\Gin@nat@width\fi}
\def\maxheight{\ifdim\Gin@nat@height>\textheight\textheight\else\Gin@nat@height\fi}
\makeatother
% Scale images if necessary, so that they will not overflow the page
% margins by default, and it is still possible to overwrite the defaults
% using explicit options in \includegraphics[width, height, ...]{}
\setkeys{Gin}{width=\maxwidth,height=\maxheight,keepaspectratio}
% Set default figure placement to htbp
\makeatletter
\def\fps@figure{htbp}
\makeatother
\setlength{\emergencystretch}{3em} % prevent overfull lines
\providecommand{\tightlist}{%
  \setlength{\itemsep}{0pt}\setlength{\parskip}{0pt}}
\setcounter{secnumdepth}{5}
\ifLuaTeX
  \usepackage{selnolig}  % disable illegal ligatures
\fi
\IfFileExists{bookmark.sty}{\usepackage{bookmark}}{\usepackage{hyperref}}
\IfFileExists{xurl.sty}{\usepackage{xurl}}{} % add URL line breaks if available
\urlstyle{same}
\hypersetup{
  pdftitle={The Difference},
  pdfauthor={I.G. Monson},
  hidelinks,
  pdfcreator={LaTeX via pandoc}}

\title{The Difference}
\usepackage{etoolbox}
\makeatletter
\providecommand{\subtitle}[1]{% add subtitle to \maketitle
  \apptocmd{\@title}{\par {\large #1 \par}}{}{}
}
\makeatother
\subtitle{A Popular Guide to Denominational History and Doctrine}
\author{I.G. Monson}
\date{1915}

\begin{document}
\maketitle

{
\setcounter{tocdepth}{1}
\tableofcontents
}
\hypertarget{preface}{%
\chapter*{Preface}\label{preface}}

What is the difference between the various Christian sects? The common answer is that there is no difference, that they all have the same Bible, that all are trying to reach the same place, etc. For the ignorant this is the easiest way to get away from this question. The lazy, indifferent, and the unbeliever will find the same answer the most convenient. But any one interested in truth and in Christianity will gladly accept information concerning the most important questions in life.

\hypertarget{part-i.}{%
\chapter*{PART I.}\label{part-i.}}
\addcontentsline{toc}{chapter}{PART I.}

\hypertarget{origin-of-divided-protestantism}{%
\section{Origin of Divided Protestantism}\label{origin-of-divided-protestantism}}

When Luther in Germany opposed Roman Catholicism and popery, it was because the Roman Church had perverted the Gospel of Christ, taught righteousness by works, and prohibited the circulation of the Bible among the people. Strive as he would, his troubled soul could find no consolation in following the precepts of the Church. It was only after years of fruitless toil, in and out of the cloister, that he had at last found peace and consolation in the precious promise of the Lord: ``The just shall live by his faith'' (Hab. 2, 4)

\hypertarget{historical-summaries}{%
\section{Historical Summaries}\label{historical-summaries}}

\hypertarget{the-anglican-church}{%
\subsection{The Anglican Church}\label{the-anglican-church}}

Many dignitaries of the Anglican Church do not, at the present time, like to date the beginning of their denomination from the time of the Reformation. And some would like to ignore the Reformation altogether. Says one: ``England has never had but one religion; it has never changed that religion, and it has that religion still.''

\hypertarget{unchristian-cults}{%
\section{Unchristian Cults}\label{unchristian-cults}}

How many sects there are in this country alone no one knows, not even the Director of the Census. Catholics would fain tell us that this state of affairs dates back only to the time of the Reformation. There may, at the present time, be more sects than formerly, but divisions in the Church have always existed, and the Pope of Rome has not only had his hands full in pacifying and judging between warring factions of monks and their so-called ``eminent teachers,'' but he has constantly had on hand a war of extermination against dissenters, who would not acquiesce in his teachings. The attitude of the pope towards the Modernists of our own day is only the old story over again. But their silence after his dondemnation is the slence of the dead, -- and of the same value.

\hypertarget{part-ii.-doctrines-of-the-church}{%
\chapter*{PART II. Doctrines of the Church}\label{part-ii.-doctrines-of-the-church}}
\addcontentsline{toc}{chapter}{PART II. Doctrines of the Church}

\hypertarget{concerning-the-truth-of-the-christian-religion}{%
\section{Concerning the Truth of the Christian Religion}\label{concerning-the-truth-of-the-christian-religion}}

\hypertarget{the-holy-scriptures}{%
\section{The Holy Scriptures}\label{the-holy-scriptures}}

\hypertarget{original-sin}{%
\section{Original Sin}\label{original-sin}}

\hypertarget{redemption}{%
\section{Redemption}\label{redemption}}

\hypertarget{conversion}{%
\section{Conversion}\label{conversion}}

\hypertarget{predestination}{%
\section{Predestination}\label{predestination}}

\hypertarget{justification}{%
\section{Justification}\label{justification}}

\hypertarget{sanctification}{%
\section{Sanctification}\label{sanctification}}

\hypertarget{the-means-of-grace}{%
\section{The Means of Grace}\label{the-means-of-grace}}

\hypertarget{after-death}{%
\section{After Death}\label{after-death}}

\end{document}
